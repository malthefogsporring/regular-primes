\documentclass[12pt,oneside]{report} 
\usepackage[y4project]{edmaths}
\flushbottom
%\documentclass[titlepage,12pt,a4paper]{article}
%\documentclass[12pt]{report} 
%\usepackage[y4project]{edmaths} 
\usepackage[utf8x]{inputenc}
\usepackage[T1]{fontenc}
\usepackage{mathptmx} % Use Times Font
%\usepackage{listings}
\usepackage{amsmath}
\usepackage{mathtools}
\usepackage{amssymb}
\usepackage{tikz-cd}
\usepackage{tikz}
\usepackage{circuitikz}
\usepackage{hyperref}
\usepackage{graphicx,caption}
\usepackage{url}
\usepackage{amsthm}
\usepackage[capitalise]{cleveref}

\usepackage{enumerate}
%\usepackage{mathtools}
%\usepackage{theoremref}
\usepackage{amsthm}
\newcounter{thm}
\numberwithin{thm}{section}

\tikzcdset{scale cd/.style={every label/.append style={scale=#1},
    cells={nodes={scale=#1}}}}
\newcommand\norm[1]{\left\lVert#1\right\rVert}

\theoremstyle{plain}
\newtheorem{theorem}[thm]{Theorem}
\newtheorem{corollary}[thm]{Corollary}
\newtheorem{lemma}[thm]{Lemma}
\newtheorem{prop}[thm]{Proposition}
\newtheorem{question}[thm]{Question}
\newtheorem{proposition}[thm]{Proposition}

\theoremstyle{definition}
\newtheorem{definition}[thm]{Definition}
%\theoremstyle{remark}
\newtheorem{remark}[thm]{Remark}
\newtheorem{example}[thm]{Example}
\newtheorem{warning}[thm]{Warning}
%\setlength{\parindent}{0em}
\bibliographystyle{amsplain}

\newcommand{\dsum}{\oplus}
\newcommand{\iso}{\cong}
\newcommand{\homeo}{\cong}
\newcommand{\homotopic}{\simeq}
\newcommand{\defn}[1]{\textbf{#1}}
\newcommand{\cat}[1]{\mathcal{#1}}
\newcommand{\RP}[1]{\mathbb{RP}^{#1}}
\newcommand{\CP}[1]{\mathbb{CP}^{#1}}
\newcommand{\HP}[1]{\mathbb{HP}^{#1}}
\newcommand{\Top}[1]{\mathbf{Top}_{#1}}
\newcommand{\Mod}[1]{\ \mathrm{mod}\ #1}
\newcommand{\Zmod}{\mathbb{Z}/2\mathbb{Z}}
\newcommand*\bigcdot{\mathpalette\bigcdot@{.5}}
\newcommand{\Z}{\mathbb{Z}}
\newcommand{\C}{\mathbb{C}}
\newcommand{\Q}{\mathbb{Q}}
\newcommand{\A}{\mathbb{A}}
\newcommand{\R}{\mathbb{R}}
\newcommand{\N}{\mathbb{N}}
\newcommand{\ideal}{\vartriangleleft}

\newcommand{\matricno}{s1745523}
\newcommand{\projectno}{UG771}
\title{Kummer's Theory of Regular Primes}
\author{Malthe Fog Sporring}
\date{\today}



\begin{document}
\pagenumbering{roman}
\maketitle

\begin{abstract} Fermat's Last Theorem says that the equation $x^n+y^n=z^n$ has no integer solutions for $n>2$. In 1847, Ernst Kummer proved the theorem for a certain type of prime exponents, called \textit{regular primes}. We give a historically motivated account of his proof in the modern language of ideals. In particular, we show how Fermat's Last Theorem can be reformulated as a multiplicative problem in the ring $\Z[\omega]$ where $\omega=e^{2\pi i/p},$ for a prime $p$. The result is then reduced to proving that (I) the ideals of $\Z[\omega]$ factorise uniquely into \textit{prime ideals}, and (II) the equivalence classes of ideals under a certain equivalence relation is a finite abelian group. These properties are shown to follow from the fact that $\Z[\omega]$ is (I) a \textit{Dedekind domain} and (II) a \textit{number ring}.
\end{abstract}
\declaration
\dedication{To Phoebe.}
%\let\chapter\section\tableofcontents
\tableofcontents
%\addcontentsline{toc}{chapter}{Contents}
\newpage
\addcontentsline{toc}{chapter}{Acknowledgements}
\chapter*{Acknowledgements}
I want to thank my supervisor, Professor James Wright, for his guidance and advice throughout this project. I also wish to express my gratitude to my parents and friends for their constant encouragement and support.

\newpage\pagenumbering{arabic}
\chapter{Introduction}
In theorems that are easy to state and difficult to prove, Fermat's Last Theorem likely takes the crown. The Theorem, first stated in 1637, was never proven by Fermat, although he claimed to have a "a truly marvelous proof" (CITE), which the margins of his copy of Arithmetica were "too narrow to contain." It is very doubtful that he had such a proof - indeed, many mathematicians since then have thought they had a proof, only to find a slight error making the entire proof crumble. A correct proof was not published until 1995, by Sir Andrew Wiles, and this proof was an epic achievement bringing together our most advanced understanding of elliptic curves (CITE). Despite the difficulty of the proof, Fermat's Last Theorem is so easy to state, a 9-year-old can understand it.

\begin{theorem}[Fermat]
The equation
\begin{equation}\label{Fermat}
    x^n+y^n=z^n
\end{equation}
has no integer solutions for $n>2$.
\end{theorem}

The case $n=2$ is Pythagoras' Theorem, which has integer solutions. For example, $(x,y,z)=(3,4,5)$ works. It is enough to prove Fermat's Last Theorem for the case that $n$ is an odd prime. This is because a counter-example for $n=pm$ a composite number also gives a counter-example for a prime:
$$(x^m)^p+(y^m)^p=(z^m)^p.$$
(Technically, we should also prove the case $n=4$, as we know Fermat does not work for $n=2$ - however this turns out not to be hard). We can additionally assume that in a solution $(x,y,z)$, $x$, $y$ and $z$ are all coprime. Indeed, if they all three share a factor, we can divide by that factor to get another counter-example. If instead, two of three share a factor $n$, then $\cref{Fermat}$ will not hold $(\Mod n)$. We therefore reduce the proof to two types of solutions:
\begin{itemize}
    \item (type 1): $p$ does not divide either of $x,y,z$.
    \item (type 2): $p$ divides exactly one of $x,y,z$.
\end{itemize}

The case $n=3$ is simple enough to prove using just undergraduate level number theory. First, suppose $(x,y,z)$ is a type 1 solution. Then $x,y,z$ are all $\pm1 \Mod 3$, so $$x^3+y^3\neq z^3 (\Mod 3).$$
(FINISH PROOF)

(PROVE n=4 - exercise 15 on pg 8 in Number Fields).

These cases are truly the only trivial ones - proving Fermat's Theorem for $p>3$ requires more work. Type 1 solutions turn out to be the most interesting to analyse, so from now on we will assume $(x,y,z)$ is a Type 1 solution. Our first trick will be to turn $\cref{Fermat}$ into a multiplicative equation. Letting $\omega=e^{2\pi i/p}$, we factor \cref{Fermat} to
$$(x+y)(x+\omega y)\dots(x+\omega^{p-1} y)=z^p.$$
This follows from noting $1,\omega,\dots,\omega^{p-1}$ are unique (NOTE: This is what makes the $p$ case work more easily than the $n$ case!!!) solutions to $t^p-1=0$, and since $\mathbb{C}$ is algebraically closed, this implies
\begin{equation}\label{eq-omega-polynomial}t^p-1=(t-1)(t-\omega)\dots (t-\omega^{p-1}).\end{equation}
Replacing $t$ with $\frac{-x}{y}$ gives
$$x^p+y^p=(x+y)(x+\omega y)\dots (x+\omega^{p-1}y)$$
as required. This reformulation allows us to use the full artillery of ring theory on Fermat's Last Theorem.
Gabriel Lamé believed he had a proof, which relied on the assumption that the ring of polynomials $\mathbb{Z}[x]$ is a unique factorisation domain (UFD): (FINISH)

\section{Lamé's false proof}

\section{Kummer's partial rectification}

\chapter{Rings and ideals}\label{sec-rings}
Kummer's proof predated the very concept of an ideal: he worked with what he called "ideal numbers". These were later put on a solid footing by Dedekind when he invented the notion of ideals. Ideals are such a key part of modern algebra, it can be difficult to imagine a time before their invention. In fact, Kummer's story tells so well in the language of Dedekind that we will exclusively tell this side of the story. To this end, we spend this chapter establishing preliminary results about ideals, ring modules, and Noetherian rings, inspired by the account in \cite{Wright}. Some of these results will be revision from undergraduate level algebra. All rings will be assumed to be commutative with multiplicative identity, which ring homomorphisms are assumed to preserve.
\section{Ideals}
Recall the following definitions.
\begin{definition}
An \defn{ideal} $I$ of $R$ is a non-empty subset satisfying the following properties:
\begin{enumerate}[(i)]
    \item $I$ is closed under subtraction.
    \item When $i\in I$, for any $r\in R$, $ri\in I$.
\end{enumerate}
\end{definition}
\begin{definition}
Given any subset $S\subset R$, the \defn{ideal generated by} $S$, denoted $\langle S\rangle$, is the smallest ideal of $R$ containing $S$. It is exactly the set $\{r_1s_1+\dots r_ns_n: r_i\in R, s_j\in S\}$ of finite linear combinations of elements in $S$. A \defn{principal ideal} is an ideal generated by a single element.
\end{definition}

\begin{remark}
Since our rings are commutative, we do not need to distinguish between left ideals and right ideals.
\end{remark}

\begin{example}\label{Z-is-PID}
For any $n\in \Z$, $\langle n\rangle =n\Z$ is an ideal of the integral domain $\Z$. In fact, every ideal of $\Z$ is of this form. Take any nonzero ideal $I$ of $\Z$ and let $n$ be the smallest positive integer in $I$. Then given any other $m\in I$, the division and remainder algorithm gives two integers $q,r$ with $m=qn+r$ and $0\leq r<n$. It follows that $r\in I$. However, $n$ was minimal among positive integers in $I$, so in fact $r=0$ and $m\in \langle n \rangle$. Since $m$ was arbitrary, $I=\langle n \rangle.$ It follows that every ideal of $\Z$ is principal. An integral domain with this property is called a \defn{principal ideal domains} or \defn{PID}s for short. 
\end{example}
\begin{example}\label{F[x]-PID} Since the divison and remainder algorithm works in any polynomial ring in one variable over a field \cite{HonoursAlgebra}, an almost identical argument to \cref{Z-is-PID} shows that $F[x]$ is a PID for any field $F$. In particular, given an ideal $I$ of $F[x]$ we can pick a polynomial $f\in I$ of minimal degree. For any $g\in I$ we can find $q,r\in F[x]$ s.t $g=qf+r$ $deg(r)<deg(g)$ by the division and remainder algorithm. Therefore $r\in I$, and since $f$ has minimal degree in $I$, $deg(r)=0$ and $g\in \langle f\rangle$, so $I=\langle f \rangle.$
\end{example}

All rings lie on a sliding scale between being close to being a field and being far from being a field. A ring like the polynomial ring $\Z[x]$ is very far from being a field: the only units (multiplicatively invertible elements) are $\pm 1$. If we want to turn $\Z[x]$ into a field, we can try to declare some of the non-units to be zero by factoring out the ideal generated by these elements. Recall that the \defn{quotient} or \defn{factor ring} of a ring $R$ by an ideal $I$ is the ring of cosets $\{r+I:r\in R\}/\sim$ where $r+I\sim s+I\iff r-s\in I$ and addition and multiplication are defined in the canonical way. Factoring $\Z[x]$ by the ideal $\langle x^2\rangle$ is a bad choice since we miss out on the non-unit $x$, and $x+\langle x^2\rangle$ is still a non-unit in $\Z[x]/\langle x^2 \rangle.$ The failure is due to the fact $\langle x^2\rangle\subset \langle x \rangle$, as we now show: if $r$ is a non-unit and an ideal $I\subset \langle r \rangle$ is an ideal, then $r+I$ is a non-unit in $R/I.$ Indeed, for $s\in R$,
$$(r+I)(s+I)=1+I \iff rs-1\in I\subset \langle r \rangle \implies 1\in \langle r \rangle \iff \langle r \rangle = R\iff r \text{ is a unit }.$$
Therefore to form a field by factoring out an ideal we must factor by a \defn{maximal ideal}: a proper ideal $I$ only contained in $R$ and itself. For example, the maximal ideals of $\Z$ are $p\Z$ where $p$ is prime. We already know that $\Z/n\Z$ is a field if and only if $n$ is prime, so in the ring $\Z$ quotienting by a maximal ideal is not only a sufficient, but also a necessary condition for forming a field. We will show that this is true in general.

\begin{lemma}\label{field-iff-no-ideals}
A ring $R$ is a field if and only if it contains no non-trivial ideals.
\end{lemma}
\begin{proof}
If $R$ is a field and $I$ is an ideal containing $r$, then $1=rr^{-1}\in I$, so $I=R$. Conversely, if $R$ contains no trivial ideals, then for any $r\in R$, $\langle r \rangle = R$. In particular, $1\in \langle r \rangle$ so $1=sr$ for some $s\in R$.
\end{proof}

The correspondence theorem from group theory \cite{GroupTheory} is easily extended to a correspondence theorem for ideals, as we now show.
\begin{lemma}\label{ideal-bijection}[Correspondence Theorem]
For any ideal $I$ of $R$ there is a bijection between the set of ideals of $R$ containing $I$ and the set of ideals of $R/I$.
\end{lemma}
\begin{proof}
The correspondence theorem for a normal subgroup $I\triangleleft R$ says that the canonical group homomomorphism $can:R\rightarrow R/I$ defines a bijection $J\mapsto can(J)$ between subgroups of $R$ containing $I$ and subgroups of $R/I$. This theorem applies to ideals since ideals are always normal subgroups under $+$. $can$ also defines a ring homomorphism, and it is easy to see $J$ is an ideal if and only if $can(J)$ is - we already know $can$ takes abelian groups to abelian groups, so additionally if $J=rJ$ then $J+I=rJ+I$ by bijection on abelian groups and vice versa. Therefore, $J\mapsto can(J)$ is an injection between ideals $J$ containing $I$ and ideals in $R/I$. 
\end{proof} As an immediate consequence of \cref{field-iff-no-ideals} and \cref{ideal-bijection}, we get the following result.
%Recall the correspondence theorem for groups says if $$

%The bijection is given by $\phi:J\mapsto J/I$ where $J$ is an ideal of $R$ containing $I$ and $$J/I:=\{j+I:j\in J\}\subseteq R/I.$$ It is easily confirmed that $J/I$ is an ideal of $R/I$. In particular, it is closed under subtraction since $J$ is, and closed under multiplication by elements of $R/I$ since $J$ is closed under multiplication by elements of $R$.

%$\phi$ is injective: Suppose that $\phi(J)=\phi(S)$ for two ideals $J,S$ containing $I$. Let $j\in J$ and let $s\in S$ be any element such that $j+I=s+I$. Then $j-s\in I\subset S$, so $j\in S$. But $j$ was arbitrary, so $J\subset S$. A similar argument shows $S\subset J$ so $J=S$.
%$\phi$ is surjective: Let $X$ be an ideal of $R/I$. Then $X$ is a set of cosets and we can define %$$J:=\bigcup X.$$ I claim $J$ is an ideal of $R$. We check:
%$$x\in J\implies x+I,y+I\in X\implies (x-y)+I\in X\implies (x-y)\in I\subset J.$$
%$$x\in J\implies x+I\in X\implies (x+I)(r+I)\in X\implies rx+I\in X\implies rx\in J.$$
%Note $\phi(J)=X$, so $\phi$ is surjective. Therefore $\phi$ is bijective.
\begin{prop}\label{ring-over-max-field}
An ideal $I$ of $R$ is maximal if and only if $I/R$ is a field.
\end{prop}
\begin{proof}
$I$ is maximal $\iff$ the only ideals containing $I$ are $I$ and $R$ $\iff$ there are no nontrivial ideals in $R/I$ $\iff$ $R/I$ is a field.

%$(\impliedby)$ Suppose $R/I$ is a field, and $J$  is an ideal with $I\subset J$. Then $J/I:=\{jI: j\in J\}$ is an ideal of $R/I$. By the previous lemma, $J/I=R/I$, so $J=R$.
%$(\implies)$ Conversely, if $I$ is maximal, then $R/I$ does not contain any non-trivial ideals: a non-trivial ideal $JI\subset R/I$ gives rise to a non-trivial ideal $\langle J\cup I\rangle$ of $R$. By the previous lemma, $R/I$ is a field.
\end{proof}
%\begin{remark}
%Zorn's lemma, which is equivalent to the axiom of choice (citation needed), implies that every ring has a maximal ideal. The statement of Zorn's lemma is as follows: (finish this thought...)
%\end{remark}
What if we do not necessarily want to build a field, but are content with an integral domain? Recall that an integral domain is a ring $R$ with no zero divisors. It is exactly the condition that allows us to "cancel out" multiplicatively:
$$ab=ac\implies b=c$$ in an integral domain whenever $a\neq 0$. This is done not by multiplying both sides by $a^{-1}$ (which may not exist), but by noting the ring homomorphism $$f:R\rightarrow R,\quad  b\mapsto ab$$ has kernel $\{0\}$ and is therefore injective. To form an integral domain, an idea is to factor out the set $S$ of all zero divisors of a ring $R$. However, this set does not form an ideal as the difference of two zero divisors need not be a zero divisor. For example, $[2][3]=[0]$ in $\Z_6$ but $[3]-[2]=[1]$ is not a zero divisor. For the same reason, taking the ideal $I=\langle S \rangle$ generated by $S$ is not a good idea either: in the case of $\Z_6$ we would have $I=\Z_6$ whence $\Z_6/I$ is not even a ring. Instead, let us remark that $R/I$ is an integral domain if for $x,y\in R$,
$$(x+I)(y+I)=I\iff x\in I\text{ or } y \in I.$$
Noting $(x+I)(y+I)=I\iff xy\in I$, we can now write down exactly the condition $I$ needs to satisfy for $R/I$ to be an integral domain.

\begin{definition}
A \defn{prime ideal} $I$ of $R$ is a proper ideal such that $xy\in I\implies x\in I$ or $y\in I$. 
\end{definition}

\begin{example}\label{max-is-prime}
The prime ideals of $\mathbb{Z}$ are exactly $p\mathbb{Z}$ where $p$ is prime. If $xy\in p\mathbb{Z}$ then either $x$ or $y$ is divisible by $p$, so either $x$ or $y$ is in $p\mathbb{Z}$. Conversely, in $nm\mathbb{Z}$ where $n,m>1$, $nm\in nm\mathbb{Z}$ but $n,m\not \in nm\mathbb{Z}$. Thus in $\mathbb{Z}$, prime ideals and maximal ideals coincide. In general, a maximal ideal $I$ is always prime, since $I$ is maximal $\iff$ $R/I$ is a field $\implies R/I$ is an integral domain $\implies $ $I$ is prime. However, the converse is not always true. For example, $\langle x \rangle$ is a prime ideal of $\Z[x]$, but it is not maximal, since $$\langle x \rangle\subsetneq \langle x,2\rangle \subsetneq \Z[x].$$
\end{example}
It is a fun exercise to take any ring and try to build an integral domain out of it by factoring out a prime ideal. I give two more original examples.
\begin{example}
The set $C(\R,\R)$ of continuous functions $f:\R\rightarrow \R$ can be given a ring structure, with the usual addition and multiplication (not composition) of functions. The additive identity is the constant function $x\mapsto 0$ and the multiplicative identity is $x\mapsto x$. $C(\R,\R)$ is not an integral domain: take for example two continuous functions
$$f(x)=\begin{cases} 
      0 & x\leq 0 \\
      x & x>0 
   \end{cases}, \qquad g(x)=\begin{cases} 
      x & x\leq 0 \\
      0 & x>0 
   \end{cases}
$$
and note $f(x)g(x)=0$. Let us try to build a prime ideal of $C(\R,\R)$. Let $a\in \R$ and consider the ideal $I_a$ of all continuous functions $f\in C(\R,\R):f(a)=0$. This set is clearly closed by subtraction and multiplication by continuous functions. Additionally, if $f(a)g(a)=0,$ then $f(a)=0$ or $g(a)=0$, so $I_a$ is prime. It follows that $C(\R,\R)/I_a$ is an integral domain! The cosets $[f]$ are exactly the sets of functions $g$ that take the same value $f(a)=g(a)$ at $a$.
\end{example}

\begin{example}
Another strange ring\footnote{Found on \url{https://en.wikipedia.org/wiki/Ring_(mathematics)\#Commutative_rings}. Accessed 18-03-2022.} is constructed as follows: take any set $S$ and let $P(S)$ be the power set of $S$. Let multiplication be intersection $AB=A\cap B$ and let $A+B=(A\cup B)\setminus A\cap B$ be the \defn{symmetric difference}: $x\in A+B\iff x\in A$ or $x\in B$ but not both. Clearly $+$ and $\times$ are commutative, and we have additive identity $\emptyset$ and multiplicative identity $S$. Additive inverses are given by $-A=S\setminus A$. For the distributive law, note $x\in A(B+C)=A\cap (B\cup C/B\cap C)\iff x\in A$ and $x\in B$ or $C$ but not both $\iff$ $x\in A\cap B$ or $x\in A\cap C$ but not both $\iff x\in A\cap B + A \cap C=AB+AC$.

Now clearly $P(S)$ is not an integral domain: $A(S\setminus A)=\emptyset$ for any $A\subset A$, so every element is a zero divisor. Since $0\in P$ for any prime ideal $P$, prime ideals contain all zero divisors. Therefore $P(S)$ has no prime ideals, so $P(S)/I$ is not an integral domain for any $I$. It follows that $P(S)$ has no maximal ideals either. $P(S)$ is, however, a PID! Let $I$ be any ideal of $P(S)$, and note $I$ contains all the singletons $\{x\}$ such that $x\in B$ for some $B\in I$. This holds since $\{x\}=\{x\}B\in I$. Since $I$ is closed under $+,$ $I$ also contains the union $U=\{x:x\in B\text{ for some } B\in I\}$. Letting $A\in I$ we have $A=AU$, so $I\subset \langle U\rangle \subset I$. Therefore $I=\langle U\rangle$ is principal. A very strange ring indeed!
\end{example}

Now let us recall this definition given in \cref{sec-introduction}.
\begin{definition}
An invertible element $r\in R$ is called a \defn{unit}. An element $r\in R$ of an integral domain is called \defn{irreducible} if it is non-zero, not a unit and if $r=xy\implies x$ or $y$ is a unit. $r\in R$ is \defn{reducible} if it is neither $0$, a unit, or irreducible.
\end{definition}
Irreducible and reducible elements generalise the notion of, respectively, prime and composite numbers in $\mathbb{Z}$. The units in $\mathbb{Z}$ are $\pm 1$, and certainly if we allow primes to be negative, $p$ is prime if and only if $p=mn\implies m$ or $n=\pm 1$. In a field $F$ every nonzero element is a unit, so fields do not have (ir)reducibles. The next proposition motivates defining principal ideal domains and irreducible elements.

\begin{prop}\label{PID-over-irreducible-is-field}
If $R$ is a PID, then $0\neq r\in R$ is irreducible $\iff\langle r \rangle$ is maximal $\iff R/\langle r\rangle$ is a field.
\end{prop}
\begin{proof}
The second equivalence is just a restatement of \cref{ring-over-max-field}. We may assume $r$ is not a unit, as $r$ is a unit if and only if $\langle r\rangle=R$, and $R/R=\{0\}$ is not a field. 

$(\implies)$ Let $r$ be irreducible and suppose $\langle r\rangle \subset J$ for some proper ideal $J$. Since $R$ is a PID, $J=\langle s \rangle$ for some non-zero, non-unit $s\in R.$ Therefore $r=ts$ for some $t\in R$. Since $r$ is irreducible and $s$ is not a unit, $t$ is a unit. Therefore $s=t^{-1}r$ so $\langle s \rangle \subset \langle r \rangle.$ Since $J$ was arbitrary, $\langle r \rangle$ is maximal.

$(\impliedby)$ Now let $\langle r \rangle$ be maximal. Let $r=xy$ and suppose neither $x$ nor $y$ are units. It follows that $\langle r\rangle \neq \langle x \rangle$. Then $$\langle r\rangle \subsetneq \langle x \rangle \subsetneq R,$$ a contradiction. Therefore $r$ is irreducible.
\end{proof}

\begin{example}
In $\mathbb{Q}[x]$, $x^2+1$ is irreducible, since its roots are irrational so it cannot be written as a product of two linear polynomials in $\Q$. It follows that $\frac{\mathbb{Q}[x]}{\langle x^2+1\rangle}$ is a field. In fact, it is isomorphic to $\C$: taking the quotient by $x^2+1$ is enforcing the rule $x^2+1=0$ i.e. defining $x=\sqrt{-1}$. 
\end{example}

\begin{remark}
It is essential that $R$ is a PID. For example, $2$ is irreducible in $\mathbb{Z}[x]$ but $\frac{\mathbb{Z}[x]}{\langle 2\rangle}$ is not a field, as $[x]$ is not a unit. This reflects the fact that $\mathbb{Z}[x]$ is not a PID, take for example the non-principal ideal $\langle 2,x\rangle$.
\end{remark}

\section{Noetherian rings and modules}

\begin{definition}
A ring $R$ is said to be \defn{Noetherian} if every increasing sequence of ideals $$I_1\subset I_2\subset I_3 \subset \dots$$
terminates. 

We can extend this definition to $R-$Modules, recalling an $R-$module is an abelian group $M$ equipped with a (distributive, associative, identity-preserving) scalar multiplication function $R\times M\rightarrow M$. An $R-$module $M$ is \defn{Noetherian} if every increasing sequence of submodules $$N_1\subset N_2\subset N_3\subset \dots$$ terminates.
\end{definition}

Note that $R$ is trivially an $R-$module, and the submodules are exactly the ideals of $R$. Therefore $R$ is a Noetherian ring if and only if it is a Noetherian $R-$module. If $R$ is an $S-$module for some other ring $S$, then the story is more complicated: $R$ may be a Noetherian ring without being a Noetherian $S-$module - see \cref{warning-Noetherian-rings-and-mods}.

\begin{proposition}\label{noetherian-module-equivalences}
The following are equivalent for an $R$-module $M$.
\begin{enumerate}[(i)]
    \item $M$ is Noetherian.
    \item Every non-empty collection of submodules of $M$ has a maximal element.
    \item Every submodule $N$ of $M$ is finitely generated as an $R-$module.
\end{enumerate}
\end{proposition}
\begin{proof}
$(i)\implies (ii)$ easily: a counter-example to $(ii)$ is exactly a collection of submodules containing a non-terminating increasing sequence of ideals. Additionally, $(ii)\implies (i)$ because any increasing sequence of submodules
$$N_1\subseteq N_2\subseteq N_3 \subseteq \dots$$
gives a collection $\{N_i\}_{i\geq 1}$ of ideals, which then has a maximal element.

$(i)\implies (iii)$ a non-finitely generated submodule $N$ gives rise to a non-terminating sequence of submodules. Let $a_1\in N$ and $N_1=\langle a_1\rangle=\{ra_1:r\in R\}$. Then let $a_2\in N\setminus N_1$ and $N_2=\langle a_1,a_2\rangle$. Continue this process for all natural numbers to construct the increasing sequence of submodules. It will never terminate as $N$ is not finitely generated.

$(iii)\implies (i)$ Let $$N_1\subseteq N_2\subseteq \dots$$ be an increasing chain of ideals. Then $\bigcup N_i$ is a submodule of $M$, and therefore finitely generated, say by $\{a_1,\dots a_n\}$. Letting $m$ be the smallest integer such that $$\{a_1,\dots a_n\}\subset N_m,$$ then $N_m$ is a maximal element of the sequence.
\cite{Wright}
\end{proof}

\begin{example}
Any principal ideal domain is Noetherian, as all its ideals are generated by a single element.
\end{example}

\begin{remark}
Any ideal $I$ of a Noetherian ring $R$ is always contained in a maximal ideal: simply take the set of all proper ideals containing $I$. By \cref{noetherian-module-equivalences}, this set has a maximal element, which is then a maximal ideal. If one accepts Zorn's Lemma, then this statement is also true of general rings \cite{Wright}.
\end{remark}

We now give some propositions that allow us to generate Noetherian modules from Noetherian rings. First, a result by Hilbert which relies on the axiom of choice.

\begin{theorem}[Hilbert's basis theorem]\label{Hilbert-basis-theorem}
If $R$ is a Noetherian, so is $R[x]$.
\end{theorem}
\begin{proof}
Suppose $I$ is an ideal of $R[x]$ that is not finitely generated. Then we may choose $f_1\in I$ of minimal degree. Now choose $f_2\in I\setminus \langle f_1 \rangle$ of minimal degree. By assumption, we never have $I=\langle f_1,\dots,f_n\rangle,$ so using the axiom of choice, we may continue this process indefinitely. The sequence $f_1,f_2,\dots$ is of (not necessarily strictly) increasing degree. The sequence of increasing ideals of leading coefficients of the $f_i$'s 
$$\langle a_1\rangle\subset \langle a_1,a_2\rangle \subset \dots$$
terminates since $R$ is Noetherian, say at $a_{n-1}$. It follows that $a_n\in \langle a_1,\dots,a_{n-1}\rangle$, so $a_n=\sum_i^{n-1} r_ia_i$ for some $r_i\in R$. By multiplying by appropriate powers of $x$ we may make the degrees of all the $f_i$'s for $i<n$ equal $deg(f_n)$. Then $$(\sum r_if_i)-f_n\not \in \langle f_1,\dots,f_{n-1}\rangle$$ and has degree less than $f_n$ by construction. This contradicts $f_n$ being of minimal degree. \cite{Hilbert}
\end{proof}

\begin{corollary}
If $R$ is Noetherian, so is $R[x_1,x_2,x_3,\dots,x_n]$
\end{corollary}
\begin{proof}
This follows by induction from Hilbert's basis theorem, by noting $$R[x_1,x_2,x_3,\dots,x_n]=R[x_1,\dots,x_{n-1}][x_n].$$
\end{proof}


\begin{example}
$\mathbb{Z}[x]$ is Noetherian as $\mathbb{Z}$ is. Any field $F$ is clearly Noetherian, therefore $F[x]$ is also Noetherian.
\end{example}

\begin{remark} \label{warning-Noetherian-rings-and-mods}
If a ring $R$ is Noetherian as an $S-module$ for a subring $S$, then $R$ is also Noetherian as a ring: its ideals are submodules, which are finitely generated over $S$ by assumption, hence also finitely generated over $R$ as ideals.

The converse is \textbf{not} true. For example, by \cref{Hilbert-basis-theorem}, $\Z[x]$ is Noetherian as a ring. However, it is not Noetherian as a $\Z$-module. Take for example, the non-terminating increasing sequence of submodules
$$R1 \subset \Z  1\oplus \Z x\subset \Z  1\oplus \Z  x\oplus \Z x^2 \subset \dots $$
\end{remark}

We will also want to show that any finitely generated $R$-module is Noetherian. (Recall $M$ is finitely generated if there is a finite subset ${m_1,\dots,m_n}\subset M$ such that every $m\in M$ can be written as $m=r_1m_1+dots+r_nm_n$ for some $r_i\in R.$) By the remarks of \cref{warning-Noetherian-rings-and-mods}, this will show that any ring that is a finitely generated module over a Noetherian ring is a Noetherian ring. In particular, $\Z[\omega]$ is a Noetherian ring since $\Z$ is. Note that a submodule of a finitely generated module $M$ need not be finitely generated in general. For example, $R=\Z[x_1,x_2,\dots]$ is a finitely generated $R-module$ with generating set $\{1\}$. However, the submodule $\langle x_1,x_2,\dots\rangle$ is not finitely generated as an $R-$module. We therefore cannot use property $(iii)$ of \cref{noetherian-module-equivalences} directly. The Noetherian property of $R$ will have to play a key role. We will proceeed as in \cite{Wright} by first proving a lemma.

\begin{lemma}
Let $M$ be an $R-$module and $N$ be a submodule. Then $M$ is a Noetherian $R-$module if and only if both $N$ and $M/N$ are.
\end{lemma}
\begin{proof}
$(\implies)$ $N$ is Noetherian, since its submodules are submodules of $M$, hence finitely generated. Additionally, a submodule of $M/N$ is of the form $A/N$ for a submodule $A$ of $M$, so if $\{m_1,\dots m_n\}$ generates $A$, then $\{m_1+N,\dots m_n +N\}$ generates $A/N$.

$(\impliedby)$ Let $$M_1\subset M_2\subset \dots$$ be an increasing sequence of submodules of $M$ and note $$M_1\cap N\subset M_2\cap N \dots $$ and $$M_1/N\subset M_2/N\subset \dots$$ are increasing sequences of submodules of respectively $N$ and $M/N$. By assumption, both terminate, say at $M_n\cap N$ and $M_{n'}/N$. Let $m$ be any integer such that both sequences have terminated by the $m$th member. Now let $a\in M_{m+1}$. Since $M_m/N=M_{m+1}/N$, there is a $b\in M_m$ with $a-b\in N$. Note $a-b\in M_{m+1}$ since $M_m\subset M_{m+1}$. Since $M_m\cap N=M_{m+1}\cap N$, $a-b\in M_m$ as well, so $a\in M_m$. Since $a$ was arbitrary, $M_m=M_{m+1}$. Since $m$ was an arbitrary integer with $m>n,m>n'$, in fact $M_m=M_{m'}$ for all $m'>m$ so the sequence terminates.
\cite{Wright}
\end{proof}

\begin{corollary}\label{finitely-gen-module-Noetherian}
If $R$ is Noetherian, then any finitely generated module $M$ over $R$ is Noetherian.
\end{corollary}
\begin{proof}
We prove this by induction on the generating elements of $M$. If $M$ is generated by a single element $m_1$, then we have an ideal $I$ of $R$ given by $I=ker(f)$ for $$f:R\rightarrow M, r\mapsto rm.$$ Since $f$ is surjective, the first isomorphism theorem for $R$-modules gives $M\iso R/I$ and the latter is a Noetherian $R-$module by the lemma, since $R$ is a Noetherian $R-$module.

Now suppose a Noetherian module $M$ is generated by $\{m_1,\dots m_{n-1}\}$ and let $M'$ be generated by $\{m_1,\dots m_{n-1}, m_n\}$. Then $M'/M$ is generated by a single element $\{m_n\}$ and is therefore Noetherian as we have just shown. Since $M$ was Noetherian, $M'$ is Noetherian by the Lemma.
\cite{Wright}. 
\end{proof}

We now reach our main motivation for examining Noetherian modules. 
In $\mathbb{Z}$, we can write any element as a product of primes. In general rings, we want to think about irreducibles instead of primes, so it is natural to ask: in which rings can elements be written as a product of irreducible elements? As it turns out, all Noetherian rings have this property, as proven in \cite{Wright}. The proof given here is my own.
\begin{lemma}\label{noetherian-irreducible-elements}
Any (non-zero, non-unit) element $r\in R$ of a Noetherian ring $R$ can be written as a finite product of irreducible elements
$$r=r_1r_2r_3\dots r_n.$$
\end{lemma}
\begin{proof}
Let $A=\{\text{ proper principal ideals } I \text{ containing } r\}$. $\langle r \rangle \in A$ so $A$ is non-empty. Since $R$ is Noetherian, $A$ has a maximal element $\langle r_1\rangle$. I claim $r_1$ is irreducible: it is not a unit since $\langle r_1 \rangle$ is a proper ideal. Furthermore, if $r_1=ab$ for non-units $a,b$ then $a\not\in \langle r \rangle$ since $b$ is not a unit. Then $r\in \langle a \rangle \not \subset \langle r_1 \rangle$ which contradicts $r_1$ being maximal among proper principal ideals containing $r$. This is all to show that there exists an irreducible dividing $r$. Now write $r=r'_2r_1$ and repeat the process on $r'_2$ to find an irreducible $r_2$ dividing $r'_2$. This process either stops with $r'_n$ being a unit, giving a finite factorisation $r=u r_1r_2r_3\dots r_n$ for a unit $u$, or it gives an increasing chain
$$\langle r_1\rangle\subset \langle r_1r_2\rangle \subset \dots$$ which by the Noetherian property terminates: say $\langle r_1r_2r_3\dots r_{m-1}\rangle = \langle r_1\dots r_m\rangle$. Then $r_1r_2\dots r_{m-1}=sr_1r_2\dots r_m$ for some $s\in R$, so since $R$ is an integral domain, $1=sr_{m}$ so $r_m$ is a unit, a contradiction.
\end{proof}

We are mainly interested in Noetherian rings because of this factorisation property. However, not only Noetherian rings have this property. As an example, let $x\in \Z[x_1,x_2,\dots]$, the ring of polynomials in infinitely many variables, which we have already argued is not Noetherian. Nonetheless, since $x$ is also in the Noetherian ring $\Z[x_1,x_2,\dots,x_n]$ for some $n$ depending on $x$, $x$ factorises into a product of irreducibles in $\Z[x_1,x_2,\dots,x_n]$, which are easily seen to also be irreducible in $\R[x_1,x_2,\dots]$. $\Z[x_1,x_2,\dots]$ therefore has the factorisation property despite not being Noetherian.

%\begin{remark}
%As we will sho prime decompositions in $\mathbb{Z}$ are unique, but this is not necessarily the case in Noetherian rings! For example, $\dots$. Noetherian rings where the decomposition into irreducibles is unique (up to reordering and multiplication by units) are called \defn{unique factorisation domains} or \defn{UFDs}. The main failure of Lamé's proof is that number rings are not UFDs. For example $\dots$
%\end{remark}
%\begin{proof}
%\end{proof}

\begin{definition}
An integral domain $R$ is a unique factorisation domain (UFD) if every non-zero non-unit $r\in R$ can be written as a product of irreducible elements $r=r_1r_2\dots r_n$ \textbf{uniquely} up to reordering and multiplication by units.
\end{definition}
\begin{definition}
A non-zero non-unit $p\in R$ of an integral domain is \defn{prime} if $p|ab\implies p|a$ or $p|b$. 
\end{definition}
\begin{remark}
Note $p\in R$ is prime exactly when $\langle p \rangle $ is a prime ideal, since $p|a \iff a\in \langle p \rangle$.
\end{remark}


%In the previous section, we showed that all PIDs are UFDs. Unfortunately, we also saw that PID is too strict a requirement for all number rings. Number rings are Noetherian, however, as we have just shown. Our hope is that there is that there is an easier requirement to satisfy being a UFD, sitting between Notherian rings and PIDs. Such a requirement exists - but first we need a definition.

We argued that the correct notion of "primeness" was irreducibility. We can show that this new notion is strictly stronger:  every prime element is irreducible: Let $r$ be prime and $r=xy$. Then $r|xy$, so without loss of generality, $r|x\implies x=cr$ for some $c\in R$. Then $$r=rcy\implies r(1-cy)=0\implies cy=1 \text{ (integral domain) }\implies y \text{ is a unit.}$$ Additionally, as is easily shown, $\langle p \rangle$ is a prime ideal. As we now show, prime elements give rise to a easier UFD condition.

\begin{proposition}\label{Noetherian-prime-UFD}
A Noetherian ring $R$ is a UFD if and only if every irreducible element $r\in R$ is prime.
\end{proposition}
\begin{proof}
$(\implies)$ Let $r\in R$ be irreducible and let $r|xy$. By uniqueness of the factorisation into irreducibles, $r$ appears in the unique factorisation of either $x$ or $y$, so $r|x$ or $r|y$.

$(\impliedby)$ By \cref{noetherian-irreducible-elements}, any element has a decomposition into irreducibles. We only need to take care of uniqueness. Suppose $r=r_1r_2r_3\dots r_n=s_1s_2\dots s_m$ are two decompositions into irreducibles. Since $r_1$ is prime, $r_1|s_i$ for some $i$. Since $s_i$ is irreducible, $s_i=ur_1$ for some unit $u$. Since $R$ is an integral domain, $$r_2r_3\dots r_n=us_2s_3\dots s_m,$$ where we have relabeled the $s_i$'s. We can define $r_2'=r_2u^{-1}$, which is also irreducible, then continue as before. If $m=n$ we have shown that every $r_i=u_is_j$ for some $j$ and unit $u_i$, as required. Otherwise, WLOG, suppose $m>n$. Then at the end, after relabeling the $s_i$'s, we have
$$\bar u= s_1\dots s_k.$$ Then $s_i|\bar{u}\implies \bar{u}=zs_i$ for some $z$. Then $(\bar{u}^{-1}z)s_i=1,$ so $s_i$ is a unit, a contradiction. \cite{Wright}
\end{proof}

It is an easy corollary that every PID is a UFD, as we now remark.

\begin{corollary}\label{PIDs-are-UFDs}
Any principal ideal domain $R$ is a UFD.
\end{corollary}
\begin{proof}
PIDs are integral domains by definition and we have already remarked that they are Noetherian. For an irreducible $r\in R$, $\langle r \rangle$ is maximal by \cref{PID-over-irreducible-is-field}, in particular it is prime by \cref{max-is-prime}. It follows that $r$ is prime.
\end{proof}

\begin{example}\label{Z_p[x]-is-UFD}
By \cref{F[x]-PID}, $\Z_p[x]$ is a PID where $p$ is prime and $\Z_p$ is the finite field of $p$ elements. It follows that $\Z_p[x]$ is a UFD.
\end{example}

\chapter{Number fields and number rings}\label{sec-number-fields}
We start this chapter with a brief summary of Galois theory applied to finite dimensional field extensions of $\Q$. Much of this is standard material covered in any undergraduate level Galois course, including one I was enrolled in while writing this dissertation. I have included proofs for results that were new to me at the time of writing and were independently researched from \cite{NumberFields}.
\section{Number fields}
A \defn{number field} is a subfield of $\C$ that is a finite dimensional vector space over $\Q$. Examples include $\Q$ itself, $\Q[\omega]$ where $\omega=e^{2\pi i/p},p$ prime - the \defn{$p$-th cyclotomic field}, and $\Q[\sqrt{n}]=\{a+b\sqrt{n}:a,b\in \Q\}$. Non-examples include $\R$ and $\C$: both are vector spaces over $\Q$, but by virtue of being uncountable, could never have a finite basis over a countable field. The examples $\Q[\sqrt{n}]$ and $\Q[e^{2\pi i/p}]$ have the following in common: both $\sqrt{n}$ and $e^{2\pi i/n}$ satisfy a monic (leading coefficient $1$) polynomial over $\Q$. If every element of a ring $R$ satisfies a monic polynomial with coefficients in a subring $S$ we say $R$ is \defn{integral over} $S$. If a field $F$ is integral over a subfield $K$ we say $F$ is \defn{algebraic over} $K$. Thus we claim that $\Q[\sqrt{n}]$ and $\Q[e^{2\pi i/p}]$ are algebraic over $\Q$.

\begin{example}\label{ex-sqrt-n-algebraic}
$\sqrt{n}$ is algebraic over $\Q$, as it satisfies $f(x)=x^2-n$. If $n$ is not a square of another integer, then the two roots of $f$ are irrational. It follows that $f$ is irreducible, as it cannot written as a product of linear polynomials in $\Q$. If $n=m^2$ then $x^2-n$ is not irreducible, but $f(x)=x-m$ is.
\end{example}
\begin{example}\label{ex-omega-algebraic}
$\omega=e^{2\pi i/p}$ is algebraic over $\Q$. It satisfies the monic $f(x)=x^p-1$. Note $f$ is not irreducible as it is divisible by $(x-1)$. We have already shown (\cref{eq-omega-polynomial}) that $\frac{x^p-1}{x-1}=(x-\omega)\dots(x-\omega^{p-1})=:g(x)$ when $x\neq 1$.
We may additionally note
$$(1+x+x^2+\dots+x^{p-1})(x-1)=(x+x^2+\dots+x^p)-(1+x+x^2+\dots+x^{p-1})=x^p-1.$$
Therefore, 
when $x\neq 1$,
$$g(x)=(1+x+x^2+\dots+x^{p-1}),$$
and since both functions are analytic on all of $\C$ they also equal on $x=1$. $g$ is therefore a monic rational polynomial satisfied by $\omega$. In fact, $g$ is irreducible, as shown in \cite{Galois}.
\end{example}

\begin{remark}
It was important in \cref{ex-omega-algebraic} that $p$ is prime. $e^{2\pi i/n}$ for $n\in \N$ is certainly algebraic over $\Q$ since it satisfies $x^n-1$, but $1+x+x^2+\dots+x^{n-1}$ is \textbf{not} irreducible over $\Q$ in general. For example, $e^{2\pi i/6}$ satisfies $f(x)=x^2-x+1=0$: simply check $e^{2\pi i/3}-e^{\pi i/3}=2\cos(\pi/3)=1$. As shown in \cite{Galois}, irreducible polynomials are always of minimal degree among polynomials annihilating $a$, so the existence of $f$ shows $(1+x+\dots+x^{5})$ is not irreducible.
\end{remark}

\cref{ex-sqrt-n-algebraic} and \cref{ex-omega-algebraic} have two things in common: in both cases the annihilating monic polynomial $f\in \Q[x]$ can be taken to be irreducible, and in both cases $\Q[a]$ is a number field. It is a standard result of Galois theory that these properties hold in generality: if $a$ is algebraic over $\Q$, its monic annihilating polynomial can be taken to be irreducible, and $\Q[a]$ is a number field. We will also show that any $\alpha\in Q[a]$ is also algebraic over $\Q$. Finally, we will show that any number field is of this form.

Recall the following definition.
\begin{definition}
If $L$ and $K$ are fields with an injective field homomorphism $K\rightarrow L$  then $L$ is said to be a \defn{field extension} of $K$, and we write $L:K$. The degree of $L$ as a vector space over $K$ is denoted $[L:K]$. 
\end{definition}
The following two lemmas are core results of Galois theory, and we refer to \cite{Galois} for their proofs.

\begin{lemma}\label{lemma-monics-are-Z}
If $f\in \Z[x]$ is monic and $f=gh$ where $g,h\in \Q[x]$ are monics, then in fact $g,h\in \Z[x]$.
\end{lemma}
\begin{lemma}\label{algebraic-satisfies-irreducible}
Let $M$ be a field extension of $K$ and let $\alpha\in M$ be algebraic over $K$. Then there exists an irreducible monic $f\in K[x]$ s.t. $f(\alpha)=0.$

%In particular, any algebraic $\alpha$ of a 

%Let $K$ be a number field and suppose $a\in \C$ is algebraic over $K$. Then $a$ satisfies a monic irreducible polynomial over $K$.
\end{lemma}
%Let $I\subset K[x]$ be the set of all polynomials annihilating $a$. By assumption, $I$ is non-empty. $I$ is an ideal: if $f,g\in K[x]$ annihilate $a$ then so does $f-g$. Additionally, if $f$ annihilates $a$ so does $fh$ for any $h\in K[x]$. $K[x]$ is a PID by \cref{F[x]-PID}, so $I=\langle f\rangle$ for some $f\in K[x]$. By dividing by the leading coefficient, we may take $f$ to be monic. Note $f$ is necessarily of minimal degree in $I$. Now let $f=gh$ and suppose WLOG that $g(a)=0$. Since $f$ is of minimal degree, $deg(f)=deg(g)$ so $h$ is a unit. Therefore $f$ is irreducible. 


\begin{lemma}\label{algebraic-gives-field} 
Let $K[\alpha]:K$ be a field extension where $\alpha$ is algebraic over $K$ and satisfies monic irreducible $f$. Then $K[x]/\langle f \rangle \iso K[a].$

%(NOTE: can I drop "number" in this? I need that later)


%Let $a\in \mathbb{C}$ be algebraic over a number field $K$ with monic irreducible polynomial $f$. Then $K[x]/\langle f\rangle$ is a number field and $K[x]/\langle f \rangle \iso K[a].$
\end{lemma}
%\begin{proof}
%By \cref{PID-over-irreducible-is-field}, $K/\langle f\rangle$ is a field. It is finite dimensional over $K$ with basis $\{1,x,\dots,x^{n-1}\}$ where $n=deg(f)$, since any polynomial $g(x)$ with $deg(g)\geq n$ can reduced by the division algorithm to $g(x)=h(x)f(x)+r(x)$, $deg(r)<n$, such that $g(x)\sim r(x)$. It is therefore a number field.

%By definition, $K[a]$ is the image of the evaluation map $$ev_a:K[x]\rightarrow \C, f\mapsto f(a).$$ By the proof of \cref{algebraic-satisfies-irreducible}, $\langle f \rangle =ker(ev_a)$, so by the first isomorphism theorem for rings, there exists a ring isomorphism $\phi:K/\langle f \rangle \iso K[a]$. Since $K[x]/\langle f\rangle$ is a field, $K[a]$ is also a field: $\phi(g)$ has multiplicative inverse $\phi(g^{-1})$, so $\phi$ is actually a field isomorphism.
%\end{proof}

\begin{remark}
\cref{algebraic-satisfies-irreducible} and \cref{algebraic-gives-field} together prove that the $n-$th cyclotomic fields $\Q[\omega],\omega=e^{2\pi i/n}$ are number fields without explicitly constructing their monic irreducible polynomial over $\Q$. 
\end{remark}

We will now restrict ourselves to finite-dimensional  field extensions between number fields $K$ and $L$. Note by assumption, $[L:\Q]$ and $[K:\Q]$ are finite - it follows that $[L:K]$ is finite.

The following lemma will show that if $L$ is a field extension of $K$ then any $a\in L$ is algebraic over $K$. The statement and proof is a generalisation of a theorem from \cite{NumberFields}.

\begin{lemma}\label{a-algebraic-over-finitely-generated-ring}
Let $R$ be a ring and $N$ be a finitely-generated $R-$module. Then any $a\in N$ is integral over $R$.
\end{lemma}
\begin{proof}
Let $M$ have generating set $\{\alpha_1,\alpha_2,\dots,\alpha_n\}$ and write $a\alpha_j=\sum a_{ij}\alpha_i$ for each $j$ to produce a matrix equation
$$\begin{pmatrix}
    a\alpha_1\\
    a\alpha_2\\
    \dots\\
    a\alpha_n
\end{pmatrix}=\begin{pmatrix} \sum a_{i1}\alpha_i\\\sum a_{i2}\alpha_i\\\dots\\\sum a_{in}\alpha_i
\end{pmatrix}=: M\begin{pmatrix}\alpha_1\\\alpha_2\\\dots\\\alpha_n\end{pmatrix}$$
In other words,
\begin{equation}\label{eq-determinant-algebraic-over-R}(a I-M)\begin{pmatrix}\alpha_1\\\alpha_2\\\dots\\\alpha_n\end{pmatrix}=0.\end{equation}
Therefore $\det(aI-M)=0$. Expanding out the determinant gives a monic degree $n$ polynomial over $R$ satisfied by $a$, as required. \cite{NumberFields}
\end{proof}


\begin{example}\label{recipe-for-finding-polys-over-q}
The proof of \cref{a-algebraic-over-finitely-generated-ring} gives a recipe for finding monic polynomials over $\Q$ satisfied by any element of a number field. For example,$\Q[\sqrt{2}]$ has basis $\{1,\sqrt{2}\}$ over $\Q$. For $a=1+\frac{1}{2}\sqrt{2}\in \Q[\sqrt{2}]$, \cref{eq-determinant-algebraic-over-R} reads
$$\begin{pmatrix}
(1+\frac{1}{2}\sqrt{2}) - 1 & -\frac{1}{2}\\
-1 & (1+\frac{1}{2}\sqrt{2}) - 1
\end{pmatrix}\begin{pmatrix}1\\\sqrt{2}\end{pmatrix}=0$$
Setting the determinant to zero gives $(a-1)^2-\frac{1}{2}=0$ such that $1+\frac{1}{2}\sqrt{2}$ is algebraic over $\Q$ with monic annihilating polynomial $f(x)=x^2-2x+\frac{1}{2}$. In fact $f$ is the unique monic irreducible, as both its roots are irrational. It is important to note, however, that this process may yield a reducible polynomial.
\end{example}

We finish off the section by classifying all number fields. It turns out to be helpful to work not with the vector space structure of a number field $L$ over $\Q$, but rather with embeddings (injective field homomorphisms) $L\rightarrow \C$ that restrict to the identity on $\Q$. To explore this link, we need to borrow some results from Galois theory. We will roughly follow Appendix 2 of \cite{NumberFields}. 


%\section{Normal field extensions}
%The field $K=\mathbb{Q}[\sqrt[3]{2}]$ is entirely contained in the real line, and therefore does not contain all the conjugates of its elements (equivalently, the conjugates of $\sqrt[3]{2}$, as we will show). In particular, the irreducible $f(x)=x^3-2$ has three roots $\sqrt[3]{2},\sqrt[3]{2}\omega,\sqrt[3]{2}\omega^2$ where $\omega=e^{2\pi i/3}$ and of these only $\sqrt[3]{2}$ is real. There is, however, a field extension $L=\mathbb{Q}[\sqrt[3]{2},\sqrt[3]{2}\omega,\sqrt[3]{2}\omega^2]\subset \C$ of $K$ which contains all the conjugates of $\sqrt[3]{2}$. In fact, it contains the conjugates of all of $K$. Furthermore, this turns out to be possible for any number ring. The goal of this section is to prove this result. In doing so we will also classify all finite field extensions of number fields. Our treatment roughly follows Appendix 2 of \cite{NumberFields}.

%The following result shows that instead of working with basis elements in $K$, we can work with embeddings (injective field homomorphisms) $L\rightarrow \C$ that restrict to the canonical inclusion on $K$.

\begin{lemma}\label{monic-irreducible-has-all-roots}
Let $\alpha\in \C$ be algebraic over a number field $K$ with monic irreducible $f$ of degree $n$. Then $f$ has $n$ distinct roots in $\C$.
\end{lemma}
\begin{proof}
Since $\C$ is algebraically closed, $f$ has $n$ roots counted with multiplicity. We in fact claim these roots are pairwise distinct. Suppose $l(x)$ is linear with $f(x)=l(x)^2g(x)$. Taking derivates, $f'(x)=2l(x)g(x)+l(x)^2g(x)$, so $l|f'$. However, since $I=\langle f \rangle$ is maximal in $K[x]$ and $f'(x)\not \in I$, $\langle f, f'\rangle=K[x]$. It follows that $$1=f(x)g(x)+f'(x)h(x)$$ for some $g,h\in K[x]$. This is a contradiction since $l$ divides the RHS but not the LHS. \cite{NumberFields}
\end{proof}

\begin{definition}
Let $a\in \C$ be algebraic over $K$ with irreducible monic $f$ over a number field $K$. If $b\in \C$ is such that $f(b)=0$, then $a$ and $b$ are said to be \defn{conjugate over $K$}.
\end{definition}

Let us firstly note that conjugacy over $K$ is an equivalence relation. Additionally, if $\sigma:K\rightarrow \C$ is an embedding of a number field, then $\sigma(a)$ must be conjugate to $a$ over $\Q$ by the properties of field homomorphisms: if $f$ is the monic irreducible polynomial over $\Q$ satisfied by $a,$ then
$$\sigma(f(a))=\sigma(\sum_{i=0}^n a_i a^i)=\sum_{i=0}^n a_i \sigma(a)^i.$$

\begin{proposition}\label{extend-to-[L:K]-embeddings}
Every embedding $K\rightarrow \C$ extends to exactly $[L:K]$ embeddings of $L$ in $\C$.
\end{proposition}
\begin{proof}
Let $\{a_1,a_2,\dots,a_n\}$ be a basis for $L$ over $K$. Note $$L=span(a_1,\dots,a_n)\subset K[a_1,a_2,\dots,a_n]\subset L.$$ Therefore $L= K[a_1,a_2,\dots,a_n].$ Note furthermore that $$K[a_1,a_2,\dots,a_n]=K[a_1,\dots,a_{n-1}][a_n],$$ so we may work inductively on the basis elements. The result is clear if $L=K$, so set $J=K[a_1,a_2,\dots,a_{k}]$ and suppose every embedding of $K$ extends to $[J:K]$ embeddings of $J$. Let $L=J[a]$, $\sigma:J\rightarrow \C$ be an embedding
and $f$ be the irreducible monic of $a$ over $J$. By \cref{monic-irreducible-has-all-roots}, $f$ has $n=deg(f)$ distinct roots in $\C$, which are the conjugates of $a.$ $\sigma(a)$ is necessarily one of these conjugates, and each choice of conjugate $b$ defines an embedding of $L$ restricting to $\sigma$ on $J$ in the following way:
$$\sigma f(a)=\sigma \big(\sum_{i=0}^n c_ia^i\big)=\sum_{i=0}^n \sigma(c_i) b^i.$$
Since there was no choice involved other than the choice $a\mapsto b$, every embedding of $L$ restricting to $\sigma$ on $J$ is of this form. There are therefore $n=deg(f)$ such embeddings. Since $L\iso J[x]/\langle f\rangle$, $[L:J]=deg(f)$. 

We have shown that every embedding of $K$ extends to $[L:J][J:K]$ embeddings of $J[a]$. To finish the inductive proof, we need to show this equals $[L:K]$, let $\{b_1,b_2,\dots,b_n\}$ be a basis of $J$ over $K$ and $\{c_1,c_2,\dots,c_m\}$ be a basis of $L$ over $J$. This gives a basis $\{b_ic_j:1\leq i \leq n, 1 \leq j\leq m\}$ of size $nm$. It is easy to show this spans $L$, and 

$$\sum_{i\leq n,j\leq m}a_{ij}b_ic_j=0\implies \sum_{j\leq m} \big(c_j \sum_{i\leq n}a_{ij}b_i\big)=0$$
$$\implies \forall j, \sum_{i\leq n} a_{ij}b_i=0 \text{ (linear independence of the } c_j \text{'s)}$$
$$\implies \forall (i,j), a_{ij}=0 \text{ (linear independence of the } b_i \text{'s)},$$
giving linear independence.  \cite{NumberFields}
\end{proof}

Our desired corollary follows directly by considering the canonical inclusion $K\rightarrow \C$ as an embedding.
\begin{corollary}
There are $[L:K]$ embeddings of $L$ in $\C$ restricting to the canonical inclusion on $K$.
\end{corollary}

We now justify studying the field extensions of the form $K[a]$ by proving every finite degree number field extension is of this form.

\begin{theorem}\label{classifying-finite-field-extensions}
For every number field extension $K\subset L$, $L=K[\alpha]$ for some $\alpha \in L$.
\end{theorem}
\begin{proof}
We proceed by induction on $n=[L:K]$. The proof is clear in the case $L=K=K[1]$. Now suppose $K[a_1,\dots,a_{n}]=K[\alpha]$ for some $\alpha \in L$ and let $L=K[\alpha,\beta]$ for $\beta\not\in K[\alpha]$. Let $0\neq a\in K$ and consider $\alpha+a\beta$. Suppose for sake of contradiction that $K[\alpha+a\beta]\neq L$. Then $[K[\alpha+a\beta]:K]<n$ where we recall $[K[\alpha+a\beta]:K]$ is the number of conjugates of $\alpha+a\beta$ over $K$. There are $n$ embeddings of $L$ restricting to the identity on $K$, and since each one takes $\alpha+a\beta$ to one of its conjugates, of which there are fewer than $n$, $\sigma(\alpha+a\beta)=\sigma'(\alpha+a\beta)$ for two such embeddings $\sigma,\sigma'$. Since these are field homomorphisms fixing $K$, 
$$\sigma(\alpha)+a\sigma(\beta)=\sigma'(\alpha)+a\sigma'(\beta)$$
so
$$a=\frac{\sigma(\alpha)-\sigma'(\alpha)}{\sigma(\beta)-\sigma'(\beta)}$$
Note the denominator is always nonzero since if $\sigma(\beta)=\sigma'(\beta)$ then $\sigma(\alpha)=\sigma'(\alpha)$ as well, which would imply $\sigma$ and $\sigma'$ equal on all of $L$ which we assumed they didn't.

Since there are only finitely many choices for $\sigma(\alpha),\sigma(\alpha'),\sigma(\beta)$ and $\sigma(\beta')$,
$$K[\alpha+a\beta]\neq L$$
for only finitely many $a\in K$. Choosing any other $a$ will give the desired result. \cite{NumberFields}
\end{proof}

\begin{example}
Consider $L=\Q[i,\sqrt{3}]$ as a field extension of $\Q$. The conjugates of $i$ over $\Q$ are $\pm i$ and the conjugates of $\sqrt{3}$ over $\Q$ are $\pm \sqrt{3}$. Since $\frac{\pm 2i}{\pm 2\sqrt{3}}\not \in \Q$, the proof of \cref{classifying-finite-field-extensions} gives that $L\iso \Q[i+a\sqrt{3}]$ for any non-zero $a\in \Q$.
$$$$
\end{example}

\begin{example}
Let us show that $$\Q[\sqrt{2},\sqrt{3},\sqrt{4},\dots,\sqrt{n}]=\Q[\sqrt{2}+\sqrt{3}+\sqrt{5}+\dots\sqrt{p}]$$ where $p$ is the largest prime $p\leq n$. It is clear that $$\Q[\sqrt{2},\sqrt{3},\sqrt{4},\dots,\sqrt{n}]=\Q[\sqrt{2},\sqrt{3},\sqrt{5},\dots,\sqrt{p}]$$ since $\sqrt{nm}\in \Q[\sqrt{n},\sqrt{m}]$. We may work inductively. $\Q[\sqrt{2},\sqrt{3}]=\Q[\sqrt{2}+\sqrt{3}]$ by a similar argument to the previous example. Let $s_q=\sqrt{2}+\sqrt{3}+\dots+\sqrt{q}$ and consider $\Q[s_q,\gamma]$ where $\gamma$ is the next prime after $q$. Let $\sigma,\sigma'$ be two embeddings of $\Q[s_q,\gamma]$ fixing $\Q$. Since these are field homomorphisms sending conjugates to conjugates, $\sigma(s_q)-\sigma'(s_q)=\pm 2 \sqrt{2}\delta_2+\pm 2\sqrt{3}\delta_3+\dots+\pm 2\sqrt{q}\delta_q$, where $\delta_i=0$ or $1$ and not all $\delta_i$ are $0$. Since $$1\neq \frac{\pm 2\sqrt{2}\delta_2+\pm2 \sqrt{3}\delta_3+\dots+\pm 2\sqrt{q}\delta_q}{\pm 2\sqrt{\gamma}},$$
the proof of \cref{classifying-finite-field-extensions} gives $\Q[s_q,\gamma]\iso \Q[s_q+\gamma]$.
\end{example}

\section{Normal field extensions}
We give a brief remark that any field extension can always be extended to a normal extension, defined below. We again follow \cite{NumberFields}
\begin{definition}
A field extension $L$ of $K$ is said to be \defn{normal} if each of the $[L:K]$ embeddings of $L$ restricting to the inclusion on $K$ are automorphisms.
\end{definition}

Note this equivalent to $L$ containing all its conjugates over $K$: if $a\in L$ and $a\sim b$ then there is an embedding of $L$ fixing $K$ such that $a\mapsto b$ -- take any extension, guaranteed by \cref{extend-to-[L:K]-embeddings}, of the embedding $K[a]\rightarrow \C : \sum a_i a^i\mapsto \sum a_i b^i$. Since this embedding is an automorphism, $b\in L$. Since $a,b$ were arbitrary, $L$ contains all its conjugates over $K$. 

Conversely, by \cref{classifying-finite-field-extensions} we can set $L=K[\alpha]$. Every embedding fixing $K$ is entirely determined by a choice of conjugate $\alpha\mapsto \beta$. If $\beta \in L$, then $$\sigma(\sum a_i \alpha^i)=\sum a_i \beta^i\in L$$
so this embedding is an automorphism.


%\begin{remark} (BAD PLACE TO PUT THIS REMARK)
%In the next theorem, we will consider nested field extensions $K\subset L \subset M$. The inclusion symbol is used somewhat erroniously, in actuality we want to have injective field homomorphisms $K\rightarrow L \rightarrow M$. If $m,n\in M$ are conjugate over $L$, they are necessarily conjugate over $K$: let $f$ be the monic irreducible over $L$ and let $m$ and $n$ satisfy monic irreducibles $g$ and $h$ respectively over $K$. Since $g,h\in \langle f\rangle_L$, $$g=h+jf.$$
%Since $g(m)=jf(m)=0, h(m)=0$. However, $g$ was the unique monic irreducible with this property so $g=h$.

%The converse, however, is not true. Take for example $\Q\subset \C\subset \C$. $i,-i$ are conjugate over $\Q$ with minimal polynomial $f(x)=x^2+1$, however they are not conjugate over $\C$.
%\end{remark}

\begin{theorem}
Every finite field extension $K\subset L$ can be extended to a finite normal field extension $K\subset L \subset M$ (normal over both $K$ and $L$).
\end{theorem}
\begin{proof}
By \cref{classifying-finite-field-extensions} we may write $L=K[\alpha_1]$ for some $\alpha_1 \in L$. Let $\alpha_1,\alpha_2,\dots,\alpha_n$ be the $n=[L:K]$ conjugates of $\alpha_1$ over $K$. Let $M=K[\alpha_1,\alpha_2,\dots,\alpha_n]$ be a field extension of $L$ (and by extension $K$) via the inclusion $L\rightarrow M$. We need to show $M$ contains all its conjugates over $K$, and we will then get the second part for free. Indeed, any embedding of $M$ restricting to the inclusion on $L$ also restricts to the inclusion on $K$ and would therefore be an automorphism.

By construction, $M$ contains all the conjugates of $\alpha_1$. Let $\sigma$ be an embedding of $M$ fixing $K$, and let $x=\sum_i \sum_j a_{ij}\alpha_j^i$ be an element of $M$. $\sigma$ maps $x$ to one of its conjugates, and as $\sigma$ is a field homomorphism, $\sigma(x)=\sum_i \sum_j a_{ij}\sigma(\alpha_j)^i\in M$. Therefore $M$ contains all its conjugates over $K$. \cite{NumberFields}
\end{proof}

\section{Number rings}
Inside a number field sit the \defn{algebraic integers}: the elements integral over $\Z$. Letting $\A$ be the algebraic integers in $\C$, we define a \defn{number ring} as the ring $\A\cap K$ of algebraic integers in a number field $K$. For Fermat's Last Theorem, we are particularly interested in the number ring $\A\cap \Q[\omega],$ where $\Q[\omega]$ is the $p$-th cyclotomic field for $p$ prime. Of course, we would like to prove that number rings are indeed rings. As shown in \cite{NumberFields}, this follows from the fact that $a\in \C$ is an algebraic integer iff $\Z[a]$ is a finitely generated $\Z-$module. In fact we prove this more generally for an element $a$ of a field integral over a subring $R.$

%(TODO: Prove in more generality)
\begin{proposition}\label{algebraic-int-iff-finitely-generated}
The following are equivalent for an element $a$ of a field $L$ with subring $R.$
\begin{enumerate}[(1)]
    \item $a$ is integral over $R$.
    \item $R[a]$ is a finitely generated $R$-module
    \item There is a nonzero finitely generated $R-$module $M$ with $aM\subset M.$
\end{enumerate}
%$a\in L$ is integral over a subring $R$ if and only if $R[a]$ is a finitely generated $R$-module.
\end{proposition}
\begin{proof}
$(1 \implies 2)$ $R[a]$ is clearly a $R-$module for any $a$. Let $f\in R[x]$ be a monic annihilating polynomial of $a$ of degree $n$. Then I claim $\{1,a,a^2,\dots,a^{n-1}\}$ is a generating set for $R[a]$. For an arbitrary element $\alpha=\sum_i^m a_ia^i\in R[a]$ where $m\geq n$ we can let $g(x)=a_mx^{m-n}f(x)$ s.t. $g(a)=0$. It follows that $$\alpha=\sum_i^{m}a_ia^i=\sum_i^{m}a_ia^i-g(a),$$ and by construction the RHS can be rewritten in the form $\sum_i^{m-1}a_i'a^i$. We may repeat this process $m-(n-1)$ times until $\alpha$ is written in the form $\alpha=\sum_i^{n-1} a_i^*a^i$.

$(2\implies 3)$ trivially, taking $M=R[a]$.

$(3\implies 2)$ follows directly from \cref{a-algebraic-over-finitely-generated-ring}.
\end{proof}


\begin{corollary}\label{number-rings-are-rings}
Number rings are rings.
\end{corollary}
\begin{proof}
We show the algebraic integers $\A\in \C$ form a ring. By the subring test, it is enough to show that number rings are closed under multiplication and subtraction, and contain $1$. $1$ is clearly an algebraic integer. If $\Z[a]$ and $\Z[b]$ are finitely generated, then so is $Z[a,b]$ with generating set $\{a_ib_j\}$ where $\{a_i\}$ and $\{b_j\}$ are the finite generating sets of $\Z[a]$ and $\Z[b]$ respectively. Since $a+b,ab\in \Z[a,b]$ they are both algebraic by \cref{a-algebraic-over-finitely-generated-ring}.

Finally, we note that the intersection of two rings is always a ring, therefore $\A\cap R$ is a ring for any ring $R\subset \C$.
\end{proof}
\begin{example}
As in \cref{recipe-for-finding-polys-over-q}, we can use \cref{a-algebraic-over-finitely-generated-ring} as a recipe for finding monic polynomials over $\Z$ satisfied by an algebraic integer $a$, given a generating set for $\Z[a]$. For example, take $a=i+\sqrt{2}$. $\Z[i]$ has generating set $\{1,i\}$ and $\Z[\sqrt{2}]$ has generating set $\{1,\sqrt{2}\}$. By the proof of $\cref{number-rings-are-rings}$, $i+\sqrt{2}\in\Z[i,\sqrt{2}]$ which has generating set $\{1,i,\sqrt{2},\sqrt{2}i\}$. Then \cref{eq-determinant-algebraic-over-R} reads
$$\begin{pmatrix}
(i+\sqrt{2}) & -1 & -1 & 0\\ 
1 & i+\sqrt{2} & 0 & -1\\
-2 & 0 & i+\sqrt{2} & -1\\
0 & -2 & 1 & i+\sqrt{2}
\end{pmatrix}\begin{pmatrix}1\\i\\\sqrt{2}\\i\sqrt{2}\end{pmatrix}=0$$

The determinant of this matrix expands to
$$a^4-2a^2+9$$
so $i+\sqrt{2}$ is an algebraic integer satisfying monic $f(x)=x^4-2x^2+9$ over $\Z$.
\end{example}

\begin{remark}\label{algebraic-integer-irreducible-poly-has-integer-coefficients}
Note \cref{lemma-monics-are-Z} shows that if $a\in K$ is an algebraic integer, then its monic irreducible polynomial over $\Q$ has integer coefficients. If $a$ satisfies $f\in \Z[x]$ and has monic irreducible $g\in \Q[x]$ then $f=gh$ for some $h\in \Q[x]$ since $f\in \langle g \rangle$, and so $g\in \Z[x]$ by the lemma.
\end{remark}

\begin{lemma}\label{cref-is-transitive}
If $A\subset B\subset C$ are rings such that $C$ is integral over $B$ and $B$ is integral over $A,$ then $C$ is integral over $A$.
\end{lemma}
\begin{proof}
Let $c\in C$ and write $c=b_0+b_1c+\dots +b_{n-1}c^{n-1}+c^n=0$. Let $B'=A[b_0,b_1,\dots,b_{n-1}]\subset B$ so $B'$ is integral over $A$ hence finitely generated as an $A$-module by \cref{algebraic-int-iff-finitely-generated}. Additionally, $B'[c]$ is a finitely generated $B'-$module, also by \cref{algebraic-int-iff-finitely-generated}. It follows that $B'[c]$ is a finitely generated $A$-module with $cB'[c]\subset B'[c]$ so $c$ is integral over $A$ by \cref{algebraic-int-iff-finitely-generated}.
\end{proof}

\cref{a-algebraic-over-finitely-generated-ring} and \cref{cref-is-transitive} show that a complex number $\alpha$ is algebraic over any number field $K$ if and only if it is algebraic over $\Q$. One direction is clear, and the lemma gives the other direction. Thus we may simply call such elements algebraic.


\section{The p-th cyclotomic fields}
We are now ready to prove a remarkable and impactful result which is core to Kummer's proof of Fermat's Last Thereom.
\begin{theorem}\label{algebriac-integers-are-Z[w]}
For the $p-$th cyclotomic fields $\Q[\omega]$,
$$\A\cap\Q[\omega]= \Z[\omega].$$
\end{theorem}
In fact this statement is true for all $n$-th cyclotomic fields \cite{NumberFields}, but the proof of this is more involved and not needed for our purposes. Let us first note that every element in $\Z[\omega]$ is an algebraic integer. This follows directly from \cref{algebraic-int-iff-finitely-generated} since $\Z[\omega]$ is a finitely generated $\Z$-module with generating set $\{1,\omega,\omega^2,\dots,\omega^{p-1}\}$.We proceed as in \cite{NumberFields}, using the $[K:\Q]$ embeddings of a number field $K$ in $\mathbb{C}$ fixing $\Q$ to define the \defn{norm} and \defn{discriminant} of $K$. 

\begin{definition}
Let $n=[K:\Q]$ and let $\sigma_1,\sigma_2,\dots,\sigma_n$ be the $n$ embeddings of $K$ in $\C$ fixing $\Q$. Let $\alpha\in K$. %The \defn{trace} of $K$  is the function $$\alpha\mapsto\sigma_1(\alpha)+\sigma_2(\alpha)+\dots+\sigma_n(\alpha).$$
The \defn{norm} $N$ of $K$ is the function
$$\alpha\mapsto\sigma_1(\alpha)\sigma_2(\alpha)\dots \sigma_n(\alpha).$$
We will sometimes write $N^K$ to emphasize which number field we are considering the embeddings of.
\end{definition}

The first goal is to show that the codomain of $N$ s in fact $\Q$. Furthermore, we wish to show that  $N(\alpha)\in \Z$ if $\alpha$ is an algebraic integer. Firstly note since the $\sigma_i$'s are field homomorphisms that fix $\Q$, $$N(r\alpha\beta)=r^nN(\alpha)N(\beta).$$

\begin{lemma}\label{norm-of-alg-int-is-integer}
For any $\alpha\in K$, $N(\alpha)\in \Q.$ If $\alpha$ is an algebraic integer, then $N(\alpha)\in \Z$.
\end{lemma}
\begin{proof}
Let $\alpha$ be integral over $R$ where $R=\Q$ or $\Z$. By \cref{algebraic-integer-irreducible-poly-has-integer-coefficients}, the monic irreducible polynomial of $\alpha$ over $\Q$ has coefficients in $R$. First suppose $K=\Q[\alpha]$. Then each embedding $\sigma_i$ of $K$ fixing $\Q$ is uniquely defined by a choice of conjugate of $\alpha$ aka root of $f$. Therefore $N(\alpha)$ is the product of the roots of $f$, which is $\pm\frac{a_0}{a_n}=\pm a_0\in R$ where $a_n=1$ since $f$ is monic.

For general $K,$ each embedding fixing $\Q$ restricts to one of these on $\Q[\alpha]$. Furthermore, each $\sigma_i$ extends to $n:=[K:\Q[\alpha]]$ embeddings of $K$. Therefore 
$$N(\alpha)=\sigma_1(\alpha)^n\sigma_2(\alpha)^n\dots\sigma_k(\alpha)^n=(\sigma_1(\alpha)\dots\sigma_k(\alpha))^n\in R.$$
\end{proof}

%\begin{itemize}
    %\item %$T(r\alpha+\beta)=rT(\alpha)+T(\beta)$ for $\alpha,\beta\in K, r\in \Q$.
    %\item $N(r\alpha\beta)=r^nN(\alpha)N(\beta)$.
%\end{itemize}
%In particular, $T$ is linear in $\Q$. 

We will return to the norm later. Our next goal is to show that number rings are free abelian as additive groups. Our goal will be to sandwich number rings between two free abelian groups of the same rank. Recall an abelian group $G$ is said to be free abelian of rank $n$ if $G\iso \Z^n$. Recall the following fundamental theorem of group theory which we state without proof - see \cite{GroupTheory}.

\begin{theorem}[Fundamental theorem of finitely generated abelian groups.]
If $A$ is a finitely generated abelian group then
$$A\iso \Z^n\times\Z_{k_1}\times \Z_{k_2}\times \dots \times \Z_{k_m}$$
for some $k_1|k_2|\dots|k_m$.
\end{theorem}

\begin{corollary}
Every subgroup $G$ of a finitely generated free abelian group $A$ is free.
\end{corollary}
\begin{proof}
This follows from the fundamental theorem by noting $G$ cannot have any elements of order $n$, since $A$ does not.
\end{proof}

It should be clear that $G$ cannot have a greater rank than $A$. As argued in \cite{NumberFields}, it follows that if the number ring contains and is contained by free abelian groups of rank $n$, then it must itself be free abelian of rank $n.$

Next, we would like to show that any number field has a basis consisting entirely of algebraic integers. Such a basis is called an \defn{integral basis}. We fill in the details of the following proof from \cite{NumberFields}.

\begin{proposition}
Any number field $K$ has an integral basis.
\end{proposition}
\begin{proof}
Take any basis $\{\alpha_1,\alpha_2,\dots,\alpha_n\}$. Let $f\in \Q[x]$ be the monic irreducible satisfied by $\alpha_1$ and of degree $n$. Let
$$f=\sum_{i=0}^{n}\frac{a_i}{b_i}x^i$$
where the $\frac{a_i}{b_i}$ are written in lowest terms and $a_n=b_n=1$. Let $d_1=lcm(b_0,b_1,\dots,b_n)$. Then $\alpha_1$ satisfies $$d_1^n\cdot f(a)=\sum_{i=0}^n a_i \frac{d_1^{n-i}}{b_i}(d_1\alpha_1)^i=0.$$
By construction, the $\frac{d_1^{n-1}}{b^i}$ are integers, and the $n-$th coefficient is $1$. This defines a monic polynomial in integer coefficients satisfied by $d_1\alpha_1$. Repeating this process for each $\alpha_i$ gives a set of algebraic integers $\{d_1\alpha_1,\dots,d_n\alpha_n\}$ which is also a basis for $R$.
\end{proof}

As argued in \cite{NumberFields}, this shows that $R$ contains the free abelian group of rank $n$

$$d_1\alpha_1\Z\times d_2\alpha_2\Z\times \dots \times d_n\alpha_n\Z.$$

\begin{definition}
For any $n-$tuple of elements $(\alpha_1,\alpha_2,\dots \alpha_n)$ of $K$, the \defn{discriminant} is 
$$disc(\alpha_1,\alpha_2,\dots,\alpha_n)=|[\sigma_i(\alpha_j)]|^2$$
i.e. the square of the discriminant of the matrix with $(i,j)$th element $\sigma_i(\alpha_j)$. We also introduce the shorthand $disc(\omega)$=$disc(1,\omega,\omega^2,\dots,\omega^{p-1})$ for $p=e^{2\pi i/p}$.
\end{definition}

\begin{prop}\label{number-ring-contained-in-free-abelian-group}
Fix an integral basis $\{\alpha_1,\alpha_2,\dots,\alpha_n\}$ of $B=\A\cap K$ over $\Q$. Let $d=disc(\alpha_1,\dots,\alpha_n)$. Then any $r\in R$ can be written as
$$r=\frac{m_1\alpha_1+\dots+m_n\alpha_n}{d}$$ for integers $m_i$ s.t. $d|m_i^2$.
\end{prop}
\begin{proof}
Write $r=q_1\alpha_1+\dots+q_n\alpha_n, q_i\in \Q$. Let $\sigma_1,\dots \sigma_n$ be the embeddings of $K$ in $\C$. Applying each $\sigma_i$ to the above equation gives  a matrix equation
$$\begin{pmatrix}\sigma_1(r)\\\sigma_2(r)\\\dots\\\sigma_n(r)\end{pmatrix}=\begin{pmatrix}\sigma_1(\alpha_1)&\sigma_1(\alpha_2)&\dots&\sigma_1(\alpha_n)\\\sigma_2(\alpha_1)&\sigma_2(\alpha_2)&\dots&\sigma_2(\alpha_n)\\&\dots\\\sigma_n(\alpha_1)&\sigma_n(\alpha_2)&\dots&\sigma_n(\alpha_n)\end{pmatrix}
\begin{pmatrix}q_1\\q_2\\\dots \\q_n
\end{pmatrix}
$$
We know there is a solution $(q_1,q_2,\dots,q_n)$ to this matrix equation, so Cramer's rule gives us a formula $$q_i=\frac{det(A_i)}{|\sigma_i(\alpha_j)|}=:\frac{\gamma_i}{\zeta}$$ where $A_i$ is the above matrix with the $i$th column replaced by $\begin{pmatrix}\sigma_1(r)\\\sigma_2(r)\\\dots\\\sigma_n(r)\end{pmatrix}$. We note $\gamma_i$ and $\zeta$ are both algebraic integers: both determinants expand to a complicated product and sum of algebraic integers, but $\A$ is a ring so $\gamma_i,\zeta\in \A$. By definition, $d=\zeta^2$. Therefore, $$dq_i=\zeta^2\frac{\gamma_i}{\zeta}=\zeta\gamma_i\in \A.$$ Since $dq_i\in \Q$, necessarily $dq_i\in \Z$: clearly the integers are algebraic integers and the rational non-integers are not, as a rational $\rho\not \in \Z$ has monic irreducible $x-\rho\not \in \Z[x]$, so \cref{lemma-monics-are-Z} shows $\rho$ cannot satisfy a monic in integer coefficients. Therefore we can define $m_i=dq_i\in \Z$ as required. Note $$\gamma_i^2=q_i\zeta \gamma_i=dq_i^2=\frac{m_i^2}{d}$$ Therefore the rational number $\frac{m_i^2}{d}\in \A$, so it is in fact an integer, that is, $d|m_i^2$ as required. 
\cite{NumberFields}
\end{proof}
This shows that $\A\cap K\subset \frac{\alpha_1}{d}\Z\times \frac{\alpha_2}{d}\Z\times \dots \times \frac{\alpha_n}{d}\Z$. Let us summarise what we have shown in the following corollary.
\begin{corollary}\label{number-rings-are-Noetherian}
Any number ring $\A\cap K$ is a free abelian group of rank $n=[\Q[\alpha]:\Q]$. In particular, number rings are Noetherian by \cref{finitely-gen-module-Noetherian}.
\end{corollary}

Next, we give a proof of the following exercise from \cite{NumberFields}

%\begin{prop}\label{disc-invariant-on-generating-sets}
%If $\{\beta_1,\dots,\beta_n\}$ and $\{\gamma_1,\dots,\gamma_n\}$ are subsets of a number field $K$ generating the same additive subgroup $G\leq K$ of a number field, then $$disc(\beta_1,\beta_2,\dots,\beta_n)=disc(\gamma_1,\gamma_2,\dots,\gamma_n).$$
%\end{prop}
%\begin{proof}
%(...)
%\end{proof}
%As remarked in \cite{NumberFields}, this justifies defining $disc(G):=disc(\beta_1,\dots,\beta_n)$ where the $\beta_i$'s generate $G$, since $disc$ is invariant on sets generating the same group. In particular, $disc(R)$ is an invariant of any number ring $R$.

We are now ready to give the proof of \cref{algebriac-integers-are-Z[w]}. We follow the approach of \cite{NumberFields}, first proving two lemmas.
\begin{lemma}
Let $\omega=e^{2\pi i/p}$ for some prime $p.$ Then $$\Z[\omega]=\Z[1-\omega]$$ $$disc(\omega)=disc(1-\omega)$$
\end{lemma}
\begin{proof}
It is clear that $\Z[1-\omega]\subset\Z[\omega]$ by expanding an arbitrary polynomial in $1-\omega$ to give a polynomial in $\omega$. Additionally, since $\omega=1-(1-\omega)$, an arbitrary polynomial in $\omega$ can be expanded to a polynomial in $(1-\omega).$ Therefore $\Z[\omega]\subset \Z[1-\omega]$.

Ordering the embeddings $\sigma_1,\dots \sigma_n$ such that $\sigma_i(\omega)=\omega^i$, we have that
$$disc(\omega)=|\sigma_i(\omega^j)|^2=|\omega^{ij}|^2=\prod_{1\leq r<s\leq n} (\omega^s-\omega^r)$$ where the last equality is the equation for a Vandermonde determinant \cite{NumberFields}. It is a fact that the conjugates of $(1-\omega)$ over $\Q$ are the $(1-\omega^j)'s, 1\leq j< p$ (see \cite{NumberFields}), so
$$disc(\omega)=\prod_{1\leq r<s\leq n} (\omega^s-\omega^r)=\prod_{1\leq r<s\leq n} ((1-\omega^s)-(1-\omega^r))=disc(1-\omega).$$

\cite{NumberFields}
\end{proof}
\begin{lemma}\label{product-of-1-omegak-p}
$$\prod_{k=1}^{p-1}(1-\omega^k)=p$$
\end{lemma}
\begin{proof}
We have already shown in \cref{ex-omega-algebraic}
$$\prod_{k=1}^{p-1}(x-\omega^k)=1+x^2+\dots+x^{p-1}$$
so the result follows by setting $x=1$.
\end{proof}

\begin{proof}[Proof of \cref{algebriac-integers-are-Z[w]}]

Let $\alpha\in B=\A\cap\Q[\omega].$ By \cref{number-ring-contained-in-free-abelian-group}, 
$$\alpha=\frac{m_0+m_1(1-\omega)+\dots+m_{p-1}(1-\omega)^{p-1}}{d}$$ where $m_i\in \Z$, $d=disc(1-\omega)=disc(\omega)$ and $d|m_i^2$. It is shown in \cite{NumberFields} that $$disc(\omega)=p^p.$$ If $\A\cap\Q[\omega]\neq \Z[\omega]=\Z[1-\omega]$ then there must exist such an $\alpha$ where $d\not | m_i$ for some $i$. We may let $i$ be the smallest $i$ where this holds, and, by subtracting $\frac{m_0}{d}+\frac{m_1}{d}(1-\omega)+\dots+\frac{m_{i-1}}{d}(1-\omega)^{i-1}\in \Z[1-\omega]\subset B$, and multiplying by $p^{p-1}$, assume $$\alpha=\frac{m_i(1-\omega)^i+\dots+m_{p-1}(1-\omega)^{p-1}}{p}$$ with $p\not | m_i$. By \cref{product-of-1-omegak-p}, $$\frac{p}{(1-\omega)^{i+1}}=\frac{\prod_{k=1}^{p-1} (1-\omega^k)}{(1-\omega)^{i+1}}=(\prod_{k=1}^{i+1}\frac{1-\omega^k}{1-\omega})\prod_{k=i+1}^{p-1}(1-\omega^k).$$
This product lies in $\Z[1-\omega]$ since the $k$ roots of $(x^k-\omega^k)$ are $\omega,\omega^2,\dots,\omega^k$ giving the identity $(x-\omega^k)=(x-\omega)(x-\omega^2)\dots(x-\omega^k)$ which for $x=1$ gives $\frac{1-\omega^k}{1-\omega}=(x-\omega^2)\dots(x-\omega^k)\in \Z[1-\omega]\subset B.$

It follows that $\alpha \frac{p}{(1-\omega)^{i+1}}\in R.$ Note

$$\alpha \frac{p}{(1-\omega)^{i+1}}=\frac{m_i}{1-\omega}+m_{i+1}+m_{i+2}(1-\omega)+\dots+m_{p-1}(1-\omega)^{p-i-2}\in R$$
By subtracting everything else, which is in $\Z[1-\omega]\subset B$, we find $\frac{m_i}{1-\omega}\in B$.

Since $\frac{m_i}{1-\omega}$ is an algebraic integer, taking the norm over $\Q[\omega]$, $N(\frac{m_i}{1-\omega})=\frac{N(m_i)}{N(1-\omega)}\in \Z$. $N(1-\omega)=\prod(1-\omega^k)=p$ by \cref{product-of-1-omegak-p}. Additionally, $N(m_i)=m_i^p$ since there are $p$ embeddings of $\Q[\omega]$ fixing $\Q$ and each fixes $m_i$. It follows that $p|m_i^p$. However, $p\not | m_i | m_i^p$ by assumption, a contradiction. \cite{NumberFields}
\end{proof}



\chapter{Dedekind Domains}
After our tour of ideals, Noetherian rings, number fields and number rings, we are now ready to bring everything back to Kummer's partial proof of Fermat's Last Theorem. Recall that 

We are now ready to introduce the notion of a \defn{Dedekind domain}. This ad-hoc-seeming definition turns out to be exactly the condition needed for (FILL OUT), as we will show in section (MISSING). Our main goal for this section is to prove that number rings are Dedekind domains (or only p-th cyclotomics???)

\begin{definition}
(...)
\end{definition}



\begin{proposition}
Any number ring $R=\A\cap \Q[\alpha]$ is a Dedekind domains.
\end{proposition}
\begin{proof}
\begin{itemize}
    \item ($R$ is Noetherian). This was shown in  \cref{number-rings-are-Noetherian}.
    \item Every nonzero prime ideal of $R$ is maximal.
    \item ($R$ is integrally closed )
\end{itemize}
\end{proof}
\section{Number rings are Dedekind domains}

\section{Unique factorisation of ideals}
\chapter{Regular Primes}

\chapter{Conclusion}
Kummer's contributions around this proof have been described as the "first peak" \cite{History-algebraic} of the theory of algebraic number fields, while Dedekind is credited with formulating and proving the most fundamental theorems of the field of algebraic number theory \cite{History-algebraic}. Indeed, as we have hopefully shown, their contributions go far beyond a proof of Fermat's Last Theorem for a subset of exponents. Just motivated by \cref{Fermat-Theorem}, we have managed to find a condition, being Noetherian, guaranteeing that the elements of a ring factorise into a finite product of irreducibles, and shown how these can be built from each other. We have classified all the finite dimensional number fields, showing they can always be written in the form $\Q[\alpha]$, and then determined the group structure under $+$ of all number rings. We have then classified exactly the type of integral domains, namely Dedekind domains, whose ideals factorise uniquely into a finite product of prime ideals, and shown that every number ring is of this form. We have then finished by giving some preliminary results in the study of the ideal class group of a number ring.

These were the main players of our story; $\Z[\omega]$ just came along for the ride. In fact, it seemed to be almost by chance that $\Z[\omega]$ was a number ring and a Dedekind domain. Indeed, $\Z[\alpha],\alpha\in \C$ is not a number ring in general, and it is an interesting problem to classify when it is. A logical first follow-up to our story, however, would be to work towards proofs of \cref{finitely-many-prime-ideals} and \cref{ideals-and-norms} part (2). Both results follow from the theory of split primes, namely, for every prime ideal $P$ of a number ring $B$ there is a unique prime ideal $\langle p \rangle$ of $\Z$ with $P\cap B=\langle p \rangle$, for which we say $P$ lies over $\langle p \rangle$ \cite{NumberFields}. Conversely, every prime ideal $\langle p \rangle$ of $\Z$ lies under finitely many, and at least one, prime ideal $P$ of $B$ \cite{NumberFields}. Of course, there is nothing special about $\Z$ here, and the story can be made more general by replacing $\Z$ with another subring $S.$ These facts, together with further restrictions on the norm $||P||$ arising from this relation, lead us to proofs of our missing theorems, and to more interesting results still.

Suffice it to say, despite a complete proof of Fermat's Last Theorem now being available due to Wiles, Kummer's theory of regular primes is still full of interesting problems, including unsolved ones such as the open question of whether there are infinitely many regular primes. Perhaps this is Kummer's greatest achievement: the ability to mystify and intrigue mathematicians, centuries after his time.
%\addcontentsline{toc}{section}{Appendix}
%\section*{Appendix}
\newpage
\addcontentsline{toc}{chapter}{References}
\bibliography{bibtex}
\end{document}
