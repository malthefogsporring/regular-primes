\chapter{Rings and ideals}\label{sec-rings}
Kummer's proof predated the very concept of an ideal: he worked with what he called "ideal numbers". These were later put on a solid footing by Dedekind when he invented the notion of ideals. Ideals are such a key part of modern algebra, it can be difficult to imagine a time before their invention. In fact, Kummer's story tells so well in the language of Dedekind that we will exclusively tell this side of the story. To this end, we spend this chapter establishing preliminary results about ideals, ring modules, and Noetherian rings, inspired by the account in \cite{Wright}. Some of these results will be revision from undergraduate level algebra. All rings will be assumed to be commutative with multiplicative identity, which ring homomorphisms are assumed to preserve.
\section{Ideals}
Recall the following definitions.
\begin{definition}
An \defn{ideal} $I$ of $R$ is a non-empty subset satisfying the following properties:
\begin{enumerate}[(i)]
    \item $I$ is closed under subtraction.
    \item When $i\in I$, for any $r\in R$, $ri\in I$.
\end{enumerate}
\end{definition}
\begin{definition}
Given any subset $S\subset R$, the \defn{ideal generated by} $S$, denoted $\langle S\rangle$, is the smallest ideal of $R$ containing $S$. It is exactly the set $\{r_1s_1+\dots r_ns_n: r_i\in R, s_j\in S\}$ of finite linear combinations of elements in $S$. A \defn{principal ideal} is an ideal generated by a single element.
\end{definition}

\begin{remark}
Since our rings are commutative, we do not need to distinguish between left ideals and right ideals.
\end{remark}

\begin{example}\label{Z-is-PID}
For any $n\in \Z$, $\langle n\rangle =n\Z$ is an ideal of the integral domain $\Z$. In fact, every ideal of $\Z$ is of this form. Take any nonzero ideal $I$ of $\Z$ and let $n$ be the smallest positive integer in $I$. Then given any other $m\in I$, the division and remainder algorithm gives two integers $q,r$ with $m=qn+r$ and $0\leq r<n$. It follows that $r\in I$. However, $n$ was minimal among positive integers in $I$, so in fact $r=0$ and $m\in \langle n \rangle$. Since $m$ was arbitrary, $I=\langle n \rangle.$ It follows that every ideal of $\Z$ is principal. An integral domain with this property is called a \defn{principal ideal domains} or \defn{PID}s for short. 
\end{example}
\begin{example}\label{F[x]-PID} Since the divison and remainder algorithm works in any polynomial ring in one variable over a field \cite{HonoursAlgebra}, an almost identical argument to \cref{Z-is-PID} shows that $F[x]$ is a PID for any field $F$. In particular, given an ideal $I$ of $F[x]$ we can pick a polynomial $f\in I$ of minimal degree. For any $g\in I$ we can find $q,r\in F[x]$ s.t $g=qf+r$ $deg(r)<deg(g)$ by the division and remainder algorithm. Therefore $r\in I$, and since $f$ has minimal degree in $I$, $deg(r)=0$ and $g\in \langle f\rangle$, so $I=\langle f \rangle.$
\end{example}

All rings lie on a sliding scale between being close to being a field and being far from being a field. A ring like the polynomial ring $\Z[x]$ is very far from being a field: the only units (multiplicatively invertible elements) are $\pm 1$. If we want to turn $\Z[x]$ into a field, we can try to declare some of the non-units to be zero by factoring out the ideal generated by these elements. Recall that the \defn{quotient} or \defn{factor ring} of a ring $R$ by an ideal $I$ is the ring of cosets $\{r+I:r\in R\}/\sim$ where $r+I\sim s+I\iff r-s\in I$ and addition and multiplication are defined in the canonical way. Factoring $\Z[x]$ by the ideal $\langle x^2\rangle$ is a bad choice since we miss out on the non-unit $x$, and $x+\langle x^2\rangle$ is still a non-unit in $\Z[x]/\langle x^2 \rangle.$ The failure is due to the fact $\langle x^2\rangle\subset \langle x \rangle$, as we now show: if $r$ is a non-unit and an ideal $I\subset \langle r \rangle$ is an ideal, then $r+I$ is a non-unit in $R/I.$ Indeed, for $s\in R$,
$$(r+I)(s+I)=1+I \iff rs-1\in I\subset \langle r \rangle \implies 1\in \langle r \rangle \iff \langle r \rangle = R\iff r \text{ is a unit }.$$
Therefore to form a field by factoring out an ideal we must factor by a \defn{maximal ideal}: a proper ideal $I$ only contained in $R$ and itself. For example, the maximal ideals of $\Z$ are $p\Z$ where $p$ is prime. We already know that $\Z/n\Z$ is a field if and only if $n$ is prime, so in the ring $\Z$ quotienting by a maximal ideal is not only a sufficient, but also a necessary condition for forming a field. We will show that this is true in general.

\begin{lemma}\label{field-iff-no-ideals}
A ring $R$ is a field if and only if it contains no non-trivial ideals.
\end{lemma}
\begin{proof}
If $R$ is a field and $I$ is an ideal containing $r$, then $1=rr^{-1}\in I$, so $I=R$. Conversely, if $R$ contains no trivial ideals, then for any $r\in R$, $\langle r \rangle = R$. In particular, $1\in \langle r \rangle$ so $1=sr$ for some $s\in R$.
\end{proof}

The correspondence theorem from group theory \cite{GroupTheory} is easily extended to a correspondence theorem for ideals, as we now show.
\begin{lemma}\label{ideal-bijection}[Correspondence Theorem]
For any ideal $I$ of $R$ there is a bijection between the set of ideals of $R$ containing $I$ and the set of ideals of $R/I$.
\end{lemma}
\begin{proof}
The correspondence theorem for a normal subgroup $I\triangleleft R$ says that the canonical group homomomorphism $can:R\rightarrow R/I$ defines a bijection $J\mapsto can(J)$ between subgroups of $R$ containing $I$ and subgroups of $R/I$. This theorem applies to ideals since ideals are always normal subgroups under $+$. $can$ also defines a ring homomorphism, and it is easy to see $J$ is an ideal if and only if $can(J)$ is - we already know $can$ takes abelian groups to abelian groups, so additionally if $J=rJ$ then $J+I=rJ+I$ by bijection on abelian groups and vice versa. Therefore, $J\mapsto can(J)$ is an injection between ideals $J$ containing $I$ and ideals in $R/I$. 
\end{proof} As an immediate consequence of \cref{field-iff-no-ideals} and \cref{ideal-bijection}, we get the following result.
%Recall the correspondence theorem for groups says if $$

%The bijection is given by $\phi:J\mapsto J/I$ where $J$ is an ideal of $R$ containing $I$ and $$J/I:=\{j+I:j\in J\}\subseteq R/I.$$ It is easily confirmed that $J/I$ is an ideal of $R/I$. In particular, it is closed under subtraction since $J$ is, and closed under multiplication by elements of $R/I$ since $J$ is closed under multiplication by elements of $R$.

%$\phi$ is injective: Suppose that $\phi(J)=\phi(S)$ for two ideals $J,S$ containing $I$. Let $j\in J$ and let $s\in S$ be any element such that $j+I=s+I$. Then $j-s\in I\subset S$, so $j\in S$. But $j$ was arbitrary, so $J\subset S$. A similar argument shows $S\subset J$ so $J=S$.
%$\phi$ is surjective: Let $X$ be an ideal of $R/I$. Then $X$ is a set of cosets and we can define %$$J:=\bigcup X.$$ I claim $J$ is an ideal of $R$. We check:
%$$x\in J\implies x+I,y+I\in X\implies (x-y)+I\in X\implies (x-y)\in I\subset J.$$
%$$x\in J\implies x+I\in X\implies (x+I)(r+I)\in X\implies rx+I\in X\implies rx\in J.$$
%Note $\phi(J)=X$, so $\phi$ is surjective. Therefore $\phi$ is bijective.
\begin{prop}\label{ring-over-max-field}
An ideal $I$ of $R$ is maximal if and only if $I/R$ is a field.
\end{prop}
\begin{proof}
$I$ is maximal $\iff$ the only ideals containing $I$ are $I$ and $R$ $\iff$ there are no nontrivial ideals in $R/I$ $\iff$ $R/I$ is a field.

%$(\impliedby)$ Suppose $R/I$ is a field, and $J$  is an ideal with $I\subset J$. Then $J/I:=\{jI: j\in J\}$ is an ideal of $R/I$. By the previous lemma, $J/I=R/I$, so $J=R$.
%$(\implies)$ Conversely, if $I$ is maximal, then $R/I$ does not contain any non-trivial ideals: a non-trivial ideal $JI\subset R/I$ gives rise to a non-trivial ideal $\langle J\cup I\rangle$ of $R$. By the previous lemma, $R/I$ is a field.
\end{proof}
%\begin{remark}
%Zorn's lemma, which is equivalent to the axiom of choice (citation needed), implies that every ring has a maximal ideal. The statement of Zorn's lemma is as follows: (finish this thought...)
%\end{remark}
What if we do not necessarily want to build a field, but are content with an integral domain? Recall that an integral domain is a ring $R$ with no zero divisors. It is exactly the condition that allows us to "cancel out" multiplicatively:
$$ab=ac\implies b=c$$ in an integral domain whenever $a\neq 0$. This is done not by multiplying both sides by $a^{-1}$ (which may not exist), but by noting the ring homomorphism $$f:R\rightarrow R,\quad  b\mapsto ab$$ has kernel $\{0\}$ and is therefore injective. To form an integral domain, an idea is to factor out the set $S$ of all zero divisors of a ring $R$. However, this set does not form an ideal as the difference of two zero divisors need not be a zero divisor. For example, $[2][3]=[0]$ in $\Z_6$ but $[3]-[2]=[1]$ is not a zero divisor. For the same reason, taking the ideal $I=\langle S \rangle$ generated by $S$ is not a good idea either: in the case of $\Z_6$ we would have $I=\Z_6$ whence $\Z_6/I$ is not even a ring. Instead, let us remark that $R/I$ is an integral domain if for $x,y\in R$,
$$(x+I)(y+I)=I\iff x\in I\text{ or } y \in I.$$
Noting $(x+I)(y+I)=I\iff xy\in I$, we can now write down exactly the condition $I$ needs to satisfy for $R/I$ to be an integral domain.

\begin{definition}
A \defn{prime ideal} $I$ of $R$ is a proper ideal such that $xy\in I\implies x\in I$ or $y\in I$. 
\end{definition}

\begin{example}\label{max-is-prime}
The prime ideals of $\mathbb{Z}$ are exactly $p\mathbb{Z}$ where $p$ is prime. If $xy\in p\mathbb{Z}$ then either $x$ or $y$ is divisible by $p$, so either $x$ or $y$ is in $p\mathbb{Z}$. Conversely, in $nm\mathbb{Z}$ where $n,m>1$, $nm\in nm\mathbb{Z}$ but $n,m\not \in nm\mathbb{Z}$. Thus in $\mathbb{Z}$, prime ideals and maximal ideals coincide. In general, a maximal ideal $I$ is always prime, since $I$ is maximal $\iff$ $R/I$ is a field $\implies R/I$ is an integral domain $\implies $ $I$ is prime. However, the converse is not always true. For example, $\langle x \rangle$ is a prime ideal of $\Z[x]$, but it is not maximal, since $$\langle x \rangle\subsetneq \langle x,2\rangle \subsetneq \Z[x].$$
\end{example}
It is a fun exercise to take any ring and try to build an integral domain out of it by factoring out a prime ideal. I give two more original examples.
\begin{example}
The set $C(\R,\R)$ of continuous functions $f:\R\rightarrow \R$ can be given a ring structure, with the usual addition and multiplication (not composition) of functions. The additive identity is the constant function $x\mapsto 0$ and the multiplicative identity is $x\mapsto x$. $C(\R,\R)$ is not an integral domain: take for example two continuous functions
$$f(x)=\begin{cases} 
      0 & x\leq 0 \\
      x & x>0 
   \end{cases}, \qquad g(x)=\begin{cases} 
      x & x\leq 0 \\
      0 & x>0 
   \end{cases}
$$
and note $f(x)g(x)=0$. Let us try to build a prime ideal of $C(\R,\R)$. Let $a\in \R$ and consider the ideal $I_a$ of all continuous functions $f\in C(\R,\R):f(a)=0$. This set is clearly closed by subtraction and multiplication by continuous functions. Additionally, if $f(a)g(a)=0,$ then $f(a)=0$ or $g(a)=0$, so $I_a$ is prime. It follows that $C(\R,\R)/I_a$ is an integral domain! The cosets $[f]$ are exactly the sets of functions $g$ that take the same value $f(a)=g(a)$ at $a$.
\end{example}

\begin{example}
Another strange ring\footnote{Found on \url{https://en.wikipedia.org/wiki/Ring_(mathematics)\#Commutative_rings}. Accessed 18-03-2022.} is constructed as follows: take any set $S$ and let $P(S)$ be the power set of $S$. Let multiplication be intersection $AB=A\cap B$ and let $A+B=(A\cup B)\setminus A\cap B$ be the \defn{symmetric difference}: $x\in A+B\iff x\in A$ or $x\in B$ but not both. Clearly $+$ and $\times$ are commutative, and we have additive identity $\emptyset$ and multiplicative identity $S$. Additive inverses are given by $-A=S\setminus A$. For the distributive law, note $x\in A(B+C)=A\cap (B\cup C/B\cap C)\iff x\in A$ and $x\in B$ or $C$ but not both $\iff$ $x\in A\cap B$ or $x\in A\cap C$ but not both $\iff x\in A\cap B + A \cap C=AB+AC$.

Now clearly $P(S)$ is not an integral domain: $A(S\setminus A)=\emptyset$ for any $A\subset A$, so every element is a zero divisor. Since $0\in P$ for any prime ideal $P$, prime ideals contain all zero divisors. Therefore $P(S)$ has no prime ideals, so $P(S)/I$ is not an integral domain for any $I$. It follows that $P(S)$ has no maximal ideals either. $P(S)$ is, however, a PID! Let $I$ be any ideal of $P(S)$, and note $I$ contains all the singletons $\{x\}$ such that $x\in B$ for some $B\in I$. This holds since $\{x\}=\{x\}B\in I$. Since $I$ is closed under $+,$ $I$ also contains the union $U=\{x:x\in B\text{ for some } B\in I\}$. Letting $A\in I$ we have $A=AU$, so $I\subset \langle U\rangle \subset I$. Therefore $I=\langle U\rangle$ is principal. A very strange ring indeed!
\end{example}

Now let us recall this definition given in \cref{sec-introduction}.
\begin{definition}
An invertible element $r\in R$ is called a \defn{unit}. An element $r\in R$ of an integral domain is called \defn{irreducible} if it is non-zero, not a unit and if $r=xy\implies x$ or $y$ is a unit. $r\in R$ is \defn{reducible} if it is neither $0$, a unit, or irreducible.
\end{definition}
Irreducible and reducible elements generalise the notion of, respectively, prime and composite numbers in $\mathbb{Z}$. The units in $\mathbb{Z}$ are $\pm 1$, and certainly if we allow primes to be negative, $p$ is prime if and only if $p=mn\implies m$ or $n=\pm 1$. In a field $F$ every nonzero element is a unit, so fields do not have (ir)reducibles. The next proposition motivates defining principal ideal domains and irreducible elements.

\begin{prop}\label{PID-over-irreducible-is-field}
If $R$ is a PID, then $0\neq r\in R$ is irreducible $\iff\langle r \rangle$ is maximal $\iff R/\langle r\rangle$ is a field.
\end{prop}
\begin{proof}
The second equivalence is just a restatement of \cref{ring-over-max-field}. We may assume $r$ is not a unit, as $r$ is a unit if and only if $\langle r\rangle=R$, and $R/R=\{0\}$ is not a field. 

$(\implies)$ Let $r$ be irreducible and suppose $\langle r\rangle \subset J$ for some proper ideal $J$. Since $R$ is a PID, $J=\langle s \rangle$ for some non-zero, non-unit $s\in R.$ Therefore $r=ts$ for some $t\in R$. Since $r$ is irreducible and $s$ is not a unit, $t$ is a unit. Therefore $s=t^{-1}r$ so $\langle s \rangle \subset \langle r \rangle.$ Since $J$ was arbitrary, $\langle r \rangle$ is maximal.

$(\impliedby)$ Now let $\langle r \rangle$ be maximal. Let $r=xy$ and suppose neither $x$ nor $y$ are units. It follows that $\langle r\rangle \neq \langle x \rangle$. Then $$\langle r\rangle \subsetneq \langle x \rangle \subsetneq R,$$ a contradiction. Therefore $r$ is irreducible.
\end{proof}

\begin{example}
In $\mathbb{Q}[x]$, $x^2+1$ is irreducible, since its roots are irrational so it cannot be written as a product of two linear polynomials in $\Q$. It follows that $\frac{\mathbb{Q}[x]}{\langle x^2+1\rangle}$ is a field. In fact, it is isomorphic to $\C$: taking the quotient by $x^2+1$ is enforcing the rule $x^2+1=0$ i.e. defining $x=\sqrt{-1}$. 
\end{example}

\begin{remark}
It is essential that $R$ is a PID. For example, $2$ is irreducible in $\mathbb{Z}[x]$ but $\frac{\mathbb{Z}[x]}{\langle 2\rangle}$ is not a field, as $[x]$ is not a unit. This reflects the fact that $\mathbb{Z}[x]$ is not a PID, take for example the non-principal ideal $\langle 2,x\rangle$.
\end{remark}

\section{Noetherian rings and modules}

\begin{definition}
A ring $R$ is said to be \defn{Noetherian} if every increasing sequence of ideals $$I_1\subset I_2\subset I_3 \subset \dots$$
terminates. 

We can extend this definition to $R-$Modules, recalling an $R-$module is an abelian group $M$ equipped with a (distributive, associative, identity-preserving) scalar multiplication function $R\times M\rightarrow M$. An $R-$module $M$ is \defn{Noetherian} if every increasing sequence of submodules $$N_1\subset N_2\subset N_3\subset \dots$$ terminates.
\end{definition}

Note that $R$ is trivially an $R-$module, and the submodules are exactly the ideals of $R$. Therefore $R$ is a Noetherian ring if and only if it is a Noetherian $R-$module. If $R$ is an $S-$module for some other ring $S$, then the story is more complicated: $R$ may be a Noetherian ring without being a Noetherian $S-$module - see \cref{warning-Noetherian-rings-and-mods}.

\begin{proposition}\label{noetherian-module-equivalences}
The following are equivalent for an $R$-module $M$.
\begin{enumerate}[(i)]
    \item $M$ is Noetherian.
    \item Every non-empty collection of submodules of $M$ has a maximal element.
    \item Every submodule $N$ of $M$ is finitely generated as an $R-$module.
\end{enumerate}
\end{proposition}
\begin{proof}
$(i)\implies (ii)$ easily: a counter-example to $(ii)$ is exactly a collection of submodules containing a non-terminating increasing sequence of ideals. Additionally, $(ii)\implies (i)$ because any increasing sequence of submodules
$$N_1\subseteq N_2\subseteq N_3 \subseteq \dots$$
gives a collection $\{N_i\}_{i\geq 1}$ of ideals, which then has a maximal element.

$(i)\implies (iii)$ a non-finitely generated submodule $N$ gives rise to a non-terminating sequence of submodules. Let $a_1\in N$ and $N_1=\langle a_1\rangle=\{ra_1:r\in R\}$. Then let $a_2\in N\setminus N_1$ and $N_2=\langle a_1,a_2\rangle$. Continue this process for all natural numbers to construct the increasing sequence of submodules. It will never terminate as $N$ is not finitely generated.

$(iii)\implies (i)$ Let $$N_1\subseteq N_2\subseteq \dots$$ be an increasing chain of ideals. Then $\bigcup N_i$ is a submodule of $M$, and therefore finitely generated, say by $\{a_1,\dots a_n\}$. Letting $m$ be the smallest integer such that $$\{a_1,\dots a_n\}\subset N_m,$$ then $N_m$ is a maximal element of the sequence.
\cite{Wright}
\end{proof}

\begin{example}
Any principal ideal domain is Noetherian, as all its ideals are generated by a single element.
\end{example}

\begin{remark}
Any ideal $I$ of a Noetherian ring $R$ is always contained in a maximal ideal: simply take the set of all proper ideals containing $I$. By \cref{noetherian-module-equivalences}, this set has a maximal element, which is then a maximal ideal. If one accepts Zorn's Lemma, then this statement is also true of general rings \cite{Wright}.
\end{remark}

We now give some propositions that allow us to generate Noetherian modules from Noetherian rings. First, a result by Hilbert which relies on the axiom of choice.

\begin{theorem}[Hilbert's basis theorem]\label{Hilbert-basis-theorem}
If $R$ is a Noetherian, so is $R[x]$.
\end{theorem}
\begin{proof}
Suppose $I$ is an ideal of $R[x]$ that is not finitely generated. Then we may choose $f_1\in I$ of minimal degree. Now choose $f_2\in I\setminus \langle f_1 \rangle$ of minimal degree. By assumption, we never have $I=\langle f_1,\dots,f_n\rangle,$ so using the axiom of choice, we may continue this process indefinitely. The sequence $f_1,f_2,\dots$ is of (not necessarily strictly) increasing degree. The sequence of increasing ideals of leading coefficients of the $f_i$'s 
$$\langle a_1\rangle\subset \langle a_1,a_2\rangle \subset \dots$$
terminates since $R$ is Noetherian, say at $a_{n-1}$. It follows that $a_n\in \langle a_1,\dots,a_{n-1}\rangle$, so $a_n=\sum_i^{n-1} r_ia_i$ for some $r_i\in R$. By multiplying by appropriate powers of $x$ we may make the degrees of all the $f_i$'s for $i<n$ equal $deg(f_n)$. Then $$(\sum r_if_i)-f_n\not \in \langle f_1,\dots,f_{n-1}\rangle$$ and has degree less than $f_n$ by construction. This contradicts $f_n$ being of minimal degree. \cite{Hilbert}
\end{proof}

\begin{corollary}
If $R$ is Noetherian, so is $R[x_1,x_2,x_3,\dots,x_n]$
\end{corollary}
\begin{proof}
This follows by induction from Hilbert's basis theorem, by noting $$R[x_1,x_2,x_3,\dots,x_n]=R[x_1,\dots,x_{n-1}][x_n].$$
\end{proof}


\begin{example}
$\mathbb{Z}[x]$ is Noetherian as $\mathbb{Z}$ is. Any field $F$ is clearly Noetherian, therefore $F[x]$ is also Noetherian.
\end{example}

\begin{remark} \label{warning-Noetherian-rings-and-mods}
If a ring $R$ is Noetherian as an $S-module$ for a subring $S$, then $R$ is also Noetherian as a ring: its ideals are submodules, which are finitely generated over $S$ by assumption, hence also finitely generated over $R$ as ideals.

The converse is \textbf{not} true. For example, by \cref{Hilbert-basis-theorem}, $\Z[x]$ is Noetherian as a ring. However, it is not Noetherian as a $\Z$-module. Take for example, the non-terminating increasing sequence of submodules
$$R1 \subset \Z  1\oplus \Z x\subset \Z  1\oplus \Z  x\oplus \Z x^2 \subset \dots $$
\end{remark}

We will also want to show that any finitely generated $R$-module is Noetherian. (Recall $M$ is finitely generated if there is a finite subset ${m_1,\dots,m_n}\subset M$ such that every $m\in M$ can be written as $m=r_1m_1+dots+r_nm_n$ for some $r_i\in R.$) By the remarks of \cref{warning-Noetherian-rings-and-mods}, this will show that any ring that is a finitely generated module over a Noetherian ring is a Noetherian ring. In particular, $\Z[\omega]$ is a Noetherian ring since $\Z$ is. Note that a submodule of a finitely generated module $M$ need not be finitely generated in general. For example, $R=\Z[x_1,x_2,\dots]$ is a finitely generated $R-module$ with generating set $\{1\}$. However, the submodule $\langle x_1,x_2,\dots\rangle$ is not finitely generated as an $R-$module. We therefore cannot use property $(iii)$ of \cref{noetherian-module-equivalences} directly. The Noetherian property of $R$ will have to play a key role. We will proceeed as in \cite{Wright} by first proving a lemma.

\begin{lemma}
Let $M$ be an $R-$module and $N$ be a submodule. Then $M$ is a Noetherian $R-$module if and only if both $N$ and $M/N$ are.
\end{lemma}
\begin{proof}
$(\implies)$ $N$ is Noetherian, since its submodules are submodules of $M$, hence finitely generated. Additionally, a submodule of $M/N$ is of the form $A/N$ for a submodule $A$ of $M$, so if $\{m_1,\dots m_n\}$ generates $A$, then $\{m_1+N,\dots m_n +N\}$ generates $A/N$.

$(\impliedby)$ Let $$M_1\subset M_2\subset \dots$$ be an increasing sequence of submodules of $M$ and note $$M_1\cap N\subset M_2\cap N \dots $$ and $$M_1/N\subset M_2/N\subset \dots$$ are increasing sequences of submodules of respectively $N$ and $M/N$. By assumption, both terminate, say at $M_n\cap N$ and $M_{n'}/N$. Let $m$ be any integer such that both sequences have terminated by the $m$th member. Now let $a\in M_{m+1}$. Since $M_m/N=M_{m+1}/N$, there is a $b\in M_m$ with $a-b\in N$. Note $a-b\in M_{m+1}$ since $M_m\subset M_{m+1}$. Since $M_m\cap N=M_{m+1}\cap N$, $a-b\in M_m$ as well, so $a\in M_m$. Since $a$ was arbitrary, $M_m=M_{m+1}$. Since $m$ was an arbitrary integer with $m>n,m>n'$, in fact $M_m=M_{m'}$ for all $m'>m$ so the sequence terminates.
\cite{Wright}
\end{proof}

\begin{corollary}\label{finitely-gen-module-Noetherian}
If $R$ is Noetherian, then any finitely generated module $M$ over $R$ is Noetherian.
\end{corollary}
\begin{proof}
We prove this by induction on the generating elements of $M$. If $M$ is generated by a single element $m_1$, then we have an ideal $I$ of $R$ given by $I=ker(f)$ for $$f:R\rightarrow M, r\mapsto rm.$$ Since $f$ is surjective, the first isomorphism theorem for $R$-modules gives $M\iso R/I$ and the latter is a Noetherian $R-$module by the lemma, since $R$ is a Noetherian $R-$module.

Now suppose a Noetherian module $M$ is generated by $\{m_1,\dots m_{n-1}\}$ and let $M'$ be generated by $\{m_1,\dots m_{n-1}, m_n\}$. Then $M'/M$ is generated by a single element $\{m_n\}$ and is therefore Noetherian as we have just shown. Since $M$ was Noetherian, $M'$ is Noetherian by the Lemma.
\cite{Wright}. 
\end{proof}

We now reach our main motivation for examining Noetherian modules. 
In $\mathbb{Z}$, we can write any element as a product of primes. In general rings, we want to think about irreducibles instead of primes, so it is natural to ask: in which rings can elements be written as a product of irreducible elements? As it turns out, all Noetherian rings have this property, as proven in \cite{Wright}. The proof given here is my own.
\begin{lemma}\label{noetherian-irreducible-elements}
Any (non-zero, non-unit) element $r\in R$ of a Noetherian ring $R$ can be written as a finite product of irreducible elements
$$r=r_1r_2r_3\dots r_n.$$
\end{lemma}
\begin{proof}
Let $A=\{\text{ proper principal ideals } I \text{ containing } r\}$. $\langle r \rangle \in A$ so $A$ is non-empty. Since $R$ is Noetherian, $A$ has a maximal element $\langle r_1\rangle$. I claim $r_1$ is irreducible: it is not a unit since $\langle r_1 \rangle$ is a proper ideal. Furthermore, if $r_1=ab$ for non-units $a,b$ then $a\not\in \langle r \rangle$ since $b$ is not a unit. Then $r\in \langle a \rangle \not \subset \langle r_1 \rangle$ which contradicts $r_1$ being maximal among proper principal ideals containing $r$. This is all to show that there exists an irreducible dividing $r$. Now write $r=r'_2r_1$ and repeat the process on $r'_2$ to find an irreducible $r_2$ dividing $r'_2$. This process either stops with $r'_n$ being a unit, giving a finite factorisation $r=u r_1r_2r_3\dots r_n$ for a unit $u$, or it gives an increasing chain
$$\langle r_1\rangle\subset \langle r_1r_2\rangle \subset \dots$$ which by the Noetherian property terminates: say $\langle r_1r_2r_3\dots r_{m-1}\rangle = \langle r_1\dots r_m\rangle$. Then $r_1r_2\dots r_{m-1}=sr_1r_2\dots r_m$ for some $s\in R$, so since $R$ is an integral domain, $1=sr_{m}$ so $r_m$ is a unit, a contradiction.
\end{proof}

We are mainly interested in Noetherian rings because of this factorisation property. However, not only Noetherian rings have this property. As an example, let $x\in \Z[x_1,x_2,\dots]$, the ring of polynomials in infinitely many variables, which we have already argued is not Noetherian. Nonetheless, since $x$ is also in the Noetherian ring $\Z[x_1,x_2,\dots,x_n]$ for some $n$ depending on $x$, $x$ factorises into a product of irreducibles in $\Z[x_1,x_2,\dots,x_n]$, which are easily seen to also be irreducible in $\R[x_1,x_2,\dots]$. $\Z[x_1,x_2,\dots]$ therefore has the factorisation property despite not being Noetherian.

%\begin{remark}
%As we will sho prime decompositions in $\mathbb{Z}$ are unique, but this is not necessarily the case in Noetherian rings! For example, $\dots$. Noetherian rings where the decomposition into irreducibles is unique (up to reordering and multiplication by units) are called \defn{unique factorisation domains} or \defn{UFDs}. The main failure of Lamé's proof is that number rings are not UFDs. For example $\dots$
%\end{remark}
%\begin{proof}
%\end{proof}

\begin{definition}
An integral domain $R$ is a unique factorisation domain (UFD) if every non-zero non-unit $r\in R$ can be written as a product of irreducible elements $r=r_1r_2\dots r_n$ \textbf{uniquely} up to reordering and multiplication by units.
\end{definition}
\begin{definition}
A non-zero non-unit $p\in R$ of an integral domain is \defn{prime} if $p|ab\implies p|a$ or $p|b$. 
\end{definition}
\begin{remark}
Note $p\in R$ is prime exactly when $\langle p \rangle $ is a prime ideal, since $p|a \iff a\in \langle p \rangle$.
\end{remark}


%In the previous section, we showed that all PIDs are UFDs. Unfortunately, we also saw that PID is too strict a requirement for all number rings. Number rings are Noetherian, however, as we have just shown. Our hope is that there is that there is an easier requirement to satisfy being a UFD, sitting between Notherian rings and PIDs. Such a requirement exists - but first we need a definition.

We argued that the correct notion of "primeness" was irreducibility. We can show that this new notion is strictly stronger:  every prime element is irreducible: Let $r$ be prime and $r=xy$. Then $r|xy$, so without loss of generality, $r|x\implies x=cr$ for some $c\in R$. Then $$r=rcy\implies r(1-cy)=0\implies cy=1 \text{ (integral domain) }\implies y \text{ is a unit.}$$ Additionally, as is easily shown, $\langle p \rangle$ is a prime ideal. As we now show, prime elements give rise to a easier UFD condition.

\begin{proposition}\label{Noetherian-prime-UFD}
A Noetherian ring $R$ is a UFD if and only if every irreducible element $r\in R$ is prime.
\end{proposition}
\begin{proof}
$(\implies)$ Let $r\in R$ be irreducible and let $r|xy$. By uniqueness of the factorisation into irreducibles, $r$ appears in the unique factorisation of either $x$ or $y$, so $r|x$ or $r|y$.

$(\impliedby)$ By \cref{noetherian-irreducible-elements}, any element has a decomposition into irreducibles. We only need to take care of uniqueness. Suppose $r=r_1r_2r_3\dots r_n=s_1s_2\dots s_m$ are two decompositions into irreducibles. Since $r_1$ is prime, $r_1|s_i$ for some $i$. Since $s_i$ is irreducible, $s_i=ur_1$ for some unit $u$. Since $R$ is an integral domain, $$r_2r_3\dots r_n=us_2s_3\dots s_m,$$ where we have relabeled the $s_i$'s. We can define $r_2'=r_2u^{-1}$, which is also irreducible, then continue as before. If $m=n$ we have shown that every $r_i=u_is_j$ for some $j$ and unit $u_i$, as required. Otherwise, WLOG, suppose $m>n$. Then at the end, after relabeling the $s_i$'s, we have
$$\bar u= s_1\dots s_k.$$ Then $s_i|\bar{u}\implies \bar{u}=zs_i$ for some $z$. Then $(\bar{u}^{-1}z)s_i=1,$ so $s_i$ is a unit, a contradiction. \cite{Wright}
\end{proof}

It is an easy corollary that every PID is a UFD, as we now remark.

\begin{corollary}\label{PIDs-are-UFDs}
Any principal ideal domain $R$ is a UFD.
\end{corollary}
\begin{proof}
PIDs are integral domains by definition and we have already remarked that they are Noetherian. For an irreducible $r\in R$, $\langle r \rangle$ is maximal by \cref{PID-over-irreducible-is-field}, in particular it is prime by \cref{max-is-prime}. It follows that $r$ is prime.
\end{proof}

\begin{example}\label{Z_p[x]-is-UFD}
By \cref{F[x]-PID}, $\Z_p[x]$ is a PID where $p$ is prime and $\Z_p$ is the finite field of $p$ elements. It follows that $\Z_p[x]$ is a UFD.
\end{example}
