\chapter{Rings and ideals}\label{sec-rings}
Kummer's rectification relies on (or rather, developed) the rich commutative ring theory. We will now give an exposition of these results, roughly following \cite{Wright}.
\section{Ideals}
All rings will be assumed to be commutative with multiplicative identity. Recall the definition of an \defn{ideal} $I$:
\begin{definition}
An \defn{ideal} $I$ of $R$ is a non-empty subset satisfying the following properties:
\begin{enumerate}[(i)]
    \item $I$ is closed under subtraction.
    \item When $i\in I$, for any $r\in R$, $ri\in I$.
\end{enumerate}
\end{definition}

\begin{remark}
Condition (i) can be replaced by $(I,+)$ being a subgroup of $(R,+)$, as $x+y=x-(-1)\cdot y$ and $-y=0-y$.
\end{remark}

\begin{remark}
Since our rings are commutative, we do not need to distinguish between left ideals and right ideals. Additionally, $(I,+)$ is automatically abelian, since $(R,+)$ is.
\end{remark}

\begin{example}
For any $s\in R$ we can define the \defn{ideal generated by $s$} as the subset $\langle s\rangle =\{rs:r\in R\}$. $\langle s\rangle$ is a subset of every ideal containing $s$, and is hence the smallest ideal containing $s$. Ideals of this form are called \defn{principal ideals}.
\end{example}
\begin{example}
When $R=\mathbb{Z}$, the ideals are exactly $n\mathbb{Z}$ for $n\in \mathbb{Z}$. These subsets clearly satisfy the requirements of an ideal, and conversely, any ideal $I$ of $\mathbb{Z}$ has a least non-negative element $n$, and this element then generates the ideal, as if it doesn't it can be shown to not be the least element. We therefore see that every ideal of $\mathbb{Z}$ is a principal ideal. Rings of this form are called \defn{principal ideal domains} or \defn{PID}s for short.
\end{example}
\begin{example}\label{F[x]-PID}
For any field $F$, $F[x]$ is a principal ideal domain. (...) This follows from the fact that the division and remainder algorithm works in any polynomial in one variable over a field \cite{HonoursAlgebra}. Let $I$ be an ideal of $F[x]$ and choose a nonzero polynomial of least degree $p(x)\in I$. I claim $\langle p(x)\rangle =I$. Let $q(x)\in I$. By long division, $q(x)=p(x)h(x)+r(x)$ for some $h(x)$ and $r(x)$ where $deg(r)<deg(p)$ or $r=0$. It follows that $r(x)\in I$, but $p(x)$ was of least degree, so $r(x)=0$. Therefore $q(x)\in \langle p(x)\rangle$. Since $q(x)$ was arbitrary, $I=\langle p(x)\rangle$.
\end{example}


Ideals tell you how close a ring is to being a field, as the following lemma shows.
\begin{lemma}\label{field-iff-no-ideals}
A ring $R$ is a field iff it contains no trivial ideals.
\end{lemma}
\begin{proof}
$(\implies)$ if $I$ is an ideal of $R$, then for any non-zero $x\in I$, and any $r\in R$, $r=(rx^{-1})x\in I$, so $I=R$.

$(\impliedby)$ If $R$ contains no non-trivial ideals, then for any $r\in R$, $\langle r\rangle =R$, so $\exists s\in R \text{ such that } rs=1$.
\end{proof}
Recall the following definition.
\begin{definition}
For an ideal $I$ of $R$ the \defn{quotient ring of $R$ by $I$} is the factor group $R/I$ with extra multiplicative structure given by $(x+I)(y+I)=xy+I$.
\end{definition}

\begin{lemma}\label{ideal-bijection}
For any ideal $I$ of $R$ there is a bijection between the set of ideals of $R$ containing $I$ and the set of ideals of $R/I$.
\end{lemma}
\begin{proof}
The bijection is given by $\phi:J\mapsto J/I$ where $J$ is an ideal of $R$ containing $I$ and $$J/I:=\{j+I:j\in J\}\subseteq R/I.$$ It is easily confirmed that $J/I$ is an ideal of $R/I$. In particular, it is closed under subtraction since $J$ is, and closed under multiplication by elements of $R/I$ since $J$ is closed under multiplication by elements of $R$.

$\phi$ is injective: Suppose that $\phi(J)=\phi(S)$ for two ideals $J,S$ containing $I$. Let $j\in J$ and let $s\in S$ be any element such that $j+I=s+I$. Then $j-s\in I\subset S$, so $j\in S$. But $j$ was arbitrary, so $J\subset S$. A similar argument shows $S\subset J$ so $J=S$.

$\phi$ is surjective: Let $X$ be an ideal of $R/I$. Then $X$ is a set of cosets and we can define $$J:=\bigcup X.$$ I claim $J$ is an ideal of $R$. We check:
$$x\in J\implies x+I,y+I\in X\implies (x-y)+I\in X\implies (x-y)\in I\subset J.$$
$$x\in J\implies x+I\in X\implies (x+I)(r+I)\in X\implies rx+I\in X\implies rx\in J.$$

Note $\phi(J)=X$, so $\phi$ is surjective. Therefore $\phi$ is bijective.\end{proof}

\begin{definition}
An ideal $I\neq R$ of $R$ is called a \defn{maximal ideal} if the only ideal it is contained in is $R$ and itself.
\end{definition}

\begin{example}
The maximal ideals of $\mathbb{Z}$ are $p\mathbb{Z}$ where $p$ is prime.
\end{example}
As an immediate consequence of \cref{field-iff-no-ideals} and \cref{ideal-bijection}, we get the following result.
\begin{lemma}\label{ring-over-max-field}
An ideal $I$ of $R$ is maximal if and only if $I/R$ is a field.
\end{lemma}
\begin{proof}
$I$ is maximal $\iff$ the only ideals containing $I$ are $I$ and $R$ $\iff$ there are no nontrivial ideals in $R/I$ $\iff$ $R/I$ is a field.

%$(\impliedby)$ Suppose $R/I$ is a field, and $J$  is an ideal with $I\subset J$. Then $J/I:=\{jI: j\in J\}$ is an ideal of $R/I$. By the previous lemma, $J/I=R/I$, so $J=R$.

%$(\implies)$ Conversely, if $I$ is maximal, then $R/I$ does not contain any non-trivial ideals: a non-trivial ideal $JI\subset R/I$ gives rise to a non-trivial ideal $\langle J\cup I\rangle$ of $R$. By the previous lemma, $R/I$ is a field.
\end{proof}

\begin{remark}
Zorn's lemma, which is equivalent to the axiom of choice (citation needed), implies that every ring has a maximal ideal. The statement of Zorn's lemma is as follows: (finish this thought...)
\end{remark}

\begin{definition}
A \defn{prime ideal} $I\neq R$ of $R$ is an ideal such that $xy\in I\implies x\in I$ or $y\in I$. 
\end{definition}

\begin{example}
The prime ideals of $\mathbb{Z}$ are exactly $p\mathbb{Z}$ where $p$ is prime, motivating the name. If $xy\in p\mathbb{Z}$ then either $x$ or $y$ is divisible by $p$, so either $x$ or $y$ is in $p\mathbb{Z}$. Conversely, in $nm\mathbb{Z}$ where $n,m>1$, $nm\in nm\mathbb{Z}$ but $n,m\not \in \mathbb{Z}$. Thus in $\mathbb{Z}$ prime ideals and maximal ideals are the same.
\end{example}
Factoring out a prime ideal always gives an integral domain.
\begin{lemma}\label{ring-over-prime-int-domain}
An ideal $I$ of $R$ is prime $\iff$ $R/I$ is an integral domain.
\end{lemma}
\begin{proof}
The two statements are equivalent. Both are saying
$$(x+I)(y+I)=I\iff xy\in I\implies x\in I \text{ or } y\in I.$$
\end{proof}
\begin{remark}\label{max-is-prime}
\cref{ring-over-max-field} and \cref{ring-over-prime-int-domain} together show that every maximal ideal is prime, as every field is an integral domain.
\end{remark}

\begin{definition}
An invertible element $r\in R$ is called a \defn{unit}. An element $r\in R$ is called \defn{irreducible} if it is not 0, not a unit (invertible in $R$) and if $r=xy\implies x$ or $y$ is a unit. $r\in R$ is called \defn{reducible} if it is not 0, a unit, or irreducible.
\end{definition}
Irreducible and reducible elements generalise the notion of respectively prime and composite numbers in $\mathbb{Z}$. The units of $\mathbb{Z}$ are $-1$ and $1$, and certainly $p$ is (a negative of a) prime if and only if $p=xy\implies x$ or $y=\pm 1$. In a field $F$ every nonzero element is a unit, so fields do not have (ir)reducibles. The next proposition motivates defining principal ideal domains and irreducible elements.

\begin{prop}\label{PID-over-irreducible-is-field}
If $R$ is a PID, then $0\neq r\in R$ is irreducible if and only if $R/\langle r\rangle$ is a field.
\end{prop}
\begin{proof}
By \cref{ring-over-max-field}, we can instead show $r$ is irreducible $\iff$ $\langle r\rangle$ is maximal. We may assume $r$ is not a unit, as $r$ is a unit if and only if $\langle r\rangle=R$, and $R/R=\{0\}$ is not a field - we require two elements $0\neq 1$. Let $r$ be irreducible. Since $R$ is a PID, it is enough to show show $\langle r\rangle$ contains every proper ideal containing $r$. Let $r\in \langle x\rangle$ for some non-unit $x\in R$. Then $r=xy$ for some $y$. Since $r$ is irreducible, $y$ is a unit. Therefore $x=ry^{-1}$, so $x\in \langle r\rangle$. It follows that $\langle x\rangle \subseteq \langle r\rangle$.

Now let $\langle r \rangle$ be maximal. Let $r=xy$ and suppose neither $x$ nor $y$ are units. Since $x$ is not a unit, $\langle x\rangle \neq R$. Additionally, since $y$ is not a unit, $x\not \in \langle r\rangle$, however $r\in \langle x\rangle$, so $\langle r\rangle \subset \langle x\rangle$ where this inclusion is proper. This contradicts $\langle r\rangle$ being maximal.
\end{proof}

\begin{example}
In $\mathbb{Q}[x]$, $x^2+1$ is irreducible. It follows that $\frac{\mathbb{Q}[x]}{\langle x^2+1\rangle}$ is a field. In fact, it is isomorphic to $\C$: taking the quotient by $x^2+1$ is enforcing the rule $x^2+1=0$ i.e. $x=\sqrt{-1}$.
\end{example}

\begin{remark}
It is essential that $R$ is a PID. For example, $2$ is irreducible in $\mathbb{Z}[x]$ but $\frac{\mathbb{Z}[x]}{\langle 2\rangle}$ is not a field, as $[x]$ is not a unit. This reflects the fact that $\mathbb{Z}[x]$ is not a PID, take for example the non-principal ideal $\langle 1,x^2\rangle$.
\end{remark}

Principal ideal domains are UFDs...


\begin{example}\label{Z_p[x]-is-UFD}
By \cref{F[x]-PID}, $\Z_p[x]$ is a PID where $p$ is prime and $\Z_p$ is the finite field of $p$ elements. It follows that $\Z_p[x]$ is a UFD.
\end{example}

If only number rings were principal ideal domains!! But they are not ...

\section{Noetherian rings}

\begin{definition}
A ring $R$ is said to be \defn{Noetherian} if every increasing sequence of ideals $$I_1\subseteq I_2\subseteq I_3 \subseteq \dots$$
terminates.
\end{definition}

\begin{proposition}\label{noetherian-module-equivalences}
The following are equivalent for a ring $R$.
\begin{enumerate}[(i)]
    \item $R$ is Noetherian.
    \item Every non-empty collection of ideals of $R$ has a maximal element.
    \item Every ideal $I$ of $R$ is finitely generated as an $R-$module.
\end{enumerate}
\end{proposition}
\begin{proof}
$(i)\implies (ii)$ easily: a counter-example to $(ii)$ is a collection of ideals containing a non-terminating increasing sequence of ideals. Additionally, $(ii)\implies (i)$ because any increasing sequence of ideals
$$I_1\subseteq I_2\subseteq I_3 \subseteq \dots$$
gives a collection $\{I_i\}_{i\geq 1}$ of ideals, which then has a maximal element.

$(i)\implies (iii)$ a non-finitely generated ideal $I$ gives rise to a non-terminating sequence of ideals. Let $a_1\in I$ and $I_1=\langle a_1\rangle$. Then let $a_2\in I\setminus I_1$ and $I_2=\langle a_1,a_2\rangle$. Continue this process for all natural numbers to construct the increasing sequence of ideals. It will never terminate as $I$ is not finitely generated.

$(iii)\implies (i)$ Let $$I_1\subseteq I_2\subseteq \dots$$ be an increasing chain of ideals. Then $\bigcup I_i$ is an ideal of $R$, and therefore finitely generated, say by $\{a_1,\dots a_n\}$. Letting $m$ be the smallest integer such that $\{a_1,\dots a_n\}\subset I_m$, $I_m$ is a maximal element of the chain.

\end{proof}

\begin{example}
Any principal ideal domain is Noetherian, as all its ideals are generated by a single element.
\end{example}

\begin{remark}
\cref{noetherian-module-equivalences} shows that any sub-module 
\end{remark}
The following proposition will allow us to construct Noetherian rings from other Noetherian rings. We give an original proof.
\begin{proposition}[Hilbert's basis theorem]
If $R$ is Noetherian, so is $R[x]$.
\end{proposition}
\begin{proof}
Let $I$ be an ideal of $R[x]$ and choose a polynomial of least degree $f_1\in I$. Suppose $I$ is not finitely generated. Then $\langle f_1 \rangle \neq I$, and we can choose $f_2\in I\setminus \langle f_1\rangle$ of least degree. By construction, $deg(f_2)\geq deg(f_1)$. We can again choose $f_3\in I\setminus \langle f_1,f_2\rangle$ of least degree. We can continue this for as long as we want, giving us an increasing sequence of ideals
$$\langle f_1\rangle\subset \langle f_1,f_2\rangle\subset \dots$$ (FINISH)
\end{proof}

\begin{example}
$\mathbb{Z}[x]$ is Noetherian as $\mathbb{Z}$ is. Any field $F$ is clearly Noetherian, therefore $F[x]$ is also Noetherian.
\end{example}

\begin{remark}

\end{remark}

\begin{corollary}
If $R$ is Noetherian, so is $R[x_1,x_2,x_3,\dots,x_n]$
\end{corollary}
\begin{proof}
This follows by induction from Hilbert's basis theorem, by noting $$R[x_1,x_2,x_3,\dots,x_n]=R[x_1,\dots,x_{n-1}][x_n].$$
\end{proof}

In $\mathbb{Z}$, we can write any element as a product of primes. In general rings, we want to think about irreducibles instead of primes. It is therefore natural to ask: in which rings can elements be written as a product of irreducible elements? As it turns out, all Noetherian rings have this property.

\begin{lemma}\label{noetherian-irreducible-elements}
Any (non-zero, non-unit) element $r\in R$ of a Noetherian ring $R$ can be written as a finite product of irreducible elements
$$r=r_1r_2r_3\dots r_n.$$
\end{lemma}
\begin{proof}

\end{proof}

\begin{remark}
The prime decompositions in $\mathbb{Z}$ are unique, but this is not necessarily the case in Noetherian rings! For example, $\dots$. Noetherian rings where the decomposition into irreducibles is unique (up to reordering and multiplication by units) are called \defn{unique factorisation domains} or \defn{UFDs}. The main failure of Lamé's proof is that number rings are not UFDs. For example $\dots$
\end{remark}

\begin{proof}

\end{proof}

In the previous section, we showed that all PIDs are UFDs. Unfortunately, we also saw that PID is too strict a requirement for all number rings. Number rings are Noetherian, however, as we have just shown. Our hope is that there is that there is an easier requirement to satisfy being a UFD, sitting between Notherian rings and PIDs. Such a requirement exists - but first we need a definition.

\begin{definition}
A non-zero non-unit $r\in R$ is \defn{prime} if $p|ab\implies p|a$ or $p|b$.
\end{definition}

We argued that the correct notion of "primeness" was \textbf{irreducibility.} In integral domains, this notion we have just defined is stronger - every prime element is irreducible: Let $r$ be prime and $r=xy$. Then $r|xy$, so without loss of generality, $r|x\implies x=cr$ for some $c\in R$. Then $$r=rcy\implies r(1-cy)=0\implies cy=1 \text{ (integral domain) }\implies y \text{ is a unit.}$$ Additionally, as is easily shown, $\langle p \rangle$ is a prime ideal. As we now show, prime elements give rise to a easier UFD condition:

\begin{proposition}\label{Noetherian-prime-UFD}
Let $R$ be a Noetherian ring and integral domain such that every irreducible element is prime. Then $R$ is a UFD.
\end{proposition}
\begin{proof}
By \cref{noetherian-irreducible-elements}, any element has a decomposition into irreducibles. We only need to take care of uniqueness. Suppose $r=r_1r_2r_3\dots r_n=s_1s_2\dots s_m$ are two decompositions into irreducibles. Since $r_1$ is prime, $r_1|s_i$ for some $i$. Since $s_i$ is irreducible, $s_i=ur_1$ for some unit $u$. Since $R$ is an integral domain, $$r_2r_3\dots r_n=us_2s_3\dots s_m,$$ where we have relabeled the $s_i$'s. We can define $r_2'=r_2u^{-1}$, which is also irreducible, then continue as before. If $m=n$ we have shown that every $r_i=u_is_j$ for some $j$ and unit $u_i$, as required. Otherwise, WLOG, suppose $m>n$. Then at the end, after relabeling the $s_i$'s, we have
$$\bar u= s_1\dots s_k.$$ Then $s_i|\bar{u}\implies \bar{u}=zs_i$ for some $z$. Then $(\bar{u}^{-1}z)s_i=1,$ so $s_i$ is a unit, a contradiction.
\end{proof}

\begin{example}
(...)
\end{example}

Armed with \cref{Noetherian-prime-UFD}, we can check which number rings have this condition. These number rings will satisfy Lamé's proof.
%\begin{definition}
%A ring $R$ is said to be Noetherian if it is Noetherian as an $R-module$.
%\end{definition}

%Note that the submodules of $R$ are exactly the ideals of $R$. Therefore \cref{noetherian-module-equivalences} can be rewritten using ideals for the case that $M$ is the ring $R$ itself. 
