\chapter{Conclusion}
Kummer's contributions around this proof have been described as the "first peak" \cite{History-algebraic} of the theory of algebraic number fields, while Dedekind is credited with formulating and proving the most fundamental theorems of the field of algebraic number theory \cite{History-algebraic}. Indeed, as we have hopefully shown, their contributions go far beyond a proof of Fermat's Last Theorem for a subset of exponents. Just motivated by \cref{Fermat-Theorem}, we have managed to find a condition, being Noetherian, guaranteeing that the elements of a ring factorise into a finite product of irreducibles, and shown how these can be built from each other. We have classified all the finite dimensional number fields, showing they can always be written in the form $\Q[\alpha]$, and then determined the group structure under $+$ of all number rings. We have then classified exactly the type of integral domains, namely Dedekind domains, whose ideals factorise uniquely into a finite product of prime ideals, and shown that every number ring is of this form. We have then finished by giving some preliminary results in the study of the ideal class group of a number ring.

These were the main players of our story; $\Z[\omega]$ just came along for the ride. In fact, it seemed to be almost by chance that $\Z[\omega]$ was a number ring and a Dedekind domain. Indeed, $\Z[\alpha],\alpha\in \C$ is not a number ring in general, and it is an interesting problem to classify when it is. A logical first follow-up to our story, however, would be to work towards proofs of \cref{finitely-many-prime-ideals} and \cref{ideals-and-norms} part (2). Both results follow from the theory of split primes, namely, for every prime ideal $P$ of a number ring $B$ there is a unique prime ideal $\langle p \rangle$ of $\Z$ with $P\cap B=\langle p \rangle$, for which we say $P$ lies over $\langle p \rangle$ \cite{NumberFields}. Conversely, every prime ideal $\langle p \rangle$ of $\Z$ lies under finitely many, and at least one, prime ideal $P$ of $B$ \cite{NumberFields}. Of course, there is nothing special about $\Z$ here, and the story can be made more general by replacing $\Z$ with another subring $S.$ These facts, together with further restrictions on the norm $||P||$ arising from this relation, lead us to proofs of our missing theorems, and to more interesting results still.

Suffice it to say, despite a complete proof of Fermat's Last Theorem now being available due to Wiles, Kummer's theory of regular primes is still full of interesting problems, including unsolved ones such as the open question of whether there are infinitely many regular primes. Perhaps this is Kummer's greatest achievement: the ability to mystify and intrigue mathematicians, centuries after his time.