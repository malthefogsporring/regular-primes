\section{Introduction}
In theorems that are easy to state and difficult to prove, Fermat's Last Theorem likely takes the crown. The Theorem, first stated in 1637, was never proven by Fermat, although he claimed to have a "a truly marvelous proof" (CITE), which the margins of his copy of Arithmetica were "too narrow to contain." It is very doubtful that he had such a proof - indeed, many mathematicians since then have thought they had a proof, only to find a slight error making the entire proof crumble. A correct proof was not published until 1995, by Sir Andrew Wiles, and this proof was an epic achievement bringing together our most advanced understanding of elliptic curves (CITE). Despite the difficulty of the proof, Fermat's Last Theorem is so easy to state, a 9-year-old can understand it.

\begin{theorem}[Fermat]
The equation
\begin{equation}\label{Fermat}
    x^n+y^n=z^n
\end{equation}
has no integer solutions for $n>2$.
\end{theorem}

The case $n=2$ is Pythagoras' Theorem, which has integer solutions. For example, $(x,y,z)=(3,4,5)$ works. It is enough to prove Fermat's Last Theorem for the case that $n$ is an odd prime. This is because a counter-example for $n=pm$ a composite number also gives a counter-example for a prime:
$$(x^m)^p+(y^m)^p=(z^m)^p.$$
(Technically, we should also prove the case $n=4$, as we know Fermat does not work for $n=2$ - however this turns out not to be hard). We can additionally assume that in a solution $(x,y,z)$, $x$, $y$ and $z$ are all coprime. Indeed, if they all three share a factor, we can divide by that factor to get another counter-example. If instead, two of three share a factor $n$, then $\cref{Fermat}$ will not hold $(\Mod n)$. We therefore reduce the proof to two types of solutions:
\begin{itemize}
    \item (type 1): $p$ does not divide either of $x,y,z$.
    \item (type 2): $p$ divides exactly one of $x,y,z$.
\end{itemize}

The case $n=3$ is simple enough to prove using just undergraduate level number theory. First, suppose $(x,y,z)$ is a type 1 solution. Then $x,y,z$ are all $\pm1 \Mod 3$, so $$x^3+y^3\neq z^3 (\Mod 3).$$
(FINISH PROOF)

(PROVE n=4 - exercise 15 on pg 8 in Number Fields).

These cases are truly the only trivial ones - proving Fermat's Theorem for $p>3$ requires more work. Type 1 solutions turn out to be the most interesting to analyse, so from now on we will assume $(x,y,z)$ is a Type 1 solution. Our first trick will be to turn $\cref{Fermat}$ into a multiplicative equation. Letting $\omega=e^{2\pi i/p}$, we factor \cref{Fermat} to
$$(x+y)(x+\omega y)\dots(x+\omega^{p-1} y)=z^p.$$
This follows from noting $1,\omega,\dots,\omega^{p-1}$ are solutions to $t^p-1=0$, and since $\mathbb{C}$ is algebraically closed, this implies
$$t^p-1=(t-1)(t-\omega)\dots (t-\omega^{p-1}).$$
Replacing $t$ with $\frac{-x}{y}$ gives
$$x^p+y^p=(x+y)(x+\omega y)\dots (x+\omega^{p-1}y)$$
as required. This reformulation allows us to use the full artillery of ring theory on Fermat's Last Theorem.
Gabriel Lamé believed he had a proof, which relied on the assumption that the ring of polynomials $\mathbb{Z}[x]$ is a unique factorisation domain (UFD): (FINISH)

\section{Rings, ideals and Dedekind domains}\label{sec-rings}
Kummer's rectification relies on (or rather, developed) the rich theory of algebraic number theory, including ideals, number rings and Dedekind domains. We will now give an exposition of these results.
\subsection{Ideals}
All rings will be assumed to be commutative with multiplicative identity. Recall the definition of an \defn{ideal} $I$:
\begin{definition}
An \defn{ideal} $I$ of $R$ is a non-empty subset satisfying the following properties:
\begin{enumerate}[(i)]
    \item $I$ is closed under subtraction.
    \item When $i\in I$, for any $r\in R$, $ri\in I$.
\end{enumerate}
\end{definition}

\begin{remark}
Condition (i) can be replaced by $(I,+)$ being a subgroup of $(R,+)$, as $x+y=x-(-1)\cdot y$ and $-y=0-y$.
\end{remark}

\begin{remark}
Since our rings are commutative, we do not need to distinguish between left ideals and right ideals. Additionally, $(I,+)$ is automatically abelian, since $(R,+)$ is.
\end{remark}

\begin{example}
For any $s\in R$ we can define the \defn{ideal generated by $s$} as the subset $\langle s\rangle =\{rs:r\in R\}$. $\langle s\rangle$ is a subset of every ideal containing $s$, and is hence the smallest ideal containing $s$. Ideals of this form are called \defn{principal ideals}.
\end{example}
\begin{example}
When $R=\mathbb{Z}$, the ideals are exactly $n\mathbb{Z}$ for $n\in \mathbb{Z}$. These subsets clearly satisfy the requirements of an ideal, and conversely, any ideal $I$ of $\mathbb{Z}$ has a least non-negative element $n$, and this element then generates the ideal, as if it doesn't it can be shown to not be the least element. We therefore see that every ideal of $\mathbb{Z}$ is a principal ideal. Rings of this form are called \defn{principal ideal domains} or \defn{PID}s for short.
\end{example}
\begin{example}\label{F[x]-PID}
For any field $F$, $F[x]$ is a principal ideal domain. (...) This follows from the fact that the division and remainder algorithm works in any polynomial in one variable over a field \cite{Honours-Algebra}. Let $I$ be an ideal of $F[x]$ and choose a nonzero polynomial of least degree $p(x)\in I$. I claim $\langle p(x)\rangle =I$. Let $q(x)\in I$. By long division, $q(x)=p(x)h(x)+r(x)$ for some $h(x)$ and $r(x)$ where $deg(r)<deg(p)$ or $r=0$. It follows that $r(x)\in I$, but $p(x)$ was of least degree, so $r(x)=0$. Therefore $q(x)\in \langle p(x)\rangle$. Since $q(x)$ was arbitrary, $I=\langle p(x)\rangle$.
\end{example}


Ideals tell you how close a ring is to being a field, as the following lemma shows.
\begin{lemma}\label{field-iff-no-ideals}
A ring $R$ is a field iff it contains no trivial ideals.
\end{lemma}
\begin{proof}
$(\implies)$ if $I$ is an ideal of $R$, then for any non-zero $x\in I$, and any $r\in R$, $r=(rx^{-1})x\in I$, so $I=R$.

$(\impliedby)$ If $R$ contains no non-trivial ideals, then for any $r\in R$, $\langle r\rangle =R$, so $\exists s\in R \text{ such that } rs=1$.
\end{proof}
Recall the following definition.
\begin{definition}
For an ideal $I$ of $R$ the \defn{quotient ring of $R$ by $I$} is the factor group $R/I$ with extra multiplicative structure given by $(x+I)(y+I)=xy+I$.
\end{definition}

\begin{lemma}\label{ideal-bijection}
For any ideal $I$ of $R$ there is a bijection between the set of ideals of $R$ containing $I$ and the set of ideals of $R/I$.
\end{lemma}
\begin{proof}
The bijection is given by $\phi:J\mapsto J/I$ where $J$ is an ideal of $R$ containing $I$ and $$J/I:=\{j+I:j\in J\}\subseteq R/I.$$ It is easily confirmed that $J/I$ is an ideal of $R/I$. In particular, it is closed under subtraction since $J$ is, and closed under multiplication by elements of $R/I$ since $J$ is closed under multiplication by elements of $R$.

$\phi$ is injective: Suppose that $\phi(J)=\phi(S)$ for two ideals $J,S$ containing $I$. Let $j\in J$ and let $s\in S$ be any element such that $j+I=s+I$. Then $j-s\in I\subset S$, so $j\in S$. But $j$ was arbitrary, so $J\subset S$. A similar argument shows $S\subset J$ so $J=S$.

$\phi$ is surjective: Let $X$ be an ideal of $R/I$. Then $X$ is a set of cosets and we can define $$J:=\bigcup X.$$ I claim $J$ is an ideal of $R$. We check:
$$x\in J\implies x+I,y+I\in X\implies (x-y)+I\in X\implies (x-y)\in I\subset J.$$
$$x\in J\implies x+I\in X\implies (x+I)(r+I)\in X\implies rx+I\in X\implies rx\in J.$$

Note $\phi(J)=X$, so $\phi$ is surjective. Therefore $\phi$ is bijective.\end{proof}

\begin{definition}
An ideal $I\neq R$ of $R$ is called a \defn{maximal ideal} if the only ideal it is contained in is $R$ and itself.
\end{definition}

\begin{example}
The maximal ideals of $\mathbb{Z}$ are $p\mathbb{Z}$ where $p$ is prime.
\end{example}
As an immediate consequence of \cref{field-iff-no-ideals} and \cref{ideal-bijection}, we get the following result.
\begin{lemma}\label{ring-over-max-field}
An ideal $I$ of $R$ is maximal if and only if $I/R$ is a field.
\end{lemma}
\begin{proof}
$I$ is maximal $\iff$ the only ideals containing $I$ are $I$ and $R$ $\iff$ there are no nontrivial ideals in $R/I$ $\iff$ $R/I$ is a field.

%$(\impliedby)$ Suppose $R/I$ is a field, and $J$  is an ideal with $I\subset J$. Then $J/I:=\{jI: j\in J\}$ is an ideal of $R/I$. By the previous lemma, $J/I=R/I$, so $J=R$.

%$(\implies)$ Conversely, if $I$ is maximal, then $R/I$ does not contain any non-trivial ideals: a non-trivial ideal $JI\subset R/I$ gives rise to a non-trivial ideal $\langle J\cup I\rangle$ of $R$. By the previous lemma, $R/I$ is a field.
\end{proof}

\begin{remark}
Zorn's lemma, which is equivalent to the axiom of choice (citation needed), implies that every ring has a maximal ideal. The statement of Zorn's lemma is as follows: (finish this thought...)
\end{remark}

\begin{definition}
A \defn{prime ideal} $I\neq R$ of $R$ is an ideal such that $xy\in I\implies x\in I$ or $y\in I$. 
\end{definition}

\begin{example}
The prime ideals of $\mathbb{Z}$ are exactly $p\mathbb{Z}$ where $p$ is prime, motivating the name. If $xy\in p\mathbb{Z}$ then either $x$ or $y$ is divisible by $p$, so either $x$ or $y$ is in $p\mathbb{Z}$. Conversely, in $nm\mathbb{Z}$ where $n,m>1$, $nm\in nm\mathbb{Z}$ but $n,m\not \in \mathbb{Z}$. Thus in $\mathbb{Z}$ prime ideals and maximal ideals are the same.
\end{example}
Factoring out a prime ideal always gives an integral domain.
\begin{lemma}\label{ring-over-prime-int-domain}
An ideal $I$ of $R$ is prime $\iff$ $R/I$ is an integral domain.
\end{lemma}
\begin{proof}
The two statements are equivalent. Both are saying
$$(x+I)(y+I)=I\iff xy\in I\implies x\in I \text{ or } y\in I.$$
\end{proof}
\begin{remark}\label{max-is-prime}
\cref{ring-over-max-field} and \cref{ring-over-prime-int-domain} together show that every maximal ideal is prime, as every field is an integral domain.
\end{remark}

\begin{definition}
An invertible element $r\in R$ is called a \defn{unit}. An element $r\in R$ is called \defn{irreducible} if it is not 0, not a unit (invertible in $R$) and if $r=xy\implies x$ or $y$ is a unit. $r\in R$ is called \defn{reducible} if it is not 0, a unit, or irreducible.
\end{definition}
Irreducible and reducible elements generalise the notion of respectively prime and composite numbers in $\mathbb{Z}$. The units of $\mathbb{Z}$ are $-1$ and $1$, and certainly $p$ is (a negative of a) prime if and only if $p=xy\implies x$ or $y=\pm 1$. In a field $F$ every nonzero element is a unit, so fields do not have (ir)reducibles. The next proposition motivates defining principal ideal domains and irreducible elements.

\begin{prop}\label{PID-over-irreducible-is-field}
If $R$ is a PID, then $0\neq r\in R$ is irreducible if and only if $R/\langle r\rangle$ is a field.
\end{prop}
\begin{proof}
By \cref{ring-over-max-field}, we can instead show $r$ is irreducible $\iff$ $\langle r\rangle$ is maximal. We may assume $r$ is not a unit, as $r$ is a unit if and only if $\langle r\rangle=R$, and $R/R=\{0\}$ is not a field - we require two elements $0\neq 1$. Let $r$ be irreducible. Since $R$ is a PID, it is enough to show show $\langle r\rangle$ contains every proper ideal containing $r$. Let $r\in \langle x\rangle$ for some non-unit $x\in R$. Then $r=xy$ for some $y$. Since $r$ is irreducible, $y$ is a unit. Therefore $x=ry^{-1}$, so $x\in \langle r\rangle$. It follows that $\langle x\rangle \subseteq \langle r\rangle$.

Now let $\langle r \rangle$ be maximal. Let $r=xy$ and suppose neither $x$ nor $y$ are units. Since $x$ is not a unit, $\langle x\rangle \neq R$. Additionally, since $y$ is not a unit, $x\not \in \langle r\rangle$, however $r\in \langle x\rangle$, so $\langle r\rangle \subset \langle x\rangle$ where this inclusion is proper. This contradicts $\langle r\rangle$ being maximal.
\end{proof}

\begin{example}
In $\mathbb{Q}[x]$, $x^2+1$ is irreducible. It follows that $\frac{\mathbb{Q}[x]}{\langle x^2+1\rangle}$ is a field. 
\end{example}

\begin{remark}
It is essential that $R$ is a PID. For example, $2$ is irreducible in $\mathbb{Z}[x]$ but $\frac{\mathbb{Z}[x]}{\langle 2\rangle}$ is not a field, as $[x]$ is not a unit. This reflects the fact that $\mathbb{Z}[x]$ is not a PID, take for example the non-principal ideal $\langle 1,x^2\rangle$.
\end{remark}

Principal ideal domains are UFDs...

If only number rings were principal ideal domains!! But they are not ...

\subsection{Noetherian rings}

\begin{definition}
A ring $R$ is said to be \defn{Noetherian} if every increasing sequence of ideals $$I_1\subseteq I_2\subseteq I_3 \subseteq \dots$$
terminates.
\end{definition}

\begin{proposition}\label{noetherian-module-equivalences}
The following are equivalent for a ring $R$.
\begin{enumerate}[(i)]
    \item $R$ is Noetherian.
    \item Every non-empty collection of ideals of $R$ has a maximal element.
    \item Every ideal $I$ of $R$ is finitely generated.
\end{enumerate}
\end{proposition}
\begin{proof}
$(i)\implies (ii)$ easily: a counter-example to $(ii)$ is a collection of ideals containing a non-terminating increasing sequence of ideals. Additionally, $(ii)\implies (i)$ because any increasing sequence of ideals
$$I_1\subseteq I_2\subseteq I_3 \subseteq \dots$$
gives a collection $\{I_i\}_{i\geq 1}$ of ideals, which then has a maximal element.

$(i)\implies (iii)$ a non-finitely generated ideal $I$ gives rise to a non-terminating sequence of ideals. Let $a_1\in I$ and $I_1=\langle a_1\rangle$. Then let $a_2\in I\setminus I_1$ and $I_2=\langle a_1,a_2\rangle$. Continue this process for all natural numbers to construct the increasing sequence of ideals. It will never terminate as $I$ is not finitely generated.

$(iii)\implies (i)$ Let $$I_1\subseteq I_2\subseteq \dots$$ be an increasing chain of ideals. Then $\bigcup I_i$ is an ideal of $R$, and therefore finitely generated, say by $\{a_1,\dots a_n\}$. Letting $m$ be the smallest integer such that $\{a_1,\dots a_n\}\subset I_m$, $I_m$ is a maximal element of the chain.

\end{proof}

\begin{example}
Any principal ideal domain is Noetherian, as all its ideals are generated by a single element.
\end{example}
The following proposition will allow us to construct Noetherian rings from other Noetherian rings.
\begin{proposition}[Hilbert's basis theorem]
If $R$ is Noetherian, so is $R[x]$.
\end{proposition}
\begin{proof}
(finish proof)
\end{proof}

\begin{example}
$\mathbb{Z}[x]$ is Noetherian as $\mathbb{Z}$ is. Any field $F$ is clearly Noetherian, therefore $F[x]$ is also Noetherian.
\end{example}

\begin{corollary}
If $R$ is Noetherian, so is $R[x_1,x_2,x_3,\dots,x_n]$
\end{corollary}
\begin{proof}
This follows by induction from Hilbert's basis theorem, by noting $$R[x_1,x_2,x_3,\dots,x_n]=R[x_1,\dots,x_{n-1}][x_n].$$
\end{proof}

In $\mathbb{Z}$, we can write any element as a product of primes. In general rings, we want to think about irreducibles instead of primes. It is therefore natural to ask: in which rings can elements be written as a product of irreducible elements? As it turns out, all Noetherian rings have this property.

\begin{lemma}\label{noetherian-irreducible-elements}
Any (non-zero, non-unit) element $r\in R$ of a Noetherian ring $R$ can be written as a finite product of irreducible elements
$$r=r_1r_2r_3\dots r_n.$$
\end{lemma}
\begin{proof}

\end{proof}

\begin{remark}
The prime decompositions in $\mathbb{Z}$ are unique, but this is not necessarily the case in Noetherian rings! For example, $\dots$. Noetherian rings were the decomposition into irreducibles is unique (up to multiplication by a unit (CHECK)) are called \defn{unique factorisation domains} or \defn{UFDs}. The main failure of Lamé's proof is that number rings are not UFDs. For example $\dots$
\end{remark}

\begin{proposition}[Number rings are Noetherian]

\end{proposition}

In the previous section, we showed that all PIDs are UFDs. Unfortunately, we also saw that PID is too strict a requirement for all number rings. Number rings are Noetherian, however, as we have just shown. Our hope is that there is that there is an easier requirement to satisfy being a UFD, sitting between Notherian rings and PIDs. Such a requirement exists - but first we need a definition.

\begin{definition}
A non-zero non-unit $r\in R$ is \defn{prime} if $p|ab\implies p|a$ or $p|b$.
\end{definition}

We argued that the correct notion of "primeness" was \textbf{irreducibility.} In integral domains, this notion we have just defined is stronger - every prime element is irreducible: Let $r$ be prime and $r=xy$. Then $r|xy$, so without loss of generality, $r|x\implies x=cr$ for some $c\in R$. Then $$r=rcy\implies r(1-cy)=0\implies cy=1 \text{ (integral domain) }\implies y \text{ is a unit.}$$ Additionally, as is easily shown, $\langle p \rangle$ is a prime ideal. As we now show, prime elements give rise to a easier UFD condition:

\begin{proposition}\label{Noetherian-prime-UFD}
Let $R$ be a Noetherian ring and integral domain such that every irreducible element is prime. Then $R$ is a UFD.
\end{proposition}
\begin{proof}
By \cref{noetherian-irreducible-elements}, any element has a decomposition into irreducibles. We only need to take care of uniqueness. Suppose $r=r_1r_2r_3\dots r_n=s_1s_2\dots s_m$ are two decompositions into irreducibles. Since $r_1$ is prime, $r_1|s_i$ for some $i$. Since $s_i$ is irreducible, $s_i=ur_1$ for some unit $u$. Since $R$ is an integral domain, $$r_2r_3\dots r_n=us_2s_3\dots s_m,$$ where we have relabeled the $s_i$'s. We can define $r_2'=r_2u^{-1}$, which is also irreducible, then continue as before. If $m=n$ we have shown that every $r_i=u_is_j$ for some $j$ and unit $u_i$, as required. Otherwise, WLOG, suppose $m>n$. Then at the end, after relabeling the $s_i$'s, we have
$$\bar u= s_1\dots s_k.$$ Then $s_i|\bar{u}\implies \bar{u}=zs_i$ for some $z$. Then $(\bar{u}^{-1}z)s_i=1,$ so $s_i$ is a unit, a contradiction.
\end{proof}

Armed with \cref{Noetherian-prime-UFD}, we can check which number rings have this condition. These number rings will satisfy Lamé's proof.

\begin{proposition}\label{Z_p[x]-is-UFD}
$\mathbb{Z}_p[x]$ is a UFD. 
\end{proposition}
\begin{proof}
MISSING.
\end{proof}

%\begin{definition}
%A ring $R$ is said to be Noetherian if it is Noetherian as an $R-module$.
%\end{definition}

%Note that the submodules of $R$ are exactly the ideals of $R$. Therefore \cref{noetherian-module-equivalences} can be rewritten using ideals for the case that $M$ is the ring $R$ itself. 
\subsection{Dedekind domains}
\section{Number rings}
A \defn{number field} is a subfield of $\C$ that is a finite dimensional vector space over $\Q$. Examples include $\Q[\omega]$, the \defn{$n$th cyclotomic field} where $\omega=e^{2\pi i/n}$ for some integer $n$. More generally, let $a\in\C$ be \defn{algebraic over $\Q$}, i.e. satisfy a monic irreducible polynomial $f$ with coefficients in $\Q$. Then I claim $\Q[a]$ is a number field. 
\begin{lemma}
For $a\in \mathbb{C}$ algebraic over $\Q,$ $\Q[a]$ is a number field.
\end{lemma}
\begin{proof}
We may choose $f$ to be irreducible, as if $f=gh$ for non-units (that is, non-constants) $f,g$, then  then WLOG both $g$ and $h$ are monic and either $g(a)=0$ or $h(a)=0$. Then $\Q[a]=\Q[x]/I$, where $I$ is the ideal of all polynomials annihilating $a$. In fact, $I$ is principal and generated by $f$: If $g(a)=0$ then by the Euclidean algorithm, $g(x)=f(x)h(x)+r(x)$ where $deg(r(x))<deg(f(x))$. Since $r(a)=0$, we must have that $r=0$, since $f$ is of minimal degree\footnote{By the Euclidean algorithm, $f(x)=j(x)r(x)+s(x)$. Either $s=0$ and $f$ is not irreducible, or $s(a)=0$ and $deg(s)<deg(r)<deg(f)$. Repeated applications of the Euclidean algorithm will eventually reach a contradiction - less than $deg(f)$ applications are necessary.}. Now $\Q[x]$ is a PID by \cref{F[x]-PID}, so by \cref{PID-over-irreducible-is-field}, $\Q[a]=\Q[x]/\langle f\rangle$ is a field. If $f$ has degree $n$, $\Q[a]=\{q_0+q_1a+\dots+q_{n-1}a^{n-1}:q_i\in \Q\}$, as any polynomial $g(x)$ of degree $n$ or higher can reduced by the Euclidean algorithm to $g(x)=h(x)f(x)+r(x)$, $deg(r)<n$, such that $g(a)=r(a)$. It is therefore also finitely generated as a $\Q-$vector space.
\end{proof}
In fact, every number field is of this form for some algebraic number $a\in\C$ \cite{NumberFields} (in fact, proven later). Inside a number field sit the \defn{algebraic integers}: the elements algebraic over $\Z$. Letting $\A$ be the algebraic integers in $\C$, we define a \defn{number ring} as the ring $\A\cap \Q[a]$ of algebraic integers in a number field. For Fermat, we will work with the algebraic integers in the $p-$th cyclotomic field $\A\cap \Q[\omega],$ where $p$ is prime prime. Of course, we would like to prove that number rings are indeed rings. \cite{NumberFields} shows this follows from the classification that $a$ is an algebraic integer iff $(\Z[a],+)$ is finitely generated. Instead, we give a direct proof, which exposes the inner workings of the number ring.
\begin{proposition}
Number rings are rings.
\end{proposition}
\begin{proof}
By the subring test, it is enough to show number rings are closed under multiplication and subtraction, and contain $1$. $1$ is algebraic since $x-1$ is irreducible in $\Q$. Let $a$ and $b$ be algebraic integers with monic, minimal, irreducible polynomials $f$ and $g$, respectively. (...)
\end{proof}
(...)
\subsection{The cyclotomic fields $\Q[\omega]$}
A remarkable and non-obvious result will simplify our lives drastically.
\begin{theorem}\label{algebriac-integers-are-Z[w]}
For the $p-$th cyclotomic fields $\Q[\omega]$,
$$\A\cap\Q[\omega]= \Z[\omega].$$
\end{theorem}
This immediately shows us that $\A\cap \Q[\omega]$ is a PID, and in the following section will show that it is something called a Dedekind domain, a central part of our proof. This is a difficult and highly non-obvious result, and we will spend all section on proving it. Our approach follows that given in \cite{NumberFields} chapter 2. First, a useful lemma.

\begin{lemma}\label{lemma-monics-are-Z}
If $f\in \Z[x]$ is monic and $f=gh$ where $g,h\in \Q[x]$ are monics, then in fact $g,h\in \Z[x]$.
\end{lemma}
\begin{proof}
Omitted, see \cite{NumberFields}.
\end{proof}

Now we want to study the $m^{th}$ cyclotomic number fields $\Q[\omega]$. If two numbers $\alpha,\beta\in \mathbb{C}$ satisfy the same monic polynomial over a subfield $K$, we say they are \defn{conjugate}. The following Theorem and its Corollary are proven in \cite{NumberFields}. We give adapted proofs filling in a few missing details.
\begin{theorem}\label{conjugates-of-Q[w]}
In the $m^{th}$ cyclotomic number fields $\Q[\omega]$, $\omega^k, 1\leq k\leq m, (k,m)=1$ are exactly the conjugates to $\omega$. 
\end{theorem}
\begin{proof}
Let $f$ be the monic irreducible polynomial of $\omega$. (NOTE: have I proven that $\omega$ is algebraic over $\Q$? or in fact that the same is true for any complex \#?) Since $\omega$ is a root of $x^m-1$ and $f$ is irreducible (minimal), $f$ divides $x^m-1$:
$$x^m-1=f(x)g(x)$$
for some $g(x)$. By \cref{lemma-monics-are-Z}, both $f$ and $g$ are in $\Z[x]$.

For our proof, it is enough to show that for every $k$ with $(k,m)=1$ and every prime $p$ not dividing $m$, $\omega^{kp}$ is conjugate to $\omega^k=:\theta$. If we show this, then the general result follows inductively. I.e. if $k$ is s.t. $(k,m)=1$ and the prime decomposition of $k$ is $k=p_1^{a_1}p_2^{a_2}\dots p_n^{a_n}$ then
$$\omega\sim\omega^{p_1}\sim\omega^{p_1^2}\sim\dots\sim\omega^{p_1^{a_1}}\sim\omega^{p_1^{a_1}p_2}\sim\dots \omega^{p_1^{a_1}p_2^{a_2}\dots p_n^{a_n}}.$$

We want to show that $f(\theta^p)=0$. Suppose for sake of contradiction that $g(\theta^p)=0$. We get another monic polynomial $g^*(x):=g(x^p)$ over $\mathbb{Z}$ which $\theta$ is a root of. Therefore $g^*=fh$ for some $h$ which again will be in $\mathbb{Z}$ by \cref{lemma-monics-are-Z}. Now we may reduce all coefficients mod $p$ to get monics $\overline{g^*}=\bar{f}\bar{h}$. By multinomial expansion, $\overline{g}(x^p)=\bar{g}(x)^p$ since $$(\sum_{i=0}^n a_ix^i)^p=\sum_{k_1+k_2+\dots+k_n=p}{p\choose k_1,k_2,\dots ,k_n}\prod_{i=0}^n(a_ix^i)^{k_i}$$
The multinomial coefficient $${p\choose k_1,k_2,\dots ,k_n}=\frac{p!}{k_1!k_2!\dots k_n!}$$
will be a product of $p$ (and hence be $0$ in $\Z_p$) iff none of the $k_i$ equal $p$. Conversely, if $k_i=p$ then $k_j=0$ for $j\neq i$. Therefore in $\Z_p$,
$$\overline{g}(x)^p=(\sum_{i=0}^n a_ix^i)^p=\sum_{i=0}^n(a_ix^i)^p=\sum_{i=0}^n a_i^p (x^p)^i=\sum_{i=0}^n a_i (x^p)^i=\overline{g}(x^p)$$
where the second-to-last equality follows by the fact that the multiplicative group of units $\Z_p^{\times}$ is cyclic of order $p-1$. In this argument we have implicitly used that reducing coefficents $\mod p$ is a ring homomorphism, so $\overline{\sum a_ix^i}=\sum \overline{a_i}x^i.$

We have that $(\overline{g})^p=\overline{f}\overline{h}$. By \cref{Z_p[x]-is-UFD}, $\Z_p$ is a UFD, so both sides of the equation have the same factorisation into irreducible elements. This gives an irreducible $j\in Z_p[x]$ that divides both $\overline{g}$ and $\overline{f}$. Therefore $j^2|\overline{f}\overline{g}=x^m-1$. Writing $x^m-1=j^2k$ for $k\in \Z_p[x]$ and taking derivatives of both sides gives $\overline{m}x^m=2jj'k+j^2k'$ such that $j|\overline{m}x^m$ in $\Z_p$ (remember this trick - we will use it again). Since $(m,p)=1$, $\overline{m}\neq 0$. Since $\overline{m}x^m$ has factorisation $\overline{m}x^m=\overline{m}x\cdot x\cdot \dots \cdot x$, and since $\Z_p$ is a UFD, in fact $j$ must be a monomial itself. However $j|x^m-1$, a contradiction.
\end{proof}

\begin{definition}
Let $K\subset L\subset \mathbb{C}$ be fields. We say $L$ is a \defn{field extension} of $K$. The \defn{degree} of $L$ is the dimension of $L$ as a vector space over $K$, and is denoted $[L:K]$.
\end{definition}



\begin{corollary}\label{Q[w]-degree}
$[\Q[\omega]:\Q]=\varphi(m)$, where $\varphi(m)$ is the number of integers $1\leq k\leq m$ with $(k,m)=1$.
\end{corollary}

\begin{proof} Let $f$ be the monic irreducible polynomial of $\omega$. $\Q[\omega]$ has (?) $f$ as basis, and therefore its degree is the degree of $f$ (PROVE THIS). By algebraic closure of $\mathbb{C}$, $f$ has $deg(f)$ roots, counted with multiplicity. In fact, we would like to show the roots are distinct. Suppose $l$ is linear with $l^2|f$. Then as shown in the proof of \cref{conjugates-of-Q[w]}, $l|f'$. However, since $\Q[x]$ is a PID, $\langle f(x),f'(x) \rangle=\Q[x]$. It follows that $1=f(x)g(x)+f'(x)h(x)$ for some $g,h\in \Q[x]$. This is a contradiction since $l$ divides the RHS but not the LHS. (? bad proof)
\end{proof}

\subsection{Normal field extensions}
The field $K=\mathbb{Q}[\sqrt[3]{2}]$ is entirely contained in the real line, and therefore does not contain all the conjugates of its elements (equivalently, the conjugates of $\sqrt[3]{2}$, as we will show). In particular, the irreducible $f(x)=x^3-2$ has three roots $\sqrt[3]{2},\sqrt[3]{2}\omega,\sqrt[3]{2}\omega^2$ where $\omega=e^{2\pi i/3}$ and of these only $\sqrt[3]{2}$ is real. There is, however, a field extension $L=\mathbb{Q}[\sqrt[3]{2},\sqrt[3]{2}\omega,\sqrt[3]{2}\omega^2]\subset \C$ of $K$ which contains all the conjugates of $\sqrt[3]{2}$. In fact, it contains the conjugates of all of $K$. Furthermore, this turns out to be possible for any number ring. The goal of this section is to prove this result. In doing so we will also classify all finite field extensions of number fields. Our treatment roughly follows Appendix 2 of \cite{NumberFields}.

For the rest of the section, let $L$ be a number field of finite degree $[L:K]$ over $K.$

\begin{prop}\label{L-algebraic-over-K}
Every $\alpha \in L$ is algebraic over $K$, i.e., satisfies an irreducible monic polynomial $f\in K[x]$.
\end{prop}
\begin{proof}
Let $I=\{f\in K[x]:f(\alpha)=0\}$. Then clearly $I$ is an ideal ($f(a)=0\implies g(a)f(a)=0$ for any $g(x)$. Similarly if $f(a)=g(a)=0, f(a)-g(a)=0$.) Since $K[x]$ is a PID (\cref{F[x]-PID}), $I=\langle f(x)\rangle$ for some $f$ which we may assume to be monic by dividing by the leading coefficient. Necessarily $f$ is irreducible: Let $f=gh$ and suppose WLOG $g(\alpha)=0$. $f$ has minimal degree in $I$ so $deg(g)=deg(f)$ and $deg(h)=0$ i.e. $h$ is a unit. 
\end{proof}

\begin{warning}
\cref{L-algebraic-over-K} shows in particular that every element $\alpha\in \C[\omega]$ of the $p-th$ cyclotomic field is algebraic over $\Q$. The claim of \cref{algebriac-integers-are-Z[w]} is that not every element is algebraic \textbf{over} $\mathbf{\Z}$ - only the elements of the form $\Z[\omega]$.
\end{warning}

 The following result shows that instead of working with basis elements in $K$, we can work with embeddings (injective field homomorphisms) $L\rightarrow \C$ that restrict to the canonical inclusion on $K$.

\begin{proposition}
Every embedding $K\rightarrow \C$ extends to exactly $[L:K]$ embeddings of $L$ in $\C$.
\end{proposition}
\begin{proof}
Let $\{a_1,a_2,\dots,a_n\}$ be a basis for $L$ over $K$. Note $$L=span(a_1,\dots,a_n)\subset K[a_1,a_2,\dots,a_n]\subset L.$$ Therefore $L= K[a_1,a_2,\dots,a_n]=K[a_1,\dots,a_{n-1}][a_n]$, so we may work inductively on the basis elements. The result is clear if $L=K$, so suppose otherwise . Let $L=K[a]$ for $a\not\in K$, which need not be of degree 1 over $K$. Let $\sigma:K\rightarrow \C$ be an embedding
and $f$ be the irreducible monic of $a$ over $K$. Note by the proof of \cref{Q[w]-degree}, $f$ has $n=deg(f)$ distinct roots in $\C$, which are the conjugates of $a.$ $\sigma(a)$ is necessarily one of these conjugates, since $0=\sigma(f(a))=f(\sigma(a)).$ Additionally, each choice of conjugate $b$ defines an embedding of $L$ restricting to $\sigma$ on $K$ in the following way:
$$\sigma f(a)=\sigma \big(\sum_{i=0}^n c_ia^i\big)=\sum_{i=0}^n \sigma(c_i) b^i.$$
Since there was no choice involved other than the choice $a\mapsto b$, every embedding of $L$ restricting to $\sigma$ on $K$ is of this form. There are therefore $n=deg(f)$ such embeddings. Since $L=K[x]/f$ (is this proven?), $[L:K]=deg(f)$, with basis elements $\{1,a,a^2,\dots, a^{n-1}\}$ (is this proven? Both can probably be done in generality).

We have shown every embedding of $K$ extends to $[K[a_1]:K]$ embeddings of $K[a_1]$, which extend to $[K[a_1,a_2]:K[a_1]][K[a_1]:K]=[K[a_1,a_2]:K]$ embeddings of $K[a_1,a_2]$ (see below). Therefore, proceeding by induction gives the desired result.
\end{proof}

The equality $[K[a_1,a_2]:K[a_1]][K[a_1]:K]=[K[a_1,a_2]:K]$ follows by considering basis elements. If $A\subset B\subset C$ are all number fields with finite $[B:A]$ and $[C:B]$, then a basis $\{b_1,b_2,\dots,b_n\}$ of $B$ over $A$ and a basis $\{c_1,\dots c_m\}$ of $C$ over $B$ gives a basis $\{b_ic_j:1\leq i \leq n, 1 \leq j\leq m\}$ of size $nm$. It is easy to show this spans $C$, and 

$$\sum_{i\leq n,j\leq m}a_{ij}b_ic_j=0\implies \sum_{j\leq m} \big(c_j \sum_{i\leq n}a_{ij}b_i\big)=0$$
$$\implies \forall j, \sum_{i\leq n} a_{ij}b_i=0 \text{ (linear independence of the } c_j \text{'s)}$$
$$\implies \forall (i,j), a_{ij}=0 \text{ (linear independence of the } b_i \text{'s)},$$
giving linear independence.

Our desired corollary follows directly by considering the canonical inclusion $K\rightarrow \C$ as an embedding.
\begin{corollary}
There are $[L:K]$ embeddings of $L$ in $\C$ restricting to the canonical inclusion on $K$.
\end{corollary}

We now justify studying the field extensions of the form $K[a]$ by proving every finite degree number field extension is of this form.

\begin{theorem}\label{classifying-finite-field-extensions}
For every number field extension, $L=K[\alpha]$ for some $\alpha \in L$.
\end{theorem}
\begin{proof}
We proceed by induction on $n=[L:K]$. The proof is clear in the case $L=K=K[1]$. Now suppose $K[a_1,\dots,a_{n}]=K[\alpha]$ for some $\alpha \in L$ and let $L=K[\alpha,\beta]$ for $\beta\not\in K[\alpha]$. Let $0\neq a\in K$ and consider $\alpha+a\beta$. Suppose for sake of contradiction that $K[\alpha+a\beta]\neq L$. Then $[K[\alpha+a\beta]:K]<n$ where we note this degree is the number of conjugates of $\alpha+a\beta$ over $K$. There are $n$ embeddings of $L$ restricting to the identity on $K$, and since each one takes $\alpha+a\beta$ to one of its conjugates, of which there are fewer than $n$, $\sigma(\alpha+a\beta)=\sigma'(\alpha+a\beta)$ for two such embeddings $\sigma,\sigma'$. Since these are field homomorphisms fixing $K$, 
$$\sigma(\alpha)+a\sigma(\beta)=\sigma'(\alpha)+a\sigma'(\beta)$$
so
$$a=\frac{\sigma(\alpha)-\sigma'(\alpha)}{\sigma(\beta)-\sigma'(\beta)}$$
Note the denominator is always nonzero since if $\sigma(\beta)=\sigma'(\beta)$ then $\sigma(\alpha)=\sigma'(\alpha)$ as well. Since they are field homomorphisms, they must therefore equal on all of $L$, however they were assumed to be distinct.

Since there are only finitely many choices for $\sigma(\alpha),\sigma(\alpha'),\sigma(\beta)$ and $\sigma(\beta')$,
$$K[\alpha+a\beta]\neq L$$
for only finitely many $a\in K$. Choosing any other $a$ will give the desired result.

\end{proof}
\begin{definition}
A field extension $L$ of $K$ is said to be \defn{normal} if each of the $[L:K]$ embeddings of $L$ restricting to the inclusion on $K$ are automorphisms.
\end{definition}

Note this equivalent to $L$ containing all the conjugates of $K$, since each embedding of $L$ takes $a \in K$ to a conjugate $b$, so if the embedding is an automorphism, $b\in L$. Conversely, setting $L=K[a]$, since each embedding is given entirely by the choice $a\mapsto b$ for some conjugate $b$ of $a$, if each such $b$ is in $L$, then $\sigma(\sum a_i a^i)=\sum a_i b^i\in L$.

\begin{theorem}
Every finite field extension $K\subset L$ can be extended to a finite normal field extension $K\subset L \subset M$ (normal over both $K$ and $L$).
\end{theorem}
\begin{proof}
By \cref{classifying-finite-field-extensions} we may write $L=K[\alpha]$ for some $\alpha \in L$. 
\end{proof}

\subsection{$\A\cap\Q[\omega]= \Z[\omega]$}
In this section we finish the proof that in the $p$th cyclotomic field, $\A\cap\Q[\omega]= \Z[\omega]$. We proceed as in \cite{NumberFields}, using the $[K:\Q]$ embeddings of a number field $K$ in $\mathbb{Q}$, guaranteed by the previous section, to define the \defn{trace}, \defn{norm} and \defn{discriminant} of $K$. 

\begin{definition}
Let $n=[K:\Q]$ and let $\sigma_1,\sigma_2,\dots,\sigma_n$ be the $n$ embeddings of $K$ in $\Q$. Let $\alpha\in K$. The \defn{trace} of $K$  is the function $$\alpha\mapsto\sigma_1(\alpha)+\sigma_2(\alpha)+\dots+\sigma_n(\alpha).$$
The \defn{norm} $N$ of $K$ is the function
$$\alpha\mapsto\sigma_1(\alpha)\sigma_2(\alpha)\dots \sigma_n(\alpha).$$
\end{definition}

The first goal is to show that the codomain of $T$ and $K$ are in fact $\Q$. Furthermore, we wish to show that  $T(\alpha)\in \Z\iff N(\alpha)\in \Z\iff \alpha$ is algebraic. 

Firstly note since the $\sigma_i$ are field homomorphisms that fix $\Q$,

\begin{itemize}
    \item $T(r\alpha+\beta)=rT(\alpha)+T(\beta)$ for $\alpha,\beta\in K, r\in \Q$.
    \item $N(r\alpha\beta)=r^nN(\alpha)N(\beta)$.
\end{itemize}
In particular, $T$ is linear in $\Q$. In the special case where $K=\Q[\alpha],$ $T(\alpha)=$

\begin{definition}
For any $n-$tuple of elements $(\alpha_1,\alpha_2,\dots \alpha_n)$ of $K$, the \defn{discriminant} is 
$$disc(\alpha_1,\alpha_2,\dots,\alpha_n)=|[\sigma_i(\alpha_j)]|^2$$
i.e. the square of the discrimant of the matrix with $(i,j)$th element $\sigma_i(\alpha_j)$.
\end{definition}


\subsection{Number rings are Dedekind domains}



\section{Regular primes}

