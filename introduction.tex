\chapter{Introduction}
In theorems that are easy to state and difficult to prove, Fermat's Last Theorem likely takes the crown. The Theorem, first stated in 1637, was never proven by Fermat, although he claimed to have a "a truly marvelous proof" (CITE), which the margins of his copy of Arithmetica were "too narrow to contain." It is very doubtful that he had such a proof - indeed, many mathematicians since then have thought they had a proof, only to find a slight error making the entire proof crumble. A correct proof was not published until 1995, by Sir Andrew Wiles, and this proof was an epic achievement bringing together our most advanced understanding of elliptic curves (CITE). Despite the difficulty of the proof, Fermat's Last Theorem is so easy to state, a 9-year-old can understand it.

\begin{theorem}[Fermat]
The equation
\begin{equation}\label{Fermat}
    x^n+y^n=z^n
\end{equation}
has no integer solutions for $n>2$.
\end{theorem}

The case $n=2$ is Pythagoras' Theorem, which has integer solutions. For example, $(x,y,z)=(3,4,5)$ works. It is enough to prove Fermat's Last Theorem for the case that $n$ is an odd prime. This is because a counter-example for $n=pm$ a composite number also gives a counter-example for a prime:
$$(x^m)^p+(y^m)^p=(z^m)^p.$$
(Technically, we should also prove the case $n=4$, as we know Fermat does not work for $n=2$ - however this turns out not to be hard). We can additionally assume that in a solution $(x,y,z)$, $x$, $y$ and $z$ are all coprime. Indeed, if they all three share a factor, we can divide by that factor to get another counter-example. If instead, two of three share a factor $n$, then $\cref{Fermat}$ will not hold $(\Mod n)$. We therefore reduce the proof to two types of solutions:
\begin{itemize}
    \item (type 1): $p$ does not divide either of $x,y,z$.
    \item (type 2): $p$ divides exactly one of $x,y,z$.
\end{itemize}

The case $n=3$ is simple enough to prove using just undergraduate level number theory. First, suppose $(x,y,z)$ is a type 1 solution. Then $x,y,z$ are all $\pm1 \Mod 3$, so $$x^3+y^3\neq z^3 (\Mod 3).$$
(FINISH PROOF)

(PROVE n=4 - exercise 15 on pg 8 in Number Fields).

These cases are truly the only trivial ones - proving Fermat's Theorem for $p>3$ requires more work. Type 1 solutions turn out to be the most interesting to analyse, so from now on we will assume $(x,y,z)$ is a Type 1 solution. Our first trick will be to turn $\cref{Fermat}$ into a multiplicative equation. Letting $\omega=e^{2\pi i/p}$, we factor \cref{Fermat} to
$$(x+y)(x+\omega y)\dots(x+\omega^{p-1} y)=z^p.$$
This follows from noting $1,\omega,\dots,\omega^{p-1}$ are unique (NOTE: This is what makes the $p$ case work more easily than the $n$ case!!!) solutions to $t^p-1=0$, and since $\mathbb{C}$ is algebraically closed, this implies
\begin{equation}\label{eq-omega-polynomial}t^p-1=(t-1)(t-\omega)\dots (t-\omega^{p-1}).\end{equation}
Replacing $t$ with $\frac{-x}{y}$ gives
$$x^p+y^p=(x+y)(x+\omega y)\dots (x+\omega^{p-1}y)$$
as required. This reformulation allows us to use the full artillery of ring theory on Fermat's Last Theorem.
Gabriel Lamé believed he had a proof, which relied on the assumption that the ring of polynomials $\mathbb{Z}[x]$ is a unique factorisation domain (UFD): (FINISH)

Something on UFDs...

\section{Lamé's false proof for Type I solutions}
We follow the account in \cite{Wright}. Assume $p>3$.
Part I: (MISSING EQUATION). 

Factorise $z=r_1r_2\dots r_n$ such that $z^p=r_1^p r_2^p\dots r_n^p$. Similarly write $(x+\omega^jy)=r_{1j}r_{2j}\dots r_{n_j j}$. Under the assumption that $\Z[\omega]$ is a UFD, $r_{ij}=ur_i$ for some $i$. We will additionally show that $x+\omega y$ does not share any irreducibles with $x+\omega^j y$ $j>1$. If $r$ divides both, then $$r|(x+\omega y) - (x+\omega^j y)=\omega y(1-\omega^{j-1})}$$
Since $\omega$ is a unit with inverse $\omega^{p-1}$, this implies $r|y(1-\omega^{j-1})\implies r|y\prod_{j}(1-\omega^{j})=py$ by (MISSING). Since $p$,$y$ and $z$ are pairwise coprime, $py$ and $z$ are also coprime. By the Euclidean algorithm, there exist integers $m,n$ such that $$mpy+nz=1.$$ Then $r$ divides the LHS but not the RHS, a contradiction.

Since $(x-\omega y)$ does not share any irreducibles with the $(x-\omega^j y)$'s, its unique factorisation must in fact be of the form $$(x-\omega y)=ur_{\alpha(1)}^p\dots r_{alpha(i)}=:u\alpha^p.$$ for some unit $u$ and $\alpha\in \Z[\omega]$.

Part II: Under the assumption that $(x-\omega y)=u\alpha^p$, we finish the proof of Fermat. Note this part does not need the assumption that $\Z[\omega]$ is a UFD. 

\begin{lemma}
For any unit $u\in \Z[\omega]$, $$\frac{u}{\overline{u}}=\omega^k$$ for some $k\in \Z$. 
\end{lemma}
\begin{proof}

\end{proof}

Writing $\alpha=a_0+a_1\omega+\dots+a_{p-1}\omega^{p-1}, a_i\in \Z$ we find by multinomial expansion that $\alpha^p=a_0^p+\dots+a_{p-1}^p \mod p\Z[\omega]$, since every other term in the expansion is a multiple of $p$. Note by $\mod p\Z[\omega]$ we mean they are mapped to the same element under the projection $\Z[\omega]\rightarrow \Z[\omega]/p\Z[\omega]$. It follows that $$(x+\omega y)=ua\mod p\Z[\omega]$$ for some $a\in \Z$. By the lemma, $$(x+\omega y)=\omega^k\overline{(x+\omega y)}=\omega^k(x+\omega^{-1}y \mod p\Z[\omega]$$ for some $k\in \Z$. By equating coefficients as polynomials over $\omega$, we find $k=1$ so
$$x+\omega y=x\omega+y \mod p\Z[\omega] \implies x=y\mod p.$$

Rewriting Fermat's Last Theorem as $x^p+(-z)^p=(-y)^p$ then also gives $x=-z\mod p$. By Fermat's little theorem, the Last Theorem reduces to
$$x+y=z\mod p$$
By what we have shown, this implies
$$3x=0\mod p$$
Since $p\neq 3$, $p|x$, which is a contradiction since $(x,y,z)$ is a Type I solution.


\section{Kummer's partial rectification}
Kummer discovered a flaw in Part I of Lamé's proof. He managed to prove that $\Z[\omega]$ is NOT a UFD in general -- for example he showed that $\Z[e^{2\pi i/23}]$ is not a UFD \cite{Wright}. Every element of $\Z[\omega]$ \textbf{does} factor as a product of irreducibles, as we will later show by proving $\Z[\omega]$ is \textit{Noetherian}. However, this factorisation is not unique in general. This is a subtle point - in Part I we were only able to deduce that $x+\omega y=u\alpha^p$ by arguing that if $r$ is in the decomposition into irreducibles of $x+\omega y$ then it must be in the decomposition of $z^p$ as well - this is only true in a UFD.

Kummer realised that there was a strictly weaker (see MISSING) condition to UFD that would make Lamé's Part I proof work. This condition is best explained in the framework of ideals, an invention of Dedekind \cite{Wright}. Dedekind showed that all the \textit{ideals} of $\Z[\omega]$ factorise uniquely into something called \defn{prime ideals} (see MISSING). (...)


We define an equivalence relation on ideals $I$ of $\Z[\omega]$ by
$$I\sim J \iff \alpha I=\beta J \text{ for some } \alpha,\beta\in \Z[\omega]$$
This equivalence relation is clearly reflexive and symmetric. It is also transitive: if $\alpha I=\beta J$ and $\gamma J=\zeta K$ then $\alpha\gamma I=\alpha \gamma J=\alpha \zeta K$. As we show in (MISSING), the set of equivalence classes is finite, thus we can define \defn{class number} $h_p$ of $\Z[\omega],\omega=e^{2\pi i/p}$ as the size of this equivalence class. We then define

\begin{definition}
A \defn{regular prime} $p$ is a prime such that $p\not \divide h_p$.
\end{definition}

Kummer showed that Lamé's Part I proof can be rectified for regular primes. He also gave a proof for Type II solutions of regular primes. Kummer then went on to classify all the primes less than $100$, and more, showing that the only irregular primes less than $100$ are $37, 59$ and $67$ \cite{Wright}. Since $37^2>1000$, this proves Fermat's Last Theorem for all integers $3\leq n\leq 1000$ except for these three! This is remarkable progress compared to before Kummer's contribution, when Fermat's Last Theorem was only known for multiples of $3,4,5$ and $7$ (CITE).