\chapter{Introduction}\label{sec-introduction}
One of the most infamous theorems of number theory is Fermat's Last Theorem. This theorem was first stated in 1637 by Pierre de Fermat, written in the margins of his copy of \textit{Arithmetica} with the note that he had discovered "a truly marvelous proof of this, which however the margin is not large enough to contain" \cite{Heath}. Fermat never gave a full proof of the theorem, and despite their best efforts, the proof eluded many mathematicians through the centuries until a complete proof was finally published in 1995 by Sir Andrew Wiles. Simon Singh's \textit{Fermat's Last Theorem} \cite{Singh} gives an excellent historical account of the efforts to prove the theorem, in a manner accessible to the non-mathematician. 

Despite the infamous difficulty of its proof, Fermat's Last Theorem is brief to state.
\begin{theorem}[Fermat's Last Theorem]\label{Fermat-Theorem}
The equation
\begin{equation}\label{Fermat}
    x^n+y^n=z^n
\end{equation}
has no integer solutions for $n\in \N,n>2$.
\end{theorem}

When $n=2$, the equation is Pythagoras' Theorem, which is known to have integer solutions. For example, $(x,y,z)=(3,4,5)$ works. 

We now give some reductions of Fermat's Last Theorem, inspired by \cite{Wright}. We may reduce the proof of Fermat's Last Theorem to the case where $n$ is an odd prime, as a counter-example for a composite power $n=pm$ divisible by $p$ also gives a counter-example for the power $p$:
$$(x^m)^p+(y^m)^p=(z^m)^p.$$
Note this will not take care of powers of the form $n=2^n$, so we should additionally prove the theorem for the $n=4$. Luckily, Fermat himself proved this case - in fact it is the only surviving contribution of his to the proof \cite{Fermat-4}. As an additional reduction, we may assume that in a counter-example $(x,y,z)$, $x$, $y$ and $z$ are all coprime. Firstly, note that if two of the three integers are divisible by an integer $m$, then the third must also be divisible by $m$, as otherwise the theorem will not hold $\mod m$. Additionally, letting $g:=gcd(x,y,z)$, any counter-example $(x,y,z)$ gives a counter-example $(\frac{x}{g},\frac{y}{g},\frac{z}{g})$ where all three integers are jointly coprime, hence pairwise coprime by the previous remark. We make a final reduction by classifying the counterexamples for exponent $p$ as one of two types:
\begin{itemize}
    \item (Type I): $p$ does not divide either of $x,y,z$.
    \item (Type II): $p$ divides exactly one of $x,y,z$.
\end{itemize}
We will focus exclusively on Type I solutions in this text. The case $p=3$ can be proven using simple modular arithmetic. It is easy to check exhaustively that the cube of any integer not divisible by $3$ is congruent to $\pm 1\mod 9$, so it follows that $$x^3+y^3\neq z^3 (\Mod 9).$$
We may therefore assume $p>5$. We will start by turning $\cref{Fermat}$ into a multiplicative equation. Let $\omega=e^{2\pi i/p}$, and recall $\omega,\omega^2,\dots,\omega^n=1$ are the $p$ distinct $p$-th roots of unity. Since each solve $t^p-1=0$, and $\C$ is algebraically closed,
\begin{equation}\label{eq-omega-polynomial}t^p-1=(t-1)(t-\omega)\dots (t-\omega^{p-1}).\end{equation}
Letting $t=\frac{-x}{y}$ gives 
\begin{equation}\label{eq-fermat-as-mult}
z^p=x^p+y^p=(x+y)(x+\omega y)\dots (x+\omega^{p-1}y)\end{equation}
as required. This turns Fermat's Last Theorem into an algebraic problem in the ring $\Z[\omega]$ of polynomials in $\omega$ with integer coefficients. In 1847, Gabriel Lamé gave a proof which relied on the assumption that the ring of polynomials $\mathbb{Z}[\omega]$ is a unique factorisation domain (UFD). An integral domain is said to be a UFD if every non-zero, non-unit (that is, non-invertible) element $r\in R$ has a unique decomposition as a finite product of irreducibles $r=r_1r_2\dots r_n$, where an element $s$ is irreducible if $$s=ab\implies a\text{ or } b \text{ is a unit.}$$ Unfortunately, this assumption turns out to be false in general. Nonetheless, let us make the assumption that $\Z[\omega]$ is a UFD and briefly outline Lamé's false proof.

\section{Lamé's false proof for Type I solutions}
We follow the account in \cite{Wright}. The proof splits into two parts.

\textbf{Part I:} By the assumption that $\Z[\omega]$ is a UFD, we can factorise
$$z=r_1r_2\dots r_n$$ such that $$z^p=r_1^p r_2^p\dots r_n^p.$$ Similarly, write $$(x+\omega^jy)=r_{1j}r_{2j}\dots r_{n_j j}.$$ Since this factorisation is assumed to be unique, \cref{eq-fermat-as-mult} implies that for each \newline$1\leq j \leq n, 1\leq s\leq n_j$, we have $r_{sj}=vr_i$ for some $1\leq i\leq n$ and unit $v$. We will additionally show that $x+\omega y$ does not share any irreducibles with $x+\omega^j y$ for all $1<j\leq p$. Indeed, if $r$ is an irreducible dividing both, then $$r|(x+\omega y) - (x+\omega^j y)=\omega y(1-\omega^{j-1}).$$
Since $\omega$ is a unit with inverse $\omega^{p-1}$, it follows that $$r|y(1-\omega^{j-1})|y\prod_{j}(1-\omega^{j}).$$ We show that $\prod_{j}(1-\omega^j)=p$ in \cref{product-of-1-omegak-p}, so it follows that $r|py$. Since $p$,$y$ and $z$ are pairwise coprime, $py$ and $z$ also coprime. By the extended Euclidean algorithm, there exist integers $m,n$ such that $$mpy+nz=1.$$ Then $r$ divides the left-hand side but not the right-hand side, a contradiction.

It follows that no irreducibles divide both $(x-\omega y)$ and $(x-\omega^j y)$ for each \newline$1<j\leq n$. Since the irreducible factors of the $(x-\omega^jy)$'s multiply to give $z^p=r_1^p\dots r_n^p$, the irreducible factors of $(x-\omega y)$ must also appear as powers of $p$.  In fact, we are about to show that 
It follows that \begin{equation}(x-\omega y)=ur_{\alpha(1)}^p\dots r^p_{\alpha(i)}=:u\alpha^p\end{equation} for some unit $u$ and $\alpha\in \Z[\omega]$. Note we have assumed here that $(x-\omega y)$ has at least one irreducible factor, but of course if it is a unit we may still write $(x-\omega y)=u\alpha^p$ for $\alpha=1$. Part II will in fact show this is impossible.

\textbf{Part II:} Under the assumption that $(x-\omega y)=u\alpha^p$, we finish the proof of \cref{Fermat-Theorem}. We note this part does not use the false assumption that $\Z[\omega]$ is a UFD. We will take the following lemma as given -- proving it is surprisingly tricky and requires some sophisticated Galois theory.
\begin{lemma}\label{Kummers-lemma}
For any unit $u\in \Z[\omega]$, $$\frac{u}{\overline{u}}=\omega^k$$ for some $k\in \Z$. 
\end{lemma}
\begin{proof}
Omitted. See \cite{NumberFields}.
\end{proof}

Writing $\alpha=a_0+a_1\omega+\dots+a_{p-1}\omega^{p-1},$ for $a_i\in \Z$ we find by multinomial expansion that $\alpha^p=a_0^p+\dots+a_{p-1}^p \mod p\Z[\omega]$, since every other term in the expansion is a multiple of $p$. Note by equality    $\mod p\Z[\omega]$ we mean they are mapped to the same element under the projection $\Z[\omega]\rightarrow \Z[\omega]/p\Z[\omega]$. It follows that $$(x+\omega y)=ua\mod p\Z[\omega]$$ for some $a\in \Z$. By \cref{Kummers-lemma}, $$(x+\omega y)=\omega^k\overline{(x+\omega y)}=\omega^k(x+\omega^{-1}y) \mod p\Z[\omega]$$ for some $k\in \Z$. By equating coefficients as polynomials in $\omega$, we find $k=1$, so
$$x+\omega y=x\omega+y \mod p\Z[\omega] \implies x=y\mod p.$$
Rewriting Fermat's Last Theorem as $x^p+(-z)^p=(-y)^p$ then also gives $$x=-z\mod p.$$ By Fermat's little theorem, \cref{Fermat-Theorem} reduces to
$$x+y=z\mod p.$$
By what we have shown, this implies
$$3x=0\mod p.$$
This is a contradiction since $p> 3$ and does not divide $x$ by assumption.

\section{Kummer's partial rectification}
Kummer discovered the critical flaw in Part I of Lamé's proof, by showing that $\Z[\omega]$ is \textbf{not} a UFD in general. For example he showed that $\Z[e^{2\pi i/23}]$ is not a UFD \cite{Wright}. Every element of $\Z[\omega]$ \textbf{does} factor as a product of irreducibles, as we will later show by proving $\Z[\omega]$ is \defn{Noetherian}. However, this factorisation is not unique in general. This is a subtle point - in Part I of Lamé's proof we were only able to deduce that $x+\omega y=u\alpha^p$ by arguing that if $r$ is in the decomposition of $x+\omega y$ then it must be in the decomposition of $z^p$ as well. Kummer realised that there was a strictly stronger\footnote{See \cref{remark-strictly-stronger}.} condition to the UFD condition which would make Lamé's Part I proof work. This condition is best explained in the framework of ideals, an invention of Dedekind \cite{Wright}. Dedekind showed that the \defn{ideals} of $\Z[\omega]$ factorise uniquely into \defn{prime ideals}. Prime ideals are proper ideals $I$ of $R$ such that $xy\in I\implies x\in I$ or $y\in I$ where $x,y\in R.$ We now show that Part I of Lamé's proof can be rewritten in terms of ideals. From \cref{eq-fermat-as-mult}, we can take the \defn{ideal generated} by both sides (see \cref{sec-rings}) to get
\begin{equation} \langle z^p\rangle =\langle z\rangle ^p=\langle (x+y)(x+\omega y)\dots(x+\omega^{p-1}y)\rangle=\langle x+y\rangle\dots\langle x+\omega^{p-1}y\rangle
\label{eq-fermat-mult-ideals}
\end{equation}
where we have defined multiplication of ideals as $IJ=\langle ij:i\in I, j\in J\rangle$. Then decomposing $\langle x+\omega y \rangle$ into prime ideals $P_1P_2\dots P_n$ we can use (now correctly) the unique factorisation property to see $\langle x+\omega y\rangle=I^p$ for some ideal $I$, as long as we can establish that $x+\omega y$ does not share any prime ideals with the $x+\omega^j y$ (see \cref{Dedekind-returns-to-type-1}).

The final step will be to show that $I$ is principal: $I=\langle \alpha \rangle$. Then $I^p=\langle \alpha \rangle ^p=\langle \alpha^p \rangle$. Now note $$x+\omega y\in\langle \alpha^p\rangle\implies x+\omega y =s\alpha^p\text{ for some }s\in \Z[\omega],$$ and conversely $$\alpha^p\in \langle x+\omega y\rangle \implies \alpha^p=r(x+\omega y)\text{ for some }r\in \Z[\omega].$$ Then $\alpha^p=rs\alpha^p\implies 1=rs\implies s \text{ is a unit .}$ Note this implication is valid because $\Z[\omega]$ is an integral domain. Therefore, $x+y\omega=u\alpha^p$ for a unit $u$, completing Step I.

When is $I$ principal? Let us define an equivalence relation on ideals $I$ of $\Z[\omega]$ by
$$I\sim J \iff \alpha I=\beta J \text{ for some } \alpha,\beta\in \Z[\omega]$$
This equivalence relation is clearly reflexive and symmetric. It is also transitive: if $\alpha I=\beta J$ and $\gamma J=\zeta K$ then $\alpha\gamma I=\alpha \gamma J=\alpha \zeta K$. As we will show, the set of equivalence classes is a finite abelian group $Cl(\Z[\omega])$, thus we can define \defn{class number} $h_p$ of $\Z[\omega],\omega=e^{2\pi i/p}$ by $h_p=|l(\Z[\omega])|$. We then define
\begin{definition}
A prime $p$ is \defn{regular} if $p\not | h_p$.
\end{definition}
If $p$ is regular, then by Lagrange's Theorem, $Cl(\Z[\omega])$ has no elements of order $p$. In particular, $[I]$ does not have order $p$ where $I$ is the ideal s.t. $x+\omega y=I^p$. As we will show, the identity element in $Cl(\Z[\omega])$ is the equivalence class of principal ideals. It follows that $e=[\langle x+\omega y\rangle]=[I^p]=[I]^p$, and since $[I]$ does not have order $p$, it must be the identity element. In particular, $I$ is principal.

If we can fill in the missing gaps we will have, just as Kummer did in 1847, proved Fermat's Last Theorem for Type I solutions and regular exponents. Kummer also managed to prove that Type 2 solutions could not exist with regular exponents, giving the full proof of \cref{Fermat-Theorem} for regular primes. He then went on to classify all the primes less than $100$, showing that the only irregular primes less than $100$ are $37, 59$ and $67$ \cite{Wright}. Since $37^2>1000$, this proves Fermat's Last Theorem for all integers $3\leq n\leq 1000$ except for these three! This is remarkable progress compared to the prior state of affairs, when Fermat's Last Theorem was only known for multiples of $3,4,5$ and $7$ \cite{History}.
\\ \\
Our goal will be to fill out the gaps in this proof. We need to show two things:

\textbf{Property I:} the ideals of $\Z[\omega]$ factorise uniquely into prime ideals.

\textbf{Property II:} the ideals of $\Z[\omega]$ under $\sim$ form a finite abelian group.

We will start in \cref{sec-rings} by getting comfortable with the algebra of ideals and modules. We will also define what it means for a ring to be Noetherian, and show that non-units in Noetherian rings factorise (not necessarily uniquely) into a finite product of irreducibles. In \cref{sec-number-fields} we will define number fields and number rings, and show that $\Z[\omega]$ is the number ring $\A \cap\Q[\omega]$. In \cref{sec-dedekind} we define a Dedekind domain, show that all number rings are examples, and show that Dedekind domains have Property I. Finally, in \cref{sec-regular-primes}  we show all number rings have Property II, finishing the proof of Fermat's Last Theorem for Type I solutions and regular exponents.