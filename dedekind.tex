\chapter{Dedekind Domains}
After our tour of ideals, Noetherian rings, number fields and number rings, we are now ready to bring everything back to Kummer's partial proof of Fermat's Last Theorem. Recall that 

We are now ready to introduce the notion of a \defn{Dedekind domain}. This ad-hoc-seeming definition turns out to be exactly the condition needed for (FILL OUT), as we will show in section (MISSING). Our main goal for this section is to prove that number rings are Dedekind domains (or only p-th cyclotomics???)

\begin{definition}
(...)
\end{definition}



\begin{proposition}
Any number ring $R=\A\cap \Q[\alpha]$ is a Dedekind domains.
\end{proposition}
\begin{proof}
\begin{itemize}
    \item ($R$ is Noetherian). This was shown in  \cref{number-rings-are-Noetherian}.
    \item Every nonzero prime ideal of $R$ is maximal.
    \item ($R$ is integrally closed )
\end{itemize}
\end{proof}
\section{Number rings are Dedekind domains}

\section{Unique factorisation of ideals}