

\section{Notes}

\subsection{Normal field extensions}
We again follow \cite{NumberFields}
\begin{definition}
A field extension $L$ of $K$ is said to be \defn{normal} if each of the $[L:K]$ embeddings of $L$ restricting to the inclusion on $K$ are automorphisms.
\end{definition}

Note this equivalent to $L$ containing all its conjugates over $K$: if $a\in L$ and $a\sim b$ then there is an embedding of $L$ fixing $K$ such that $a\mapsto b$ -- take any extension, guaranteed by \cref{extend-to-[L:K]-embeddings}, of the embedding $K[a]\rightarrow \C : \sum a_i a^i\mapsto \sum a_i b^i$. Since this embedding is an automorphism, $b\in L$. Since $a,b$ were arbitrary, $L$ contains all its conjugates over $K$. 

Conversely, by \cref{classifying-finite-field-extensions} we can set $L=K[\alpha]$. Every embedding fixing $K$ is entirely determined by a choice of conjugate $\alpha\mapsto \beta$. If $\beta \in L$, then $$\sigma(\sum a_i \alpha^i)=\sum a_i \beta^i\in L$$
so this embedding is an automorphism.


\begin{remark} (BAD PLACE TO PUT THIS REMARK)
In the next theorem, we will consider nested field extensions $K\subset L \subset M$. The inclusion symbol is used somewhat erroniously, in actuality we want to have injective field homomorphisms $K\rightarrow L \rightarrow M$. If $m,n\in M$ are conjugate over $L$, they are necessarily conjugate over $K$: let $f$ be the monic irreducible over $L$ and let $m$ and $n$ satisfy monic irreducibles $g$ and $h$ respectively over $K$. Since $g,h\in \langle f\rangle_L$, $$g=h+jf.$$
Since $g(m)=jf(m)=0, h(m)=0$. However, $g$ was the unique monic irreducible with this property so $g=h$.

The converse, however, is not true. Take for example $\Q\subset \C\subset \C$. $i,-i$ are conjugate over $\Q$ with minimal polynomial $f(x)=x^2+1$, however they are not conjugate over $\C$.
\end{remark}

\begin{theorem}
Every finite field extension $K\subset L$ can be extended to a finite normal field extension $K\subset L \subset M$ (normal over both $K$ and $L$).
\end{theorem}
\begin{proof}
By \cref{classifying-finite-field-extensions} we may write $L=K[\alpha_1]$ for some $\alpha_1 \in L$. Let $\alpha_1,\alpha_2,\dots,\alpha_n$ be the $n=[L:K]$ conjugates of $\alpha_1$ over $K$. Let $M=K[\alpha_1,\alpha_2,\dots,\alpha_n]$ be a field extension of $L$ (and by extension $K$) via the inclusion $L\rightarrow M$. We need to show $M$ contains all its conjugates over $K$, and we will then get the second part for free. Indeed, any embedding of $M$ restricting to the inclusion on $L$ also restricts to the inclusion on $K$ and would therefore be an automorphism.

By construction, $M$ contains all the conjugates of $\alpha_1$. Let $\sigma$ be an embedding of $M$ fixing $K$, and let $x=\sum_i \sum_j a_{ij}\alpha_j^i$ be an element of $M$. $\sigma$ maps $x$ to one of its conjugates, and as $\sigma$ is a field homomorphism, $\sigma(x)=\sum_i \sum_j a_{ij}\sigma(\alpha_j)^i\in M$. Therefore $M$ contains all its conjugates over $K$.
\end{proof}
\subsection{The cyclotomic fields $\Q[\omega]$}
For proving Fermat's Last Theorem, we will be interested in understanding the $p$-th cyclotomic fields. 

 Our approach follows that given in \cite{NumberFields} chapter 2.

Now we want to study the $m^{th}$ cyclotomic number fields $\Q[\omega]$. If two numbers $\alpha,\beta\in \mathbb{C}$ satisfy the same monic polynomial over a subfield $K$, we say they are \defn{conjugate}. The following Theorem and its Corollary are proven in \cite{NumberFields}. We give adapted proofs filling in a few missing details.
\begin{theorem}\label{conjugates-of-Q[w]}
In the $m^{th}$ cyclotomic number fields $\Q[\omega]$, $\omega^k, 1\leq k\leq m, (k,m)=1$ are exactly the conjugates to $\omega$. 
\end{theorem}
\begin{proof}
Let $f$ be the monic irreducible polynomial of $\omega$. (NOTE: have I proven that $\omega$ is algebraic over $\Q$? or in fact that the same is true for any complex \#?) Since $\omega$ is a root of $x^m-1$ and $f$ is irreducible (minimal), $f$ divides $x^m-1$:
$$x^m-1=f(x)g(x)$$
for some $g(x)$. By \cref{lemma-monics-are-Z}, both $f$ and $g$ are in $\Z[x]$.

For our proof, it is enough to show that for every $k$ with $(k,m)=1$ and every prime $p$ not dividing $m$, $\omega^{kp}$ is conjugate to $\omega^k=:\theta$. If we show this, then the general result follows inductively. I.e. if $k$ is s.t. $(k,m)=1$ and the prime decomposition of $k$ is $k=p_1^{a_1}p_2^{a_2}\dots p_n^{a_n}$ then
$$\omega\sim\omega^{p_1}\sim\omega^{p_1^2}\sim\dots\sim\omega^{p_1^{a_1}}\sim\omega^{p_1^{a_1}p_2}\sim\dots \omega^{p_1^{a_1}p_2^{a_2}\dots p_n^{a_n}}.$$

We want to show that $f(\theta^p)=0$. Suppose for sake of contradiction that $g(\theta^p)=0$. We get another monic polynomial $g^*(x):=g(x^p)$ over $\mathbb{Z}$ which $\theta$ is a root of. Therefore $g^*=fh$ for some $h$ which again will be in $\mathbb{Z}$ by \cref{lemma-monics-are-Z}. Now we may reduce all coefficients mod $p$ to get monics $\overline{g^*}=\bar{f}\bar{h}$. By multinomial expansion, $\overline{g}(x^p)=\bar{g}(x)^p$ since $$(\sum_{i=0}^n a_ix^i)^p=\sum_{k_1+k_2+\dots+k_n=p}{p\choose k_1,k_2,\dots ,k_n}\prod_{i=0}^n(a_ix^i)^{k_i}$$
The multinomial coefficient $${p\choose k_1,k_2,\dots ,k_n}=\frac{p!}{k_1!k_2!\dots k_n!}$$
will be a product of $p$ (and hence be $0$ in $\Z_p$) iff none of the $k_i$ equal $p$. Conversely, if $k_i=p$ then $k_j=0$ for $j\neq i$. Therefore in $\Z_p$,
$$\overline{g}(x)^p=(\sum_{i=0}^n a_ix^i)^p=\sum_{i=0}^n(a_ix^i)^p=\sum_{i=0}^n a_i^p (x^p)^i=\sum_{i=0}^n a_i (x^p)^i=\overline{g}(x^p)$$
where the second-to-last equality follows by the fact that the multiplicative group of units $\Z_p^{\times}$ is cyclic of order $p-1$. In this argument we have implicitly used that reducing coefficents $\mod p$ is a ring homomorphism, so $\overline{\sum a_ix^i}=\sum \overline{a_i}x^i.$

We have that $(\overline{g})^p=\overline{f}\overline{h}$. By \cref{Z_p[x]-is-UFD}, $\Z_p$ is a UFD, so both sides of the equation have the same factorisation into irreducible elements. This gives an irreducible $j\in Z_p[x]$ that divides both $\overline{g}$ and $\overline{f}$. Therefore $j^2|\overline{f}\overline{g}=x^m-1$. Writing $x^m-1=j^2k$ for $k\in \Z_p[x]$ and taking derivatives of both sides gives $\overline{m}x^m=2jj'k+j^2k'$ such that $j|\overline{m}x^m$ in $\Z_p$ (remember this trick - we will use it again). Since $(m,p)=1$, $\overline{m}\neq 0$. Since $\overline{m}x^m$ has factorisation $\overline{m}x^m=\overline{m}x\cdot x\cdot \dots \cdot x$, and since $\Z_p$ is a UFD, in fact $j$ must be a monomial itself. However $j|x^m-1$, a contradiction.
\end{proof}

\begin{corollary}\label{Q[w]-degree}
$[\Q[\omega]:\Q]=\varphi(m)$, where $\varphi(m)$ is the number of integers $1\leq k\leq m$ with $(k,m)=1$.
\end{corollary}

\begin{proof} Let $f$ be the monic irreducible polynomial of $\omega$. $\Q[\omega]\iso \Q[x]/\langle f \rangle$ so its degree is the degree of $f$. By \cref{monic-irreducible-has-all-roots}.
\end{proof}

\subsection{Prequisites}
Rings are commutative with $1$.
\subsubsection{The field of fractions}
We can construct $\mathbb{Q}$ from $\Z$ as the set of equivalence classes $$\Z\times\Z / \sim $$
$$(m,n)\sim (a,b)\iff mb=an$$
This carries a field structure inspired by the field structure of $\mathbb{Q}$. Moreover, this structure works for any integral domain, giving us the \defn{field of fractions} $K$ of $R$. $R$ sits inside $K$ as a subring. $K$ is the smallest field containing $R$.

We can also define $\mathbb{R}$ from $\mathbb{Q}$ by noticing that every real number is defined uniquely by which rationals are $\leq$ than it. So $r\in \mathbb{R}$ can be identified with $\{q\in \mathbb{Q}:q\leq r\}$. The set of all such "left"-intervals in $\mathbb{Q}$ are the Dedekind cuts of $\mathbb{Q}$. They have a field structure and are isomorphic to $\mathbb{R}$. You can imagine doing this with any ordered field.

\subsubsection{Ideals}
\defn{Ideals} are subsets of a ring $R$ that are closed under $-$ and multiplication by $r\in R$. They are automatically abelian groups. Ideals are the same thing as subsets that are also R-modules. \defn{Principal ideals} $\langle r \rangle$ are ideals generated by $r\in R$. If every ideal is a principal ideal, then $R$ is a \defn{principal ideal domain}. For general sets, $\langle S \rangle$ is the ideal of finite "linear combinations" of elements in $S$. \defn{Finitely generated ideals} are ideals generated by some finite set $S$.

We can give ideals a multiplication structure and an addition structure: $I+ J=\langle I\cup J\rangle$ and $IJ=\langle \{ij\} \rangle$. We have an additive identitity $\{0\}$ and multiplicative identity $R$, and we have both associativity and distributivity. However we do not have additive or multiplicative inverses nor commutativity. However, it is possible \textit{in some cases} to extend the set of ideals to a field,  like how we can extend $\mathbb{Z}$ to the field $\mathbb{R}$. (...)

\section{Week of Jan 3rd}
\subsection{Dedekind domains}
\begin{definition}
A ring R is a Dedekind domain if
\begin{enumerate}
    \item R is Noetherian (every prime ideal is a finitely generated R-module)
    \item Every nonzero prime ideal of R is maximal (not contained in an ideal $\neq$ $R$)
    \item $R$ is integrally closed (every $x$ in the field of fractions of $R$ satisfying a monic polynomial with integer coefficients lies in $R$.)
\end{enumerate}
\end{definition}
\begin{definition}
An ideal $I\neq R$ is prime if $xy\in I\implies x\in I$ or $y\in I$.
\end{definition}

\begin{prop}
$I\subset R$ is a prime ideal iff $R/I$ is an integral domain (ring with no zero divisors).
\end{prop}
\begin{proof}
$(\implies)$ A ring can only have zero divisors if both are non-zero. Therefore suppose $r,s\in R\setminus{I}$ are such that $s+I\neq r+I$. But if $rs+I=I$, then since $I$ is a prime ideal, either $r\in I$ or $s\in I$, a contradiction.

$(\impliedby)$ Let $r,s\in R$ be non-zero, such that $rs\in I$. Since $R/I$ is an integral domain, either $r$ or $s$ is in $I$, so $I$ is a prime ideal.
\end{proof}

Strong statement to prove: 
\begin{theorem}
If for any Dedekind domain $A$ with field of fractions $K$, and $L$ is any finite field extention of $K$ with $B$ the integral closure of $A$ in $L$, then $B$ is a Dedekind domain.
\end{theorem}

I.e. Dedekind domains are stable under field extensions.

To show $x$ is integral over $A$, i.e. satisfies a monic polynomial with $A$ coefficients, it is often easier to prove the equivalent statement $xM\subset M$ for some fin. gen. $A-$mod, or that $A[x]$ is fin. gen.

$R$ is a finitely generated ring over $A$ if $R=A[r_1,\dots,r_n]$. This is \textbf{not} the set $\{a_0r_1+\dots+a_nr_n:a_i\in A\}$, but the polynomials in $r_1,\dots,r_n$, i.e. powers and multiples of these are also included.
\begin{prop}
If $A\subset B=A[b_1,\dots,b_n]$ is a subring and $B$ is integral over $A$ (every $b$ satisfies a monic polynomial in $A$ coefficients), then $B=\langle1,b,b^2,\dots,b^n\rangle_A$ - a finitely generated $A-$module. 
\end{prop}
\begin{proof}
Assume $n=1$ and do induction. $b$ satisfies a monic polynomial $f\in A[X]$. By the euclidean algorithm, any $g\in A[X]$ is s.t. $g(x)=f(x)q(x)+r(x)$ where $deg(r)<deg(f)$. Therefore $g(b)=r(b)$, so there are no polynomials in $A[b]$ with degree $\geq deg(f)$.
\end{proof}
This can be used to prove that "integral over" is transitive, but I need to understand this proof.

\begin{definition}
The algebraic integers $\mathbb{A}\subset \mathbb{C}$ is all complex numbers that are integral over $\mathbb{Z}$.
\end{definition}

\begin{definition}
A \defn{number ring} $B$ is $B=\mathbb{A}\cap \mathcal{L}$, where $\mathbb{A}\subset \mathbb{C}$ is the algebraic integers and $\mathcal{L}$ is a finite field extension\footnote{a field that is also a finite vector space over $\mathbb{Q}$} of $\mathbb{Q}$, i.e. $\mathbb{Q}[w]$ for $w=e^{2\pi i/p}$.
\end{definition}

Sanity check: $\mathbb{Q}[w]$ is a field. What is the inverse of $a_px^p+a_{p-1}x^{p-1}\dots a_0$?

\begin{question}
Why is $B=\mathbb{A}\cap \mathcal{L}$ the integral closure of $\mathbb{Z}$ in $\mathcal{L}$?
\end{question}
\begin{proof}
You would think that $\mathbb{Q}$ would be in $\overline{Z}$, since $b\frac{a}{b}-a=0$. But this is NOT a monic, as the leading coefficient is not $b$. So no! The other answer is that $\mathbb{A}$ is not the complex integers, but the complex numbers that are integral over $\mathbb{Z}$ so the definition is really rigged.
\end{proof}

\begin{question}
Why can we be so sure that $\mathcal{L}\subset \mathbb{C}$? Surely there must be field extensions that lie outside of $\mathbb{C}$.
\end{question}
Maybe the answer is that every field extension is a subset of $\mathbb{C}$, as a consequence of the non-fieldness of $\mathbb{H}$?

This fact, together with the following lemma, shows the field of fractions of $B$ is $\mathcal{L}$. ($A=\mathbb{Z}$, $K=\mathbb{Q}$, $L=\mathcal{L}$, $B=B$.)
\begin{lemma}
Let $A$ be a ring with f.f. $K$ and $L$ be an alg. ext. of $K$. Then the field of fractions of the integral closure $B$ of $A$ in $L$ is $L$.
\end{lemma}

\section{Week of Jan 17th}

\begin{lemma}
A ring $R$ is a field iff it contains no trivial ideals.
\end{lemma}
\begin{proof}
$(\implies)$ if $I$ is an ideal of $R$, then for any non-zero $x\in I$, and any $r\in R$, $r=(rx^{-1})x\in I$, so $I=R$.

$(\impliedby)$ If $R$ contains no non-trivial ideals, then for any $r\in R$, $\langle r\rangle =R$, so $\exists s\in R \text{ such that } rs=1$.
\end{proof}

\begin{lemma}
An ideal $I$ of $R$ is maximal iff $I/R$ is a field.
\end{lemma}
\begin{proof}
$(\impliedby)$ Suppose $R/I$ is a field, and $J$  is an ideal with $I\subset J$. Then $J/I:=\{jI: j\in J\}$ is an ideal of $R/I$. By the previous lemma, $J/I=R/I$, so $J=R$.

$(\implies)$ Conversely, if $I$ is maximal, then $R/I$ does not contain any non-trivial ideals: a non-trivial ideal $JI\subset R/I$ gives rise to a non-trivial ideal $\langle J\cup I\rangle$ of $R$. By the previous lemma, $R/I$ is a field.
\end{proof}

\begin{corollary}
Maximal ideals are prime.
\end{corollary}
\begin{proof}
If $I$ is a maximal ideal, then $R/I$ is a field, hence an integral domain, hence $I$ is prime.

\url{https://math.stackexchange.com/questions/68489/why-are-maximal-ideals-prime}
\end{proof}