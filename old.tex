\section{Notes}
\subsection{Prequisites}
Rings are commutative with $1$.
\subsubsection{The field of fractions}
We can construct $\mathbb{Q}$ from $\Z$ as the set of equivalence classes $$\Z\times\Z / \sim $$
$$(m,n)\sim (a,b)\iff mb=an$$
This carries a field structure inspired by the field structure of $\mathbb{Q}$. Moreover, this structure works for any integral domain, giving us the \defn{field of fractions} $K$ of $R$. $R$ sits inside $K$ as a subring. $K$ is the smallest field containing $R$.

We can also define $\mathbb{R}$ from $\mathbb{Q}$ by noticing that every real number is defined uniquely by which rationals are $\leq$ than it. So $r\in \mathbb{R}$ can be identified with $\{q\in \mathbb{Q}:q\leq r\}$. The set of all such "left"-intervals in $\mathbb{Q}$ are the Dedekind cuts of $\mathbb{Q}$. They have a field structure and are isomorphic to $\mathbb{R}$. You can imagine doing this with any ordered field.

\subsubsection{Ideals}
\defn{Ideals} are subsets of a ring $R$ that are closed under $-$ and multiplication by $r\in R$. They are automatically abelian groups. Ideals are the same thing as subsets that are also R-modules. \defn{Principal ideals} $\langle r \rangle$ are ideals generated by $r\in R$. If every ideal is a principal ideal, then $R$ is a \defn{principal ideal domain}. For general sets, $\langle S \rangle$ is the ideal of finite "linear combinations" of elements in $S$. \defn{Finitely generated ideals} are ideals generated by some finite set $S$.

We can give ideals a multiplication structure and an addition structure: $I+ J=\langle I\cup J\rangle$ and $IJ=\langle \{ij\} \rangle$. We have an additive identitity $\{0\}$ and multiplicative identity $R$, and we have both associativity and distributivity. However we do not have additive or multiplicative inverses nor commutativity. However, it is possible \textit{in some cases} to extend the set of ideals to a field,  like how we can extend $\mathbb{Z}$ to the field $\mathbb{R}$. (...)

\section{Week of Jan 3rd}
\subsection{Dedekind domains}
\begin{definition}
A ring R is a Dedekind domain if
\begin{enumerate}
    \item R is Noetherian (every prime ideal is a finitely generated R-module)
    \item Every nonzero prime ideal of R is maximal (not contained in an ideal $\neq$ $R$)
    \item $R$ is integrally closed (every $x$ in the field of fractions of $R$ satisfying a monic polynomial with integer coefficients lies in $R$.)
\end{enumerate}
\end{definition}
\begin{definition}
An ideal $I\neq R$ is prime if $xy\in I\implies x\in I$ or $y\in I$.
\end{definition}

\begin{prop}
$I\subset R$ is a prime ideal iff $R/I$ is an integral domain (ring with no zero divisors).
\end{prop}
\begin{proof}
$(\implies)$ A ring can only have zero divisors if both are non-zero. Therefore suppose $r,s\in R\setminus{I}$ are such that $s+I\neq r+I$. But if $rs+I=I$, then since $I$ is a prime ideal, either $r\in I$ or $s\in I$, a contradiction.

$(\impliedby)$ Let $r,s\in R$ be non-zero, such that $rs\in I$. Since $R/I$ is an integral domain, either $r$ or $s$ is in $I$, so $I$ is a prime ideal.
\end{proof}

Strong statement to prove: 
\begin{theorem}
If for any Dedekind domain $A$ with field of fractions $K$, and $L$ is any finite field extention of $K$ with $B$ the integral closure of $A$ in $L$, then $B$ is a Dedekind domain.
\end{theorem}

I.e. Dedekind domains are stable under field extensions.

To show $x$ is integral over $A$, i.e. satisfies a monic polynomial with $A$ coefficients, it is often easier to prove the equivalent statement $xM\subset M$ for some fin. gen. $A-$mod, or that $A[x]$ is fin. gen.

$R$ is a finitely generated ring over $A$ if $R=A[r_1,\dots,r_n]$. This is \textbf{not} the set $\{a_0r_1+\dots+a_nr_n:a_i\in A\}$, but the polynomials in $r_1,\dots,r_n$, i.e. powers and multiples of these are also included.
\begin{prop}
If $A\subset B=A[b_1,\dots,b_n]$ is a subring and $B$ is integral over $A$ (every $b$ satisfies a monic polynomial in $A$ coefficients), then $B=\langle1,b,b^2,\dots,b^n\rangle_A$ - a finitely generated $A-$module. 
\end{prop}
\begin{proof}
Assume $n=1$ and do induction. $b$ satisfies a monic polynomial $f\in A[X]$. By the euclidean algorithm, any $g\in A[X]$ is s.t. $g(x)=f(x)q(x)+r(x)$ where $deg(r)<deg(f)$. Therefore $g(b)=r(b)$, so there are no polynomials in $A[b]$ with degree $\geq deg(f)$.
\end{proof}
This can be used to prove that "integral over" is transitive, but I need to understand this proof.

\begin{definition}
The algebraic integers $\mathbb{A}\subset \mathbb{C}$ is all complex numbers that are integral over $\mathbb{Z}$.
\end{definition}

\begin{definition}
A \defn{number ring} $B$ is $B=\mathbb{A}\cap \mathcal{L}$, where $\mathbb{A}\subset \mathbb{C}$ is the algebraic integers and $\mathcal{L}$ is a finite field extension\footnote{a field that is also a finite vector space over $\mathbb{Q}$} of $\mathbb{Q}$, i.e. $\mathbb{Q}[w]$ for $w=e^{2\pi i/p}$.
\end{definition}

Sanity check: $\mathbb{Q}[w]$ is a field. What is the inverse of $a_px^p+a_{p-1}x^{p-1}\dots a_0$?

\begin{question}
Why is $B=\mathbb{A}\cap \mathcal{L}$ the integral closure of $\mathbb{Z}$ in $\mathcal{L}$?
\end{question}
\begin{proof}
You would think that $\mathbb{Q}$ would be in $\overline{Z}$, since $b\frac{a}{b}-a=0$. But this is NOT a monic, as the leading coefficient is not $b$. So no! The other answer is that $\mathbb{A}$ is not the complex integers, but the complex numbers that are integral over $\mathbb{Z}$ so the definition is really rigged.
\end{proof}

\begin{question}
Why can we be so sure that $\mathcal{L}\subset \mathbb{C}$? Surely there must be field extensions that lie outside of $\mathbb{C}$.
\end{question}
Maybe the answer is that every field extension is a subset of $\mathbb{C}$, as a consequence of the non-fieldness of $\mathbb{H}$?

This fact, together with the following lemma, shows the field of fractions of $B$ is $\mathcal{L}$. ($A=\mathbb{Z}$, $K=\mathbb{Q}$, $L=\mathcal{L}$, $B=B$.)
\begin{lemma}
Let $A$ be a ring with f.f. $K$ and $L$ be an alg. ext. of $K$. Then the field of fractions of the integral closure $B$ of $A$ in $L$ is $L$.
\end{lemma}

\section{Week of Jan 17th}

\begin{lemma}
A ring $R$ is a field iff it contains no trivial ideals.
\end{lemma}
\begin{proof}
$(\implies)$ if $I$ is an ideal of $R$, then for any non-zero $x\in I$, and any $r\in R$, $r=(rx^{-1})x\in I$, so $I=R$.

$(\impliedby)$ If $R$ contains no non-trivial ideals, then for any $r\in R$, $\langle r\rangle =R$, so $\exists s\in R \text{ such that } rs=1$.
\end{proof}

\begin{lemma}
An ideal $I$ of $R$ is maximal iff $I/R$ is a field.
\end{lemma}
\begin{proof}
$(\impliedby)$ Suppose $R/I$ is a field, and $J$  is an ideal with $I\subset J$. Then $J/I:=\{jI: j\in J\}$ is an ideal of $R/I$. By the previous lemma, $J/I=R/I$, so $J=R$.

$(\implies)$ Conversely, if $I$ is maximal, then $R/I$ does not contain any non-trivial ideals: a non-trivial ideal $JI\subset R/I$ gives rise to a non-trivial ideal $\langle J\cup I\rangle$ of $R$. By the previous lemma, $R/I$ is a field.
\end{proof}

\begin{corollary}
Maximal ideals are prime.
\end{corollary}
\begin{proof}
If $I$ is a maximal ideal, then $R/I$ is a field, hence an integral domain, hence $I$ is prime.

\url{https://math.stackexchange.com/questions/68489/why-are-maximal-ideals-prime}
\end{proof}